
%-------------------------------------------------------------------------------
% Dokumenten Klasse
\documentclass[
	final,
	a4paper,
	oneside,
	parskip=full,
	headings=standardclasses,
	headings=big,
	pointednumbers
]{scrartcl}

%-------------------------------------------------------------------------------
% Packete nutzen
\usepackage[T1]{fontenc}
\usepackage[utf8]{inputenc}
\usepackage[ngerman]{babel}
\usepackage[left=20mm,right=20mm,top=10mm,bottom=25mm]{geometry}
\usepackage{amsmath}
\usepackage{amssymb}
\usepackage{mathtools}

%-------------------------------------------------------------------------------
% Für enumerate
\usepackage{enumitem}

%-------------------------------------------------------------------------------
% Andere Schriftart
\usepackage{lmodern}

%-------------------------------------------------------------------------------
% Color
\usepackage{color}
\usepackage{xcolor}

% Farbige Tabellen (\rowcolor)
\usepackage{colortbl}

\definecolor{white}{rgb}{1,1,1}
\definecolor{darkred}{rgb}{0.3,0,0}
\definecolor{darkgreen}{rgb}{0,0.3,0}
\definecolor{darkblue}{rgb}{0,0,0.3}
\definecolor{pink}{rgb}{0.78,0.09,0.51}
\definecolor{purple}{rgb}{0.28,0.24,0.55}
\definecolor{orange}{rgb}{1,0.6,0.0}
\definecolor{grey}{rgb}{0.4,0.4,0.4}
\definecolor{lightgrey}{rgb}{0.85,0.85,0.85}
\definecolor{rechungsgrau}{rgb}{0.95, 0.95, 0.95}
\definecolor{aquamarine}{rgb}{0.4,0.8,0.65}
\definecolor{backgroundgreen}{rgb}{0.2, 0.8, 0.6}
\definecolor{backgroundred}{rgb}{1.0, 0.6, 0.6}
\definecolor{fazitgreen}{rgb}{0.0 0.6 0.2}


%-------------------------------------------------------------------------------
% Paragraph settings
\setlength{\parindent}{0em} 
\setlength{\parskip}{0.3em} 

%-------------------------------------------------------------------------------
% TikZ
\usepackage{tikz}
%\usetikzlibrary{positioning, arrows, decorations}
\usetikzlibrary{
    trees,
    shapes,
    arrows,
    backgrounds,
    positioning,
    fit,
    petri,
    decorations,
    decorations.pathmorphing,
    decorations.pathreplacing
}

\newcommand{\tikzcircle}[1]{\; \tikz[overlay, remember picture, anchor=base] \node[draw,circle,minimum size=4mm,inner sep=0pt, fill=black!20] {$#1$}; \;}

\newcommand{\tikzmark}[1]{\tikz[overlay,remember picture,baseline=(#1.base)] \node (#1) {\strut};}

\newcommand{\txb}[1]{{\color{blue}#1}}
\newcommand{\txr}[1]{{\color{red}#1}}
\newcommand{\txo}[1]{{\color{orange}#1}}
\newcommand{\txgr}[1]{{\color{grey}#1}}

\begin{document}
	
    \section*{Deskriptive Entscheidungstheorie}


    \begin{enumerate}[label=$\bullet$]
        \setlength{\parskip}{1mm}
        \item Intuitive Entscheidungen
        \item Rationale Entscheidungen
        \item Mentale Modelle
    \end{enumerate}

	\subsection*{Rationales Entscheiden}
        Eine Entscheidung ist rational, wenn der Entscheidungsprozess folgende
        Merkmale aufweist:

        \begin{enumerate}[label=$\bullet$]
            \setlength{\parskip}{1mm}
            \item Der \txgr{Entscheidungsprozess} ist durchgängig \txb{zielgerichtet} und
                  orientiert sich konsequent an \txr{Zielen}.
            \item Die im \txgr{Entscheidungsprozess} angestellten Überlegungen basieren
                  auf möglichst \txb{objektiven} Informationen.
            \item Der \txgr{Entscheidungsprozess} folgt einem systematischen Vorgehen
                  und verwendet \txb{klare methodische} Regeln, die für Nichtbeteiligte
                  \txb{nachvollziehbar} sind.
        \end{enumerate}
	
	\subsection*{Intuitives Entscheiden}
    
        -

	\subsection*{Mentales Modell}

        Ein mentales Modell ist eine vereinfachte mentale Abbildung
        der subjektiv wahrgenommenen Aussenwelt mit dem Zweck

        \begin{enumerate}[label=$\bullet$]
            \setlength{\parskip}{1mm}
            \item die durch die Sinnesorgane aufgenommene
                  grosse Informationsmenge zu bewältigen
            \item und um Entscheidungen schnell fällen zu können.
        \end{enumerate}
    
        Beispiele

        \begin{enumerate}[label=$\bullet$]
            \setlength{\parskip}{1mm}
            \item Duschwasser einstellen
            \item Strasse überqueren
            \item Schuhe binden
        \end{enumerate}

        Diese mentalen Prozessmodelle sind

        \begin{enumerate}[label=$\bullet$]
            \setlength{\parskip}{1mm}
            \item unbewusst,
            \item mittels Erfahrung gewonnen
            \item und werden sofort angepasst.
        \end{enumerate}

        Mentale Modelle zur Alltagsbewältigung

        \begin{enumerate}[label=$\bullet$]
            \setlength{\parskip}{1mm}
            \item Informationsverarbeitung
            \item Effizienzsteigerung
            \item Notwendig um komplexe Sachverhalte zu interpretieren
        \end{enumerate}

	\subsection*{Gefahr Mentale Modelle}
    
        -

	\subsection*{Superzeichen}
    
        Lernen heisst, Superzeichen zu bilden. Lesen wird effizienter, wenn Superzeichen
        als solche erkannt und interpretiert werden. Gelegentlich geschieht das vorschnell
        und führt dann zu Problemen.

        Ent-Lernen bedeutet, die Superzeichenbildung wieder rückgängig zu machen und
        wieder langsam zu lesen.

	\subsection*{Dynamik mentaler Modelle}

        -

	\subsection*{Mentale Modelle und Vorurteile}

        -

	\subsection*{Scheinsicherheit - das Ziegenproblem}

        -

	\subsection*{Kausalität vs. Korrelation}

        \minisec{Korrelation}
            Zwei Grössen sind korreliert, wenn sich statistisch eine mathematische
            Abhängigkeit nachweisen lässt.

        \minisec{Kausalität}
            Zwischen zwei Grössen besteht ein kausaler Zusammenhang, wenn
            zwischen Ihnen eine Ursache-Wirkungsbeziehung besteht.

	\subsection*{Problematik mentaler Modelle}


        \begin{enumerate}[label=\arabic*.]
            \setlength{\parskip}{1mm}
            \item Aberglaube (Ticket-Automat im Bus)
            \item Vorurteile (Rosenhan-Experiment)
            \item Mentale Modelle können eine
                  Eigendynamik entwickeln (selektive
                  Suche nach Bestätigung)
            \item Konservativismus: Mentale Modelle entwickeln sich mit der
                  Erfahrung
            \item Scheinsicherheit: Wenn mentale Modelle nicht als solche
                  wahrgenommen werden, werden sie nicht hinterfragt
        \end{enumerate}

        Problematik entschärfen: mentale Modelle plausibilisieren



        \begin{enumerate}[label=\arabic*.]
            \setlength{\parskip}{1mm}
            \item sich der eigenen mentalen Modelle bewusst werden
            \item mentale Modelle explizit machen
            \item kritisches Hinterfragen
            \item sich über mentale Modelle austauschen
        \end{enumerate}

        Mentale Modelle lassen sich \txr{nicht verifizieren}!

	\subsection*{Falsifikationismus}

        Damit eine Aussage wissenschaftlich ist, muss sie falsifizierbar sein.

        Bei Eintreffen eines bestimmten experimentellen Befunds wird die
        Hypothese verworfen.

        Hypothese: "`Alle Schwäne sind weiss"'
        
        Noch so viele weisse Schwäne beweisen die Aussage nicht.

        Wissenschaftliche Arbeit:\\
        Suche nach einem allfälligen schwarzen Schwan.

	\subsection*{Mentale Modelle: Erkenntnisse}

        \begin{enumerate}[label=$\bullet$]
            \setlength{\parskip}{1mm}
            \item Mentale Modelle sind \txb{überlebenswichtig}.
            \item Mentale Modelle sind \txb{Superzeichen}.
            \item Sie können zu vorschnellen Schlüssen verleiten.
            \item Mentale Modelle sind i.a. \txb{implizit}. Das macht sie so gefährlich.
            \item Wenn immer möglich: \txb{genau hinschauen}.
        \end{enumerate}
        


\end{document}