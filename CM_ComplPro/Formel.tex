
%-------------------------------------------------------------------------------
% Dokumenten Klasse
\documentclass[
	final,
	a4paper,
	oneside,
	parskip=full,
	headings=standardclasses,
	headings=big,
	pointednumbers
]{scrartcl}

%-------------------------------------------------------------------------------
% Packete nutzen
\usepackage[T1]{fontenc}
\usepackage[utf8]{inputenc}
\usepackage[ngerman]{babel}
\usepackage[left=20mm,right=20mm,top=10mm,bottom=25mm]{geometry}
\usepackage{amsmath}
\usepackage{amssymb}
\usepackage{mathtools}

%-------------------------------------------------------------------------------
% Andere Schriftart
\usepackage{lmodern}

%-------------------------------------------------------------------------------
% Color
\usepackage{color}
\usepackage{xcolor}

% Farbige Tabellen (\rowcolor)
\usepackage{colortbl}

\definecolor{white}{rgb}{1,1,1}
\definecolor{darkred}{rgb}{0.3,0,0}
\definecolor{darkgreen}{rgb}{0,0.3,0}
\definecolor{darkblue}{rgb}{0,0,0.3}
\definecolor{pink}{rgb}{0.78,0.09,0.51}
\definecolor{purple}{rgb}{0.28,0.24,0.55}
\definecolor{orange}{rgb}{1,0.6,0.0}
\definecolor{grey}{rgb}{0.4,0.4,0.4}
\definecolor{lightgrey}{rgb}{0.85,0.85,0.85}
\definecolor{rechungsgrau}{rgb}{0.95, 0.95, 0.95}
\definecolor{aquamarine}{rgb}{0.4,0.8,0.65}
\definecolor{backgroundgreen}{rgb}{0.2, 0.8, 0.6}
\definecolor{backgroundred}{rgb}{1.0, 0.6, 0.6}
\definecolor{fazitgreen}{rgb}{0.0 0.6 0.2}


%-------------------------------------------------------------------------------
% Paragraph settings
\setlength{\parindent}{0em} 
\setlength{\parskip}{0.3em} 

%-------------------------------------------------------------------------------
% TikZ
\usepackage{tikz}
%\usetikzlibrary{positioning, arrows, decorations}
\usetikzlibrary{
    trees,
    shapes,
    arrows,
    backgrounds,
    positioning,
    fit,
    petri,
    decorations,
    decorations.pathmorphing,
    decorations.pathreplacing
}

\newcommand{\tikzcircle}[1]{\; \tikz[overlay, remember picture, anchor=base] \node[draw,circle,minimum size=4mm,inner sep=0pt, fill=black!20] {$#1$}; \;}

\newcommand{\tikzmark}[1]{\tikz[overlay,remember picture,baseline=(#1.base)] \node (#1) {\strut};}


\begin{document}
	
    
    \begin{align*}
        & \int_{0}^{90} 15 &-& \int_{90}^{165} 10 &+& \int_{165}^{300} 2.5 &-& \int_{300}^{420} 5 \\
        & 90 \cdot 15      &-& 75 \cdot 10        &+& 135 \cdot 2.5        &-& 120 \cdot 5 \\
        & 1350             &-& 750                &+& 337.5                &-& 600 \\
        & \int_{0}^{90} 15 &-& \int_{90}^{165} 9  &+& \int_{165}^{300} 2.5 &-& \int_{300}^{420} 6 \\
        & 90 \cdot 15      &-& 75 \cdot 9         &+& 135 \cdot 2.5        &-& 120 \cdot 6 \\
        & 1350             &-& 675                &+& 337.5                &-& 720 
    \end{align*}

    \begin{align*}
        \tilde x            &= (25; 28; 4; 28; 19; 3; 9; 17; 29; 29) \\
        x                   &= (3;4;9;17;19;25;28;28;29;29) \\
        n                   &= 10 \\ 
        x_{0{,}25}          &= x_{\lfloor n \cdot 0{,}25 + 1\rfloor}
                             = x_{\lfloor 2{,}5 + 1\rfloor}
                             = x_{\lfloor 3{,}5\rfloor}
                             = x_3 = 9 \\
        x_{0{,}75}          &= x_{\lfloor n \cdot 0{,}75 +1\rfloor}
                             = x_{\lfloor 7{,}5 +1\rfloor}
                             = x_{\lfloor 8{,}5\rfloor}
                             = x_8 = 28 \\
        \operatorname{IQR}  &= x_{0{,}75} - x_{0{,}25}
                             = 28-9 = 19
    \end{align*}

    \begin{align*}
        f(x)     &= 5x + \sin(x) \\
        f(x,y)   &= 3x^2+y+y^3 \\
        f(x,y,z) &= y^{2}+z-x^{2}
    \end{align*}
	
\end{document}