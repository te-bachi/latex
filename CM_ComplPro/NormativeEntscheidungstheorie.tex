
%-------------------------------------------------------------------------------
% Dokumenten Klasse
\documentclass[
	final,
	a4paper,
	oneside,
	parskip=full,
	headings=standardclasses,
	headings=big,
	pointednumbers
]{scrartcl}

%-------------------------------------------------------------------------------
% Packete nutzen
\usepackage[T1]{fontenc}
\usepackage[utf8]{inputenc}
\usepackage[ngerman]{babel}
\usepackage[left=20mm,right=20mm,top=10mm,bottom=25mm]{geometry}
\usepackage{amsmath}
\usepackage{amssymb}
\usepackage{mathtools}

%-------------------------------------------------------------------------------
% Für enumerate
\usepackage{enumitem}

%-------------------------------------------------------------------------------
% Andere Schriftart
\usepackage{lmodern}

%-------------------------------------------------------------------------------
% Color
\usepackage{color}
\usepackage{xcolor}

% Farbige Tabellen (\rowcolor)
\usepackage{colortbl}

\definecolor{white}{rgb}{1,1,1}
\definecolor{darkred}{rgb}{0.3,0,0}
\definecolor{darkgreen}{rgb}{0,0.3,0}
\definecolor{darkblue}{rgb}{0,0,0.3}
\definecolor{pink}{rgb}{0.78,0.09,0.51}
\definecolor{purple}{rgb}{0.28,0.24,0.55}
\definecolor{orange}{rgb}{1,0.6,0.0}
\definecolor{grey}{rgb}{0.4,0.4,0.4}
\definecolor{lightgrey}{rgb}{0.85,0.85,0.85}
\definecolor{rechungsgrau}{rgb}{0.95, 0.95, 0.95}
\definecolor{aquamarine}{rgb}{0.4,0.8,0.65}
\definecolor{backgroundgreen}{rgb}{0.2, 0.8, 0.6}
\definecolor{backgroundred}{rgb}{1.0, 0.6, 0.6}
\definecolor{fazitgreen}{rgb}{0.0 0.6 0.2}


%-------------------------------------------------------------------------------
% Paragraph settings
\setlength{\parindent}{0em} 
\setlength{\parskip}{0.3em} 

%-------------------------------------------------------------------------------
% TikZ
\usepackage{tikz}
%\usetikzlibrary{positioning, arrows, decorations}
\usetikzlibrary{
    trees,
    shapes,
    arrows,
    backgrounds,
    positioning,
    fit,
    petri,
    decorations,
    decorations.pathmorphing,
    decorations.pathreplacing
}

\newcommand{\tikzcircle}[1]{\; \tikz[overlay, remember picture, anchor=base] \node[draw,circle,minimum size=4mm,inner sep=0pt, fill=black!20] {$#1$}; \;}

\newcommand{\tikzmark}[1]{\tikz[overlay,remember picture,baseline=(#1.base)] \node (#1) {\strut};}

\newcommand{\txb}[1]{{\color{blue}#1}}
\newcommand{\txr}[1]{{\color{red}#1}}
\newcommand{\txo}[1]{{\color{orange}#1}}
\newcommand{\txgr}[1]{{\color{grey}#1}}


\newcommand{\tc}[1]{\multicolumn{1}{r|}{#1}}

\begin{document}
	
    \section*{Normative Entscheidungstheorie}


    \begin{enumerate}[label=$\bullet$]
        \setlength{\parskip}{1mm}
        \item Entscheidungen mit \txb{mehreren} Zielen
        \item Entscheidung unter \txb{Ungewissheit}
        \item Entscheidung unter \txb{Risiko}
    \end{enumerate}

	\subsection*{Grundlagen der Entscheidungstheorie}
        Angenommen ein Mineralölkonzernstrebe nach \textbf{Gewinnmaximierungund} überprüfe die
        Eröffnung einer neuen Tankstelle. In Frage kommen zwei Standorte

        \begin{tabular}{| l | l | }
            \hline
                                 & Erwarteter Gewinn \\ \hline
            $a_1$ (Stadtzentrum) & \tc{CHF 125’000.-} \\ \hline
            $a_2$ (Stadtrand)    & \tc{CHF 150’000.-} \\ \hline
        \end{tabular}

	\subsection*{Zielkonflikt}

        Angenommen der Mineralölkonzernstrebe nach \textbf{Gewinnmaximierung} sowie
        \textbf{Umsatzmaximierung} und
        überprüfe die Eröffnung der beiden potenziellen Standorte

        \begin{tabular}{| l | l | l |}
            \hline
                                 & Erwarteter Gewinn & Erwarteter Umsatz \\ \hline
            $a_1$ (Stadtzentrum) & \tc{CHF 125’000.-} & \tc{CHF 2’000’000.-} \\ \hline
            $a_2$ (Stadtrand)    & \tc{CHF 150’000.-} & \tc{CHF 1’800’000.-} \\ \hline
        \end{tabular}

    \subsection*{Entscheidung unter Unsicherheit}

        Hauptsächliche Beschäftigungsgebiet der \txgr{Entscheidungstheorie} ist \txr{nicht} der
        \txb{Zielkonflikt} sondern die \txb{Unsicherheit}.

        \minisec{Umweltzustände}
        Faktoren, die eine \txgr{Entscheidungsalternative} beeinflussen,
        auf die man selber aber \txr{keinen Einfluss} hat.

        Bei \txgr{Entscheidungssituationen} in denen eine bestimmte \txgr{Entscheidung} \txr{nicht
        mit einem einzigen} möglichen \txb{Ereignis} verbunden ist,
        spricht man von einer \txgr{Entscheidung unter Unsicherheit}.
	
	\subsection*{Darstellung des Entscheidungsproblems}

        \minisec{Beispiel}

        Dem Mineralölkonzern ist bekannt, dass eine weiträumige Umgehungsstrasse
        geplant ist, die aber \txr{höchst umstritten} ist.

        \begin{enumerate}[label=$\bullet$]
            \setlength{\parskip}{1mm}
            \item Diese würden den Verkehr im \txgr{Stadtzentrum} \txr{nicht}
                  tangieren und der \txb{Gewinn} auf CHF 125’000.- belassen.
            \item Am \txgr{Stadtrand} würde sich aber eine \txr{drastische Veränderung} ergeben,
                  die den erwarteten \txb{Gewinn} auf CHF 80’000.- schmilz liesse.
        \end{enumerate}

        Welcher Standort wäre unter der Prämisse der Gewinnmaximierung nun zu präferieren?

        \minisec{Ergebnismatrix}
        Darstellungsform für

        \begin{enumerate}[label=$\bullet$]
            \setlength{\parskip}{1mm}
            \item \txgr{Entscheidungen unter Unsicherheit} wobei die
            \item \txb{Handlungsalternativen} ($a_i$) in den \txr{Zeilen} und die verschiedenen
            \item \txb{Umweltzustände} ($z_i$) in den \txr{Spalten} dargestellt werden.
        \end{enumerate}
        

        \begin{tabular}{| l | l | l |}
            \hline
                                 & $z_1$ (keine Umgehungsstrasse) & $z_2$ (Umgehungsstrasse) \\ \hline
            $a_1$ (Stadtzentrum) & \tc{$e_{11}$ = CHF 125’000.-} & \tc{$e_{12}$ = CHF 125’000.-} \\ \hline
            $a_2$ (Stadtrand)    & \tc{$e_{21}$ = CHF 150’000.-} & \tc{$e_{22}$ = CHF 80’000.-} \\ \hline
        \end{tabular}


        \minisec{Zustandsraum}
        Raum aller möglicher Entscheidungsergebnisse.

        Die \txgr{Entscheidungsmatrix} beschreibt die \txgr{Entscheidungssituation} nur dann
        \txr{hinreichend}, wenn der \txb{Zustandsraum} \txr{vollständig erfasst wird}, d.h.
        wenn über den \txb{Ergebnisraum} \txr{vollkommene Information} vorhanden ist.
    
        Dies wird vielen Entscheidungssituationen zwar unterstellt,
        ist aber in der Realität oft nicht gegeben.

	\subsection*{Klassifikation von Entscheidungssituationen unter Unsicherheit}

        \minisec{Entscheidung unter Risiko}
        Falls den \txgr{Umweltzuständen} \txr{Eintrittswahrscheinlichkeiten} zugeordnet werden können.

        \minisec{Entscheidung unter Ungewissheit}
        Falls den \txgr{Umweltzuständen} \txr{keine Wahrscheinlichkeiten} zugeordnet werden können.

        In der Ergebnismatrix trägt man bei Risikosituationen zusätzlich die Wahrscheinlichkeiten
        für die einzelnen Umweltzustände ein.

        \minisec{Beispiel}

        Angenommen der Ölkonzern schätzt die Wahrscheinlichkeit für den
        Bau der Umgehungsstrasse auf 30\%        

        \begin{tabular}{| l | l | l | l |}
            \hline
                                 & $z_1$ (keine Umgehungsstrasse) & $z_2$ (Umgehungsstrasse)      & \\ \hline
            $a_1$ (Stadtzentrum) & \tc{$e_{11}$ = CHF 125’000.-}  & \tc{$e_{12}$ = CHF 125’000.-} & \\ \hline
            $a_2$ (Stadtrand)    & \tc{$e_{21}$ = CHF 150’000.-}  & \tc{$e_{22}$ = CHF 80’000.-}  & \\ \hline
            WSK                  & \tc{$p_1 = 0.7$}               & \tc{$p_2 = 0.3$}              & $1.0$ \\ \hline
        \end{tabular}
        

	\subsection*{Gefangenendilemma}
        In einer Auszahlungsmatrix eingetragen ergibt sich inklusive
        des Gesamtergebnisses folgendes Bild:

        \begin{tabular}{| l | l | l | l |}
            \hline
                                 & $z_1$ (keine Umgehungsstrasse) & $z_2$ (Umgehungsstrasse)      & \\ \hline
            $a_1$ (Stadtzentrum) & \tc{$e_{11}$ = CHF 125’000.-}  & \tc{$e_{12}$ = CHF 125’000.-} & \\ \hline
            $a_2$ (Stadtrand)    & \tc{$e_{21}$ = CHF 150’000.-}  & \tc{$e_{22}$ = CHF 80’000.-}  & \\ \hline
            WSK                  & \tc{$p_1 = 0.7$}               & \tc{$p_2 = 0.3$}              & $1.0$ \\ \hline
        \end{tabular}
        
        

	\subsection*{a}
    
        -

	\subsection*{Superzeichen}

	\subsection*{Dynamik mentaler Modelle}

        -

	\subsection*{Mentale Modelle und Vorurteile}

        -

	\subsection*{Scheinsicherheit - das Ziegenproblem}

        -

	\subsection*{Kausalität vs. Korrelation}


	\subsection*{Problematik mentaler Modelle}
	\subsection*{Falsifikationismus}

	\subsection*{Mentale Modelle: Erkenntnisse}


\end{document}