

%-------------------------------------------------------------------------------
% Dokumenten Klasse
\documentclass[
final,
a4paper,
oneside,
parskip=full,
headings=standardclasses,
headings=big,
pointednumbers,
fleqn
]{scrartcl}

%-------------------------------------------------------------------------------
% Packete nutzen
\usepackage{ngerman,palatino,setspace}
\usepackage[T1]{fontenc}
\usepackage[utf8]{inputenc}
\usepackage[left=20mm,right=20mm,top=10mm,bottom=25mm]{geometry}
\usepackage{amsmath}
\usepackage{amssymb}
\usepackage{mathtools}

%-------------------------------------------------------------------------------
\usepackage[dvipsnames]{xcolor}

%-------------------------------------------------------------------------------
% dbox
\usepackage{dashbox}

%-------------------------------------------------------------------------------
% uline
\usepackage{ulem}

%-------------------------------------------------------------------------------
% TikZ
\usepackage{tikz}
%\usetikzlibrary{positioning, arrows, decorations}
\usetikzlibrary{arrows,decorations.pathmorphing,backgrounds,positioning,fit,petri}

\tikzset{
    myptr/.style={
        ->,
        >=stealth
    }
}


%-------------------------------------------------------------------------------
% \ifthenelse
\usepackage{ifthen}

%-------------------------------------------------------------------------------
% Für enumerate
\usepackage{enumitem}
\setlist[enumerate]{
    wide=0pt,
    leftmargin=*,
    itemsep=-1ex,
    parsep=2ex,
    labelsep=1ex,
    label=\alph*)
}

\usepackage{multirow}
\usepackage{ifthen}

%-------------------------------------------------------------------------------
% tabu
\usepackage{tabu} 

%-------------------------------------------------------------------------------
% table line breaks with \makecell
\usepackage{makecell}
\renewcommand\cellalign{bl}

%-------------------------------------------------------------------------------
% 
\usepackage{xparse}
% 1: Subscription  (default: '')
% 2: Funktion Name (default: 'f')
% 3: Argument      (default: 'x')
% \fx         = f(x)
% \fx[1]      = f_1(x)
% \fx[][u]    = u(x)
% \fx[][u][x] = u(x)
% \fx[][f][u] = f(u)
\NewDocumentCommand{\fx}{ O{} O{f} O{x} }{{#2_{#1}{\left( #3 \right)}}}
\NewDocumentCommand{\dfx}{ O{} O{f} O{x} }{{#2'_{#1}{\left( #3 \right)}}}
\NewDocumentCommand{\dx}{ O{} }{{\Delta x^{#1}}}
\NewDocumentCommand{\dy}{ O{} }{{\Delta y^{#1}}}
\NewDocumentCommand{\dt}{ O{} }{{\Delta t^{#1}}}
\NewDocumentCommand{\gp}{ O{} }{}

\NewDocumentCommand{\xyz}{ O{x} O{y} O{z} }{#1 #2 #3}
% 1: Value
% 2: Key
% 3: Background Color
\NewDocumentCommand{\tfill}{ O{} O{} O{blue!20} }{%
    \tikz[baseline, every node/.style={inner sep=3pt,outer sep=0pt,minimum width=3mm,minimum height=4mm}]{
        \node[fill=#3,anchor=base] (#2) {#1};
    }
}
\NewDocumentCommand{\tfillc}{ O{} O{} }{%
    \tikz[baseline, every node/.style={inner sep=2pt,outer sep=0pt}]{
        \node[fill=#2,anchor=base] {#1};
    }
}
\newcommand{\tfillb}[1]{\tfillc[#1][blue!20]}
\newcommand{\tfillo}[1]{\tfillc[#1][orange!40]}
\newcommand{\tfillg}[1]{\tfillc[#1][Green!20]}
\newcommand{\tfilly}[1]{\tfillc[#1][yellow!40]}
\newcommand{\tfillr}[1]{\tfillc[#1][red!20]}

% 1: Funktion Name
% 2: dx
% 3: dx^2
% 4: x_1
% 5: x_2
% 5: x_3
\NewDocumentCommand{\gpp}{ O{} O{\dx} O{\dx[2]} O{x} O{x} O{x} }{%
    \frac{1}{#3}
    \kl{#1 
        \kl{#4 \ifthenelse{\equal{#2}{}}{}{+ #2}} - 
        2 \cdot #1\kl{#5} +
        #1 \kl{#6\ifthenelse{\equal{#2}{}}{}{- #2}}
    }
}

% 1: x_1
% 2: x_2
% 3: x_3
% 4: dx^2
\NewDocumentCommand{\gppn}{ O{x_1} O{x_2} O{x_3} O{?}}{%
    \frac{1}{#4}
    \kl{#1 - 2 \cdot #2 + #3}
}

\newcommand*\difx{\; \mathop{}\!\mathrm{d}x}
\newcommand{\f}[2]{\frac{#1}{#2}}
\newcommand{\fs}[2]{{\scriptstyle\frac{#1}{#2}}}
\newcommand{\e}{\mathrm{e}}
\newcommand{\kl}[1]{{\left( #1 \right)}}
\newcommand{\kq}[1]{{\left\{ #1 \right\}}}
\newcommand{\ks}[1]{{\left[ #1 \right]}}
\newcommand{\dom}{{\Omega}}
\newcommand{\bound}{{\partial \Omega}}

%-------------------------------------------------------------------------------
% Dokument
\begin{document}
    
    %\section*{Finite Element Method (FEM)}
    {\setlength{\abovedisplayskip}{-6pt}
    \setlength{\belowdisplayskip}{-12pt}
    \begin{align*}
    \widetilde{u}\kl{x} & = a_1 \cdot v_1\kl{x} + \ldots + a_n \cdot v_n\kl{x}
    \end{align*}}

    {\bf{Weighted Residuals}} \\
    Generalization of Galerkin \\
    {\setlength{\abovedisplayskip}{-6pt}
    \setlength{\belowdisplayskip}{-12pt}
    \begin{align*}
    \int_{0}^{1}&{\kl{u''\kl{x} + f\kl{x}} \cdot w_k\kl{x} \difx} = 0  \\[0.25cm]
    M_{j,k} & = \int_{0}^{1}{v_j''\kl{x} \cdot w_k\kl{x} \difx}  \\[0.25cm]
    m_k & = \int_{0}^{1}{f\kl{x} \cdot w_k\kl{x} \difx}   \\[0.25cm]
    M &\cdot \uline{a} + \uline{m} = 0
    \end{align*}} \\
    Measures the distance between the approximation\\
    function and the exact solution: \\
    {\setlength{\abovedisplayskip}{-6pt}
    \setlength{\belowdisplayskip}{-12pt}
    \begin{align*}
    r\kl{x} & = u''\kl{x} + f\kl{x} \\[0.25cm]
    \int_{0}^{1}&{r\kl{x} \cdot w_k\kl{x} \difx}
    \end{align*}}
    
    {\bf{Method of Ritz}} \\
    {\setlength{\abovedisplayskip}{-6pt}
    \setlength{\belowdisplayskip}{-12pt}
    \begin{align*}
    R_{j,k} & = \int_{0}^{1}{v_j'\kl{x} \cdot v_k'\kl{x} \difx}  \\[0.25cm]
    r_k &= \int_{0}^{1}{f\kl{x} \cdot v_k\kl{x} \difx}   \\[0.25cm]
    R &\cdot \uline{a} = \uline{r} \quad \text{or} \quad R \cdot \uline{a} + \uline{r} = 0
    \end{align*}}

    {\bf{Method of Galerkin}} \\
    {\setlength{\abovedisplayskip}{-6pt}
    \setlength{\belowdisplayskip}{-12pt}
    \begin{align*}
    \int_{0}^{1}&{\kl{u''\kl{x} + f\kl{x}} \cdot v_k\kl{x} \difx} = 0  \\[0.25cm]
    G_{j,k} & = \int_{0}^{1}{v_j''\kl{x} \cdot v_k\kl{x} \difx}  \\[0.25cm]
    g_k &= \int_{0}^{1}{f\kl{x} \cdot v_k\kl{x} \difx}   \\[0.25cm]
    G &\cdot \uline{a} + \uline{g} = 0
    \end{align*}}
    
    {\bf{Method of Gauss}} \\
    {\setlength{\abovedisplayskip}{-6pt}
    \setlength{\belowdisplayskip}{-12pt}
    \begin{align*}
    \int_{0}^{1}&{\kl{u''\kl{x} + f\kl{x}}^2 \difx} = 0  \\[0.25cm]
    Q_{j,k} & = \int_{0}^{1}{v_j''\kl{x} \cdot v_k''\kl{x} \difx}  \\[0.25cm]
    q_k & = \int_{0}^{1}{f\kl{x} \cdot v_k''\kl{x} \difx}   \\[0.25cm]
    Q &\cdot \uline{a} + \uline{q} = 0
    \end{align*}} \\
    
    
    %===================================================================================
    \newpage
    
    {\bf{Problem 98}} \\
    Modified Problem 1 \\
    {\setlength{\abovedisplayskip}{-12pt}
    \setlength{\belowdisplayskip}{-6pt}
    \begin{align*}
    u''\kl{x} &+ f\kl{x} = 0 \\
    f\kl{x} &= 20, \quad u\kl{\tfillb{$0$}} = \tfillg{$10$} , \quad u\kl{\tfillr{$1$}} = \tfilly{$20$}
    \end{align*}} \\
    Inhomogeneous Boundary conditions: \\
    {\setlength{\abovedisplayskip}{-12pt}
    \setlength{\belowdisplayskip}{-22pt}
    \begin{align*}
    \widetilde{u}\kl{x} & = v_0\kl{x} + a_1 \cdot v_1\kl{x} + a_2 \cdot v_2\kl{x} \\[0.3cm]
    \widetilde{u}\kl{\tfillb{$0$}} & = v_0\kl{0} + a_1 \cdot v_1\kl{0} + a_2 \cdot v_2\kl{0} = \tfillg{$10$}\\
    v_0\kl{0} &= 10, \quad v_1\kl{0} = 0, \quad v_2\kl{0} = 0\\[0.3cm]
    \widetilde{u}\kl{\tfillr{$1$}} & = v_0\kl{1} + a_1 \cdot v_1\kl{1} + a_2 \cdot v_2\kl{1} = \tfilly{$20$}\\
    v_0\kl{1} &= 20, \quad v_1\kl{1} = 0, \quad v_2\kl{1} = 0\\
    \end{align*}} \\
    Basis Functions: \\
    \begin{minipage}{0.6\textwidth}
        \setlength{\abovedisplayskip}{0pt}
        \setlength{\belowdisplayskip}{-12pt}
        \begin{align*}
        v_0\kl{x} &= 10 \cdot x + 10    &v_0'\kl{x} &= 10                            &v_0''\kl{x} &= 0 \\
        v_1\kl{x} &= \sin\kl{\pi x}     &v_1'\kl{x} &= \pi \cdot \cos\kl{\pi x}      &v_1''\kl{x} &= -\pi^2 \cdot \sin\kl{\pi x}\\
        v_2\kl{x} &= \sin\kl{2 \pi x}   &v_2'\kl{x} &= 2 \pi \cdot \cos\kl{2 \pi x}  &v_2''\kl{x} &= -4 \pi^2 \cdot \sin\kl{2 \pi x}
        \end{align*}
    \end{minipage}

    {\bf{Point Collocation}} \\
    {\setlength{\abovedisplayskip}{-6pt}
    \setlength{\belowdisplayskip}{-12pt}
    \begin{align*}
        \widetilde{u}\kl{x_k} + \tfillb{$20$} = 0, \quad x_1 = \tfillg{$\fs{1}{4}$}, \quad x_2 = \tfilly{$\fs{3}{4}$}
    \end{align*}}
    {\setlength{\abovedisplayskip}{6pt}
    \setlength{\belowdisplayskip}{-12pt}
    \begin{align*}
        \widetilde{u}\kl{x}   &= v_0\kl{x}   + a_1 \cdot v_1\kl{x}   + a_2 \cdot v_2\kl{x} \\
        \widetilde{u}''\kl{x} &= v_0''\kl{x} + a_1 \cdot v_1''\kl{x} + a_2 \cdot v_2''\kl{x} = -f\kl{x} \\
        \widetilde{u}''\kl{\tfillg{$\fs{1}{4}$}} &= v_0''\kl{\fs{1}{4}} + a_1 \cdot v_1''\kl{\fs{1}{4}} + a_2 \cdot v_2''\kl{\fs{1}{4}} = \tfillb{$-20$} \\
        \widetilde{u}''\kl{\tfilly{$\fs{3}{4}$}} &= v_0''\kl{\fs{3}{4}} + a_1 \cdot v_1''\kl{\fs{3}{4}} + a_2 \cdot v_2''\kl{\fs{3}{4}} = \tfillb{$-20$} \\
    \end{align*}}
    \begin{minipage}{0.4\textwidth}
        \setlength{\abovedisplayskip}{-6pt}
        \setlength{\belowdisplayskip}{-12pt}
        \begin{align*}
        v_0''\kl{\fs{1}{4}} &= 0 & v_1''\kl{\fs{1}{4}} =& a & v_2''\kl{\fs{1}{4}} &= a \\
        v_0''\kl{\fs{3}{4}} &= 0 & v_1''\kl{\fs{3}{4}} =& a & v_2''\kl{\fs{3}{4}} &= a \\
        \end{align*}
    \end{minipage}
        
    
\end{document}
