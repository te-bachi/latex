

%-------------------------------------------------------------------------------
% Dokumenten Klasse
\documentclass[
    final,
    a4paper,
    oneside,
    parskip=full,
    headings=standardclasses,
    headings=big,
    pointednumbers,
    fleqn,
    numbers=noenddot
]{scrartcl}

%-------------------------------------------------------------------------------
% Packete nutzen
\usepackage{ngerman,palatino,setspace}
\usepackage[T1]{fontenc}
\usepackage[utf8]{inputenc}
\usepackage[left=20mm,right=20mm,top=20mm,bottom=25mm]{geometry}
\usepackage{amsmath}
\usepackage{amssymb}
\usepackage{mathtools}


%-------------------------------------------------------------------------------
% Headers
\usepackage{scrlayer-scrpage} 
\clearpairofpagestyles
\ohead{}
\chead{}
\ihead{}
\ifoot{}
\cfoot{}
\ofoot{}

%-------------------------------------------------------------------------------
% use commends \begin{comment}
\usepackage{verbatim}

%-------------------------------------------------------------------------------
\usepackage[dvipsnames]{xcolor}

%-------------------------------------------------------------------------------
% dbox
\usepackage{dashbox}

%-------------------------------------------------------------------------------
% uline
\usepackage{ulem}

%-------------------------------------------------------------------------------
% TikZ
\usepackage{tikz}
%\usetikzlibrary{positioning, arrows, decorations}
\usetikzlibrary{arrows,decorations.pathmorphing,backgrounds,positioning,fit,petri}

\tikzset{
    myptr/.style={
        ->,
        >=stealth
    }
}


%-------------------------------------------------------------------------------
% \ifthenelse
\usepackage{ifthen}

%-------------------------------------------------------------------------------
% Für enumerate
\usepackage{enumitem}
\setlist[enumerate]{
    wide=0pt,
    leftmargin=*,
    itemsep=-1ex,
    parsep=2ex,
    labelsep=1ex,
    label=\alph*)
}

\usepackage{multirow}
\usepackage{ifthen}

%-------------------------------------------------------------------------------
% tabu
\usepackage{tabu} 

%-------------------------------------------------------------------------------
% table line breaks with \makecell
\usepackage{makecell}
\renewcommand\cellalign{bl}


%-------------------------------------------------------------------------------
% for "\newcolumntype" macro
\usepackage{array}
\newlength\mylen
\settowidth{\mylen}{$50$}
\newcolumntype{Q}{>{\raggedright$}p{1cm}<{$}}

%-------------------------------------------------------------------------------
% 
\usepackage{xparse}
% 1: Subscription  (default: '')
% 2: Funktion Name (default: 'f')
% 3: Argument      (default: 'x')
% \fx         = f(x)
% \fx[1]      = f_1(x)
% \fx[][u]    = u(x)
% \fx[][u][x] = u(x)
% \fx[][f][u] = f(u)
\NewDocumentCommand{\fx}{ O{} O{f} O{x} }{{#2_{#1}{\left( #3 \right)}}}
\NewDocumentCommand{\dfx}{ O{} O{f} O{x} }{{#2'_{#1}{\left( #3 \right)}}}
\NewDocumentCommand{\dx}{ O{} }{{\Delta x^{#1}}}
\NewDocumentCommand{\dy}{ O{} }{{\Delta y^{#1}}}
\NewDocumentCommand{\dt}{ O{} }{{\Delta t^{#1}}}
\NewDocumentCommand{\gp}{ O{} }{}

\NewDocumentCommand{\xyz}{ O{x} O{y} O{z} }{#1 #2 #3}
% 1: Value
% 2: Key
% 3: Background Color
\NewDocumentCommand{\tfill}{ O{} O{} O{blue!20} }{%
    \tikz[baseline, every node/.style={inner sep=3pt,outer sep=0pt,minimum width=3mm,minimum height=4mm}]{
        \node[fill=#3,anchor=base] (#2) {#1};
    }
}
\NewDocumentCommand{\tfillc}{ O{} O{} }{%
    \tikz[baseline, every node/.style={inner sep=2pt,outer sep=0pt}]{
        \node[fill=#2,anchor=base] {#1};
    }
}

\NewDocumentCommand{\tdrawc}{ O{} O{} }{%
    \tikz[baseline, every node/.style={inner sep=2pt,outer sep=0pt}]{
        \node[rectangle,draw=#2,anchor=base] {#1};
    }
}
\newcommand{\tfillb}[1]{\tfillc[#1][blue!20]}
\newcommand{\tfillo}[1]{\tfillc[#1][orange!40]}
\newcommand{\tfillg}[1]{\tfillc[#1][Green!20]}
\newcommand{\tfilly}[1]{\tfillc[#1][yellow!40]}
\newcommand{\tfillr}[1]{\tfillc[#1][red!20]}
\newcommand{\tfillv}[1]{\tfillc[#1][Violet!20]}
\newcommand{\tfillt}[1]{\tfillc[#1][Turquoise!20]}
\newcommand{\tfillgr}[1]{\tfillc[#1][Gray!20]}


\newcommand{\tdrawb}[1]{\tdrawc[#1][blue!40]}
\newcommand{\tdrawo}[1]{\tdrawc[#1][orange!60]}
\newcommand{\tdrawg}[1]{\tdrawc[#1][Green!40]}
\newcommand{\tdrawy}[1]{\tdrawc[#1][yellow!60]}
\newcommand{\tdrawr}[1]{\tdrawc[#1][red!40]}
\newcommand{\tdrawv}[1]{\tdrawc[#1][Violet!40]}
\newcommand{\tdrawt}[1]{\tdrawc[#1][Turquoise!40]}
\newcommand{\tdrawgr}[1]{\tdrawc[#1][Gray!40]}
\newcommand{\tdrawk}[1]{\tdrawc[#1][black]}

% 1: Funktion Name
% 2: dx
% 3: dx^2
% 4: x_1
% 5: x_2
% 5: x_3
\NewDocumentCommand{\gpp}{ O{} O{\dx} O{\dx[2]} O{x} O{x} O{x} }{%
    \frac{1}{#3}
    \kl{#1 
        \kl{#4 \ifthenelse{\equal{#2}{}}{}{+ #2}} - 
        2 \cdot #1\kl{#5} +
        #1 \kl{#6\ifthenelse{\equal{#2}{}}{}{- #2}}
    }
}

% 1: x_1
% 2: x_2
% 3: x_3
% 4: dx^2
\NewDocumentCommand{\gppn}{ O{x_1} O{x_2} O{x_3} O{?}}{%
    \frac{1}{#4}
    \kl{#1 - 2 \cdot #2 + #3}
}

\newcommand*\difx{\; \mathop{}\!\mathrm{d}x}
\newcommand{\f}[2]{\frac{#1}{#2}}
\newcommand{\fs}[2]{{\tfrac{#1}{#2}}}
\newcommand{\sqrts}[1]{{\scriptstyle\sqrt{#1}}}
\newcommand{\e}{\mathrm{e}}
\newcommand{\kl}[1]{{\left( #1 \right)}}
\newcommand{\kq}[1]{{\left\{ #1 \right\}}}
\newcommand{\ks}[1]{{\left[ #1 \right]}}
\newcommand{\dom}{{\Omega}}
\newcommand{\bound}{{\partial \Omega}}


%-------------------------------------------------------------------------------
% Dokument
\begin{document}
    
    {\bf{Problem 1}} \\
    {\setlength{\abovedisplayskip}{-12pt}
    \setlength{\belowdisplayskip}{-14pt}
    \begin{align*}
        &-u''\kl{x} = f\kl{x} \\
        & u\kl{0} = 0, \quad u\kl{1} = 0
    \end{align*}} \\

    \begin{tikzpicture}[
        ]
        % draw horizontal line   
        \draw[myptr] (0,0) -- (5.5,0) node[right=3pt] {$x$};
        
        % draw vertical lines
        \foreach \k/\xk in {0/0, 1/2, 2/4, 3/6, 4/8, 5/10}
        {
            \draw (\k cm,2pt) -- (\k cm,-2pt)
                    node[below=3pt] (k\k) {$\k$}
                    node[below=20pt] (xk\k) {$\xk$};
        }
        % draw Temp
        \foreach \k/\y/\v in {0/4/40, 1/6.5/0, 2/9.3/0, 3/12.4/0, 4/15.9/0, 5/20/200}
        {
            \ifthenelse{\NOT 0 = \v }{
                \draw[densely dashed] (\k cm, 0) -- (\k cm,\y*0.1)
                        node[above=3pt] (T\k) {$T_{\k} = \v$};
            }{
                \draw[densely dashed] (\k cm, 0) -- (\k cm,\y*0.1)
                        node[above=3pt] (T\k) {$T_{\k}$};
            }
        }
        \node[left of=k0] {$k$};
        \node[left of=xk0] {$x_k$};
        % draw nodes
    \end{tikzpicture}

    {\bf{Example 5}} \\
    Stationary Temperature Profile $T\kl{x}$ in a long, slim, not isolated rod of length $L$:
    \begin{align*}
        \SwapAboveDisplaySkip
        T''\kl{x} + h \cdot \kl{T_A - T\kl{x}} = 0, \qquad T\kl{0} = T_0, \quad T\kl{L}=T_L & &
    \end{align*}
    Concrete values:
    \begin{align*}
        \SwapAboveDisplaySkip
        L = 10\mathrm{m} \quad
        h = 0.01 {\textstyle \frac{1}{\mathrm{cm}^2}} \quad
        T_A = 20^\circ\text{C} \quad
        T_0 = 40^\circ\text{C} \quad
        T_L = 200^\circ\text{C} & &
    \end{align*}

    \begin{comment}
    FDM: \\
    \begin{tabular}{p{1cm}l}
        \multicolumn{2}{l}{$T\kl{x_0} = 40, \quad T\kl{x_5}=200$} \\
        $T''\kl{x}$ & $\gpp[\widetilde{T}]$ \\
        $x_1$       & $\gpp[\widetilde{T}][2][2^2][x_1][x_1][x_1]$ \\
                    & $\gpp[\widetilde{T}][][2^2][x_2][x_1][x_0]$ \\
                    & $\gppn[T_2][T_1][T_0][2^2]$ \\
                    & $\gppn[40][T_1][T_0][2^2]$ \\
        \multicolumn{2}{l}{$T''\kl{x} + h \cdot \kl{T_A - T\kl{x}} = 0$} \\
        $x_1$       & $\gpp[\widetilde{T}][][4][x_2][x_1][x_0] + 0.01 \cdot \kl{ 20 - \widetilde{T}\kl{x_1}}$ = 0 \\
                    & $\gppn[T_2][T_1][T_0][4] + 0.01 \cdot \kl{ 20 - T_1}$ = 0 \\
                    & $\f{1}{4} T_0 - \f{2}{4}T_1 + \f{1}{4}T_2 + 0.2 - 0.01 T_1 = 0$ \\
                    & $\f{40}{4} - \kl{\f{2}{4} + 0.01} T_1 + \f{1}{4}T_2 + 0.2 = 0$ \\
                    & $10.2 = 0.51 T_1 - 0.25 T_2$
    \end{tabular}
    \end{comment}

    {\renewcommand{\arraystretch}{1.3}
    \begin{align*}
        &T\kl{x_0} = 40, \quad T\kl{x_5}=200 \\
        &\begin{array}{Ql}
        \tfillgr{$T''\kl{x}$} & \gpp[\widetilde{T}] \\
        x_1       & \gpp[\widetilde{T}][2][2^2][x_1][x_1][x_1] \\
                  & \tfillgr{$\gpp[\widetilde{T}][][2^2][x_2][x_1][x_0]$} \\
                  & \gppn[T_2][T_1][T_0][2^2] \\
                  & \gppn[40][T_1][T_0][2^2] \\
        \end{array} \\
        &\tfillgr{$T''\kl{x}$} + \tfillg{$h$} \cdot \kl{\tfillb{$T_A$} - \tfillr{$T\kl{x}$}} = 0 \\
        &\begin{array}{Ql}
        x_1       & \tfillgr{$\gpp[\widetilde{T}][][4][x_2][x_1][x_0]$} + \tfillg{$0.01$} \cdot \kl{ \tfillb{$20$} - \tfillr{$\widetilde{T}\kl{x_1}$}} = 0 \\
                  & \tfillgr{$\gppn[T_2][T_1][T_0][4]$} + \tfillg{$0.01$} \cdot \kl{ \tfillb{$20$} - \tfillr{$T_1$}} = 0 \\
                  & \f{1}{4} \tfillt{$T_0$} - \f{2}{4}T_1 + \f{1}{4}T_2 + 0.2 - 0.01 T_1 = 0 \\
                  & \f{\tfillt{$\scriptstyle 40$}}{4} - \kl{\f{2}{4} + 0.01} T_1 + \f{1}{4}T_2 + 0.2 = 0 \\
                  & 0.51 T_1 - 0.25 T_2 = 10.2\\
        x_2       & \gppn[T_3][T_2][T_1][4] + 0.01 \cdot \kl{ 20 - T_2} = 0 \\
                  & 0.25 T_1 - 0.5 T_2 + 0.25 T_2 + 0.2 - 0.01 T_2 = 0 \\
                  & -0.25 T_1 + 0.51 T_2 - 0.25 T_2 = 0.2 \\
        x_4       & \gppn[T_5][T_4][T_3][4] + 0.01 \cdot \kl{ 20 - T_4} = 0 \\
                  & 0.25 T_3 - 0.5 T_4 + 0.25 \tfillt{$T_5$} + 0.2 - 0.01 T_4 = 0 \\
                  & -0.25 T_3 + 0.51 T_4 - 0.25 \cdot \tfillt{$200$} = 0.2 \\
                  & -0.25 T_3 + 0.51 T_4 = 50.2
        \end{array}
    \end{align*} \\
    \begin{align*}
        \begin{array}{c@{\ }c@{\ }c@{\ }c@{\ }c@{\ }}
            A & \cdot & \uline{u} & = & \uline{f} \\
            \\
            \begin{bmatrix*}[r]
                 0.51 & -0.25 &     0 &     0 \\
                -0.25 &  0.51 & -0.25 &     0 \\
                    0 & -0.25 &  0.51 & -0.25  \\
                    0 &     0 & -0.25 &  0.51
            \end{bmatrix*} & \cdot &
            \begin{pmatrix}
                T_1\\
                T_2 \\
                T_3 \\
                T_4
            \end{pmatrix} & = &
            \begin{pmatrix}
                10.2 \\
                0.2 \\
                0.2 \\
                50.2
            \end{pmatrix}
            \\
        \end{array}
    \end{align*}}

    {\bf{Homework 8}} \\
    {\setlength{\abovedisplayskip}{-6pt}
    \setlength{\belowdisplayskip}{-12pt}
    \begin{align*}
        &u''\kl{x} - x \cdot u'\kl{x} + 4 \cdot u\kl{x} = x, \quad u(0) = 1, \quad u(1) = 0 \\
        &
    \end{align*}}

    {\setlength{\abovedisplayskip}{-6pt}
    \setlength{\belowdisplayskip}{-12pt}
    \begin{align*}
        & 0 \quad 1 \quad 2 \quad 3 \quad 4 \quad 5 \quad 6 \quad 7 \quad 8 \quad 9 \\
        & 0 \quad -1 \quad -2 \quad -3 \quad -4 \quad -5 \quad -6 \quad -8 \quad -9 \\
        & 0 \quad {-1} \quad {-2} \quad {-3} \quad {-4} \quad {-5} \quad {-6} \quad {-8} \quad {-9} \\
        & X \quad x \quad t \\
        & P_{{x-1},\;t} \quad P_{{x},\;t} \quad P_{{x+1},\;t} \quad P_{{x},\;{t+1}} \\
        & x_0 \quad x_1 \quad x_2 \quad x_3 \quad x_4 \quad x_5 \quad x_6 \quad x_7 \quad x_8 \quad x_9 \quad x_{10} \\
        & 0 \quad \fs{1}{4} \quad \fs{2}{4} \quad \fs{3}{4} \quad 1 \\
        & 0 \quad \fs{1}{5} \quad \fs{2}{5} \quad \fs{3}{5} \quad \fs{4}{5} \\
    \end{align*}} \\

    

\end{document}
