

%-------------------------------------------------------------------------------
% Dokumenten Klasse
\documentclass[
	final,
	a4paper,
	oneside,
	parskip=full,
	headings=standardclasses,
	headings=big,
	pointednumbers,
    pagesize,
    fleqn
]{scrartcl}

%-------------------------------------------------------------------------------
% Packete nutzen
\usepackage{ngerman,palatino,setspace}
\usepackage[T1]{fontenc}
\usepackage[utf8]{inputenc}
\usepackage[left=20mm,right=20mm,top=20mm,bottom=20mm]{geometry}
\usepackage{amsmath}
\usepackage{amssymb}
\usepackage{mathtools}

\usepackage{printlen}
\uselengthunit{mm}

%-------------------------------------------------------------------------------
% Dokument
\begin{document}
    % String Variablen
    \newcommand{\mys}{\; Hallo Welt}

    % Integer Variablen
    \newcounter{myc}
    \setcounter{myc}{1}

    % Float-Variablen (mit Einheit!)
    \newlength{\myf}
    \setlength{\myf}{12.5mm}
    
    \themyc \; \Roman{myc} \mys \; \printlength{\myf}
    
    \renewcommand{\mys}{\; Hello World}
    \stepcounter{myc}
    
    \themyc \; \Roman{myc} \mys

\end{document}