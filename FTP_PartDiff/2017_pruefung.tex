

%-------------------------------------------------------------------------------
% Dokumenten Klasse
\documentclass[
	final,
	a4paper,
	oneside,
	parskip=full,
	headings=standardclasses,
	headings=big,
	pointednumbers
]{scrartcl}

%-------------------------------------------------------------------------------
% Packete nutzen
\usepackage[T1]{fontenc}
\usepackage[utf8]{inputenc}
\usepackage[ngerman]{babel}
\usepackage[left=20mm,right=20mm,top=10mm,bottom=25mm]{geometry}
\usepackage{amsmath}
\usepackage{amssymb}
\usepackage{mathtools}

%-------------------------------------------------------------------------------
% Für enumerate
\usepackage{enumitem}
\setlist[enumerate]{
    wide=0pt,
    leftmargin=*,
    itemsep=-1ex,
    parsep=2ex,
    labelsep=1ex,
    label=\alph*)
}

%-------------------------------------------------------------------------------
% TikZ
\usepackage{tikz}
\usetikzlibrary{positioning, arrows, decorations}

%-------------------------------------------------------------------------------
% \ifthenelse
\usepackage{ifthen}

%-------------------------------------------------------------------------------
% 
\usepackage{xparse}
\NewDocumentCommand{\dx}{ O{} }{{\Delta x^{#1}}}
\NewDocumentCommand{\dy}{ O{} }{{\Delta y^{#1}}}
\NewDocumentCommand{\dt}{ O{} }{{\Delta t^{#1}}}
\NewDocumentCommand{\du}{ O{} }{{\Delta u^{#1}}}
\NewDocumentCommand{\px}{ O{} }{{\partial x^{#1}}}
\NewDocumentCommand{\py}{ O{} }{{\partial y^{#1}}}
\NewDocumentCommand{\pt}{ O{} }{{\partial t^{#1}}}
\NewDocumentCommand{\pu}{ O{} }{{\partial u^{#1}}}
\NewDocumentCommand{\p}{ O{} }{{\partial^{#1}}}
\NewDocumentCommand{\re}{ O{} }{{\mathbb{R}^{#1}}}
\NewDocumentCommand{\lap}{}{\mathcal{L}}

\newcommand{\f}[2]{\frac{#1}{#2}}
\newcommand{\fs}[2]{{\tfrac{#1}{#2}}}

% kl = ()
\newcommand{\kl}[1]{{\left( #1 \right)}}

% kq = {}
\newcommand{\kq}[1]{{\left\{ #1 \right\}}}

% ks = []
\newcommand{\ks}[1]{{\left[ #1 \right]}}

\newcommand{\dom}{{\Omega}}
\newcommand{\bound}{{\partial \Omega}}
\newcommand{\e}{\mathrm{e}}


\newcommand{\writen}[2]{%  
    \begin{tikzpicture}[]
        
        % draw vertical lines
        \foreach \x in {0,...,#1}
        {
            \draw[lightgray] (\x * 4mm, 0mm) -- (\x * 4mm, #2 * 4mm);
        }

        % draw horizontal lines
        \foreach \y in {0,...,#2}
        {
            \draw[lightgray] (0mm, \y * 4mm) -- (#1 * 4mm, \y * 4mm);
        }

        % border
        \draw[black] (0mm, 0mm) -- (#1 * 4mm, 0mm);
        \draw[black] (0mm, 0mm) -- (0mm, #2 * 4mm);
        \draw[black] (#1 * 4mm, 0mm) -- (#1 * 4mm, #2 * 4mm);
        \draw[black] (0mm, #2 * 4mm) -- (#1 * 4mm, #2 * 4mm);
    \end{tikzpicture}
}

%-------------------------------------------------------------------------------
% Paragraph
\setlength{\parindent}{0em}
\setlength{\parskip}{0.35em}

%-------------------------------------------------------------------------------
% Dokument
\title{Prüfung 2017}
\author{}
\date{}
\begin{document}

    %-- Page 1 -----------------------------------------------------------------
    \maketitle

    %-- Page 2 -----------------------------------------------------------------
    \newpage
    \section*{Grundlagen / Analytisch}
	\subsection*{Aufgabe 1}

    Sei $0 < x_0 < 1 $ gegeben und $\Omega$ sei das Gebiet
    \begin{align}
        \Omega = \left\{\left(x, y\right) \mid x_0 < x < 1 \land 0 < y < 1 \right\}
    \end{align}
    Die Funktion $u$ erfüllt in $\Omega$ die partielle Differentialgleichung
    \begin{align}
        \frac{1}{x^2}\frac{\partial u}{\partial x} + \frac{\partial u}{\partial y} = 1
    \end{align}
    und die Randbedingung
    \begin{align*}
        u{\left(x_0, y \right)} = \sin{\left(y\right)}, \quad 0 < y < 1.
    \end{align*}
    \begin{enumerate}
        \item Finden Sie eine Funktion $u$ mit dieser Eigenschaft.
        \item Ist die Lösung $u$ durch die gegebenen Randbedingungen eindeutig bestimmt?
    \end{enumerate}

    \writen{42}{44}

    %-- Page 3 -----------------------------------------------------------------
    \newpage

    \writen{42}{66}

    %-- Page 4 -----------------------------------------------------------------
    \newpage

	\subsection*{Aufgabe 2}

    Gegeben ist die Differentialgleichung
    \begin{align}
        \frac{\partial^3 u}{\partial t^3} + \frac{\partial u}{\partial x} = 0
    \end{align}
    auf dem Gebiet
    \begin{align*}
        \Omega = \left\{\left(x, t\right) \mid t > 0 \land x > 0 \right\}
    \end{align*}
    mit Randbedingungen
    \begin{align}
        \begin{split}
            u{\left(x,  0\right)} &= 0, \quad 
                \frac{\partial u}{\partial t}{\left(x,  0\right)} = 0, \quad 
                \frac{\partial^2 u}{\partial t^2}{\left(x,  0\right)} = 1, \\
            u{\left(0,  t\right)} &= 1.
        \end{split}
    \end{align}
    \begin{enumerate}
        \item Führen Sie Laplace-Transformation nach der Variablen $t$ durch und stellen Sie
              eine Differentialgleichung für $\mathcal{L} u{\left( x, s \right)}$ samt
              Anfangsbedingungen auf.
        \item Bestimmen Sie die Funktion $\mathcal{L} u{\left( x, s \right)}$.
    \end{enumerate}
    \textit{Hinweis.} \quad Die gewöhnliche Differentialgleichung $y' + ay = b$ mit Anfangsbedingung $y \left(0\right) = c$ hat die allgemeine Lösung
    \begin{align*}
        y \left(x\right) = \left( c - \frac{b}{a} \right) \mathrm{e}^{-ax} + \frac{b}{a}.
    \end{align*}

    \writen{42}{40}

    %-- Page 5 -----------------------------------------------------------------
    \newpage

    \writen{42}{66}
    
    %-- Page 6 -----------------------------------------------------------------
    \newpage
    
	\subsection*{Aufgabe 3}

    Auf dem Gebiet
    \begin{align*}
        \Omega = \left\{\left(x, y\right) \mid 1 < x^2 + y^2 \right\}
    \end{align*}
    soll die folgende partielle Differentialgleichung in Polarkoordinaten
    \begin{align}
        \frac{\partial^2 u}{\partial r^2} + \frac{\partial^2 u}{\partial \varphi^2} = 0
    \end{align}
    gelöst werden. Auf welchem Teil des Randes $x^2 + y^2 = 1$ müssen Randwerte vorgegeben werden,
    damit die Lösung $u{\left(x,  y\right)}$ für Punkte mit $x > 0$ und $x^2 + y^2 = 4$ eindeutig bestimmt ist?

    \writen{42}{50}

    %-- Page 7 -----------------------------------------------------------------
    \newpage

    \writen{42}{66}

    %-- Page 8 -----------------------------------------------------------------
    \newpage

    \subsection*{Aufgabe 4}

    Gegeben ist die nichtlineare partielle Differentialgleichung zweiter Ordnung
    \begin{align}
        \left( 1 + \left( \frac{\partial u}{\partial y} \right)^2 \right) \frac{\partial^2 u}{\partial x^2} -
        2 \frac{\partial u}{\partial y} \frac{\partial u}{\partial x} \frac{\partial^2 u}{\partial x \partial y} +
        \left( 1 + \left( \frac{\partial u}{\partial x} \right)^2 \right) \frac{\partial^2 u}{\partial y^2} = 0
    \end{align}
    für die Funktion $u{\left(x,  y\right)}$ auf dem Gebiet $\Omega = \mathbb{R}^2$. Die Funktion $u$ soll zudem
    die Eigenschaft
    \begin{align*}
        u{\left(0, 0 \right)} = 0 \qquad \text{und} \qquad \operatorname{grad}{u{\left(0, 0\right)}} = 0
    \end{align*}
    haben.
    \begin{enumerate}
        \item Führen Sie Separation durch mit Hilfe eines Ansatzes der Form
              $u{\left(x,  y\right)} = X{\left(x\right)} + Y{\left(y\right)}$ und stellen
              Sie Differentialgleichungen und Anfangsbedingungen für $X{\left(x\right)}$ und
              $Y{\left(y\right)}$ auf.
        \item Legen die Vorgaben die Separationslösung eindeutig fest?
    \end{enumerate}

    \writen{42}{46}

    %-- Page 9 -----------------------------------------------------------------
    \newpage

    \writen{42}{66}

    %-- Page 10 -----------------------------------------------------------------
    \newpage
    
    \section*{Numerik}

    \subsection*{Aufgabe 5}
    Die reelle Funktion $u{\left(x,  y\right)}$ ist auf dem Quadrat
    \begin{align*}
        \Omega = \left[ 0, 1 \right] \times \left[ 0, 1 \right]
    \end{align*}
    definiert. Sie erfüllt im Inneren von $\Omega$ die Bipotentialgleichung
    \begin{align*}
        \Delta{\left( \Delta u{\left(x,  y\right)} \right)} = 4
    \end{align*}
    und auf dem Rand von $\Omega$ die Bedingung
    \begin{align*}
        u{\left(x,  y\right)} = 0
    \end{align*}
    und
    \begin{align*}
        \du{\left(x,  y\right)} =
        \setlength{\arraycolsep}{0pt}
        \renewcommand{\arraystretch}{1.2}
        \left\{
            \begin{array}{l @{\quad} l r l}
                x^2-x & \text{falls\quad} & y &{}= 0 \\
                x^2-x & \text{falls\quad} & y &{}= 1 \\
                y^2-y & \text{falls\quad} & x &{}= 0 \\
                y^2-y & \text{falls\quad} & x &{}= 1
            \end{array}
        \right.
    \end{align*}
    Gesucht ist eine Approximation der Werte von $u$ in den vier Punkten von $\Omega$
    \begin{align*}
        \begin{pmatrix}
            {^1/_3} \\ {^1/_3}
        \end{pmatrix},\quad
        \begin{pmatrix}
            {^2/_3} \\ {^1/_3}
        \end{pmatrix},\quad
        \begin{pmatrix}
            {^1/_3} \\ {^2/_3}
        \end{pmatrix},\quad
        \begin{pmatrix}
            {^2/_3} \\ {^2/_3}
        \end{pmatrix}
    \end{align*}
    Benutzen Sie hierzu geeignete Finite Differenzen.

    \writen{42}{33}

    %-- Page 11 -----------------------------------------------------------------
    \newpage

    \writen{42}{66}

    %-- Page 12 -----------------------------------------------------------------
    \newpage

    \subsection*{Aufgabe 6}
    Die reelle Funktion $u{\left(x,  t\right)}$ ist auf dem Streifen
    \begin{align*}
        \Omega = \left[ 0, 1 \right] \times \left[ 0, \infty \right)
    \end{align*}
    definiert. Sie erfüllt im Inneren von $\Omega$ die Wärmeleitungsgleichung
    \begin{align*}
        u_t{\left(x,  t\right)} = u_{xx}{\left(x,  t\right)}
    \end{align*}
    und auf dem Rand von $\Omega$ die Bedingung
    \begin{align*}
        u{\left(0,  t\right)} = 1 \qquad \text{und} \quad u{\left(1,  t\right)} = 2
    \end{align*}
    sowie
    \begin{align*}
        u{\left(x,  0\right)} = 1 + 2 \cdot x - x^2
    \end{align*}
    Gesucht ist eine Approximation der Werte von $u$ in den zwei Punkten von
    \begin{align*}
        \begin{pmatrix}
            {^1/_3} \\ {^1/_4}
        \end{pmatrix},\quad
        \begin{pmatrix}
            {^2/_3} \\ {^1/_4}
        \end{pmatrix}
    \end{align*}
    Benutzen Sie hierzu das elementare Implizite Finite Differenzen Verfahren mit Schrittweiten
    \begin{align*}
        \Delta{x} = {^1/_3} \qquad \text{und} \qquad \Delta{t} = {^1/_{12}}
    \end{align*}
    
    \writen{42}{40}

    %-- Page 13 -----------------------------------------------------------------
    \newpage

    \writen{42}{66}

    %-- Page 14 -----------------------------------------------------------------
    \newpage

    \subsection*{Aufgabe 7}
    Die reelle Funktion $u{\left(x,  y\right)}$ ist auf dem Quadrat
    \begin{align*}
        \Omega = \left[ 0, 1 \right] \times \left[ 0, 1 \right]
    \end{align*}
    definiert. Sie erfüllt im Inneren von $\Omega$ die Laplacegleichung
    \begin{align*}
        \Delta u{\left(x,  y\right)} = 0
    \end{align*}
    und auf dem Rand von $\Omega$ die Randbedingung
    \begin{align*}
        u{\left(x,  y\right)} =
        \setlength{\arraycolsep}{0pt}
        \renewcommand{\arraystretch}{1.2}
        \left\{
            \begin{array}{l @{\quad} l r l}
                0 & \text{falls\quad} & y &{}= 0 \\
                x & \text{falls\quad} & y &{}= 1 \\
                0 & \text{falls\quad} & x &{}= 0 \\
                y & \text{falls\quad} & x &{}= 1
            \end{array}
        \right.
    \end{align*}
    Gesucht ist eine Approximation der Werte von $u$ in den vier Punkten
    \begin{align*}
        \tilde{u}_1 = \tilde{u}{\begin{pmatrix}
            {^1/_5} \\ {^1/_5}
        \end{pmatrix}},\quad
        \tilde{u}_2 = \tilde{u}{\begin{pmatrix}
            {^3/_5} \\ {^1/_5}
        \end{pmatrix}},\quad
        \tilde{u}_3 = \tilde{u}{\begin{pmatrix}
            {^1/_5} \\ {^3/_5}
        \end{pmatrix}},\quad
        \tilde{u}_4 = \tilde{u}{\begin{pmatrix}
            {^3/_5} \\ {^3/_5}
        \end{pmatrix}}
    \end{align*}
    Benutzen Sie hierzu Finite Volumina nach Voronoi.
    
    \writen{42}{38}

    %-- Page 15 -----------------------------------------------------------------
    \newpage

    \writen{42}{66}

    %-- Page 16 -----------------------------------------------------------------
    \newpage

    \subsection*{Aufgabe 8}
    Die reelle Funktion $u{\left(x\right)}$ ist auf dem Intervall
    \begin{align*}
        \Omega = \left[ 0, 1 \right]
    \end{align*}
    definiert. Sie erfüllt im Inneren von $\Omega$ die Differentialgleichung
    \begin{align*}
        u''{\left(x\right)} + u{\left(x\right)} = 1
    \end{align*}
    und auf dem Rand von $\Omega$ die Randbedingung
    \begin{align*}
        u{\left( 0 \right)} = 0, \qquad u{\left( 1 \right)} = 1
    \end{align*}
    Berechnen Sie $a_1$ und $a_2$ im Approximationsansatz für $u$
    \begin{align*}
        \tilde{u} = v_0{\left(x\right)} + a_1 \cdot v_1{\left(x\right)} + a_2 \cdot v_2{\left(x\right)}
    \end{align*}
    mit
    \begin{align*}
        v_0{\left(x\right)} = x, \quad v_1{\left(x\right)} = x^2 - x, \quad v_2{\left(x\right)} = x^3 - x^2.
    \end{align*}
    Benutzen Sie hierzu die Methode der Gewichteten Residuen mit Gewichtsfunktion
    \begin{align*}
        w_1{\left(x\right)} = 1, \quad w_2{\left(x\right)} = x
    \end{align*}

    \writen{42}{40}

    %-- Page 17 -----------------------------------------------------------------
    \newpage

    \writen{42}{66}

\end{document}