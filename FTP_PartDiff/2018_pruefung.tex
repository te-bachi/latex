

%-------------------------------------------------------------------------------
% Dokumenten Klasse
\documentclass[
	final,
	a4paper,
	oneside,
	parskip=full,
	headings=standardclasses,
	headings=big,
	pointednumbers
]{scrartcl}

%-------------------------------------------------------------------------------
% Packete nutzen
\usepackage[T1]{fontenc}
\usepackage[utf8]{inputenc}
\usepackage[ngerman]{babel}
\usepackage[left=20mm,right=20mm,top=10mm,bottom=25mm]{geometry}
\usepackage{amsmath}
\usepackage{amssymb}
\usepackage{mathtools}

%-------------------------------------------------------------------------------
% Für enumerate
\usepackage{enumitem}
\setlist[enumerate]{
    wide=0pt,
    leftmargin=*,
    itemsep=-1ex,
    parsep=2ex,
    labelsep=1ex,
    label=\alph*)
}

%-------------------------------------------------------------------------------
% TikZ
\usepackage{tikz}
\usetikzlibrary{positioning, arrows, decorations}

%-------------------------------------------------------------------------------
% \ifthenelse
\usepackage{ifthen}

%-------------------------------------------------------------------------------
% Links
\usepackage{hyperref}
\hypersetup{
    colorlinks=true,
    linkcolor=red,
    filecolor=magenta,      
    urlcolor=cyan,
}

%-------------------------------------------------------------------------------
% 
\usepackage{xparse}
\NewDocumentCommand{\dx}{ O{} }{{\Delta x^{#1}}}
\NewDocumentCommand{\dy}{ O{} }{{\Delta y^{#1}}}
\NewDocumentCommand{\dt}{ O{} }{{\Delta t^{#1}}}
\NewDocumentCommand{\du}{ O{} }{{\Delta u^{#1}}}
\NewDocumentCommand{\px}{ O{} }{{\partial x^{#1}}}
\NewDocumentCommand{\py}{ O{} }{{\partial y^{#1}}}
\NewDocumentCommand{\pt}{ O{} }{{\partial t^{#1}}}
\NewDocumentCommand{\pu}{ O{} }{{\partial u^{#1}}}
\NewDocumentCommand{\p}{ O{} }{{\partial^{#1}}}
\NewDocumentCommand{\re}{ O{} }{{\mathbb{R}^{#1}}}
\NewDocumentCommand{\lap}{}{\mathcal{L}}

\newcommand{\f}[2]{\frac{#1}{#2}}
\newcommand{\fs}[2]{{\tfrac{#1}{#2}}}

% kl = ()
\newcommand{\kl}[1]{{\left( #1 \right)}}

% kq = {}
\newcommand{\kq}[1]{{\left\{ #1 \right\}}}

% ks = []
\newcommand{\ks}[1]{{\left[ #1 \right]}}

\newcommand{\dom}{{\Omega}}
\newcommand{\bound}{{\partial \Omega}}
\newcommand{\e}{\mathrm{e}}


\newcommand{\writen}[2]{%  
    \begin{tikzpicture}[]
        
        % draw vertical lines
        \foreach \x in {0,...,#1}
        {
            \draw[lightgray] (\x * 4mm, 0mm) -- (\x * 4mm, #2 * 4mm);
        }

        % draw horizontal lines
        \foreach \y in {0,...,#2}
        {
            \draw[lightgray] (0mm, \y * 4mm) -- (#1 * 4mm, \y * 4mm);
        }

        % border
        \draw[black] (0mm, 0mm) -- (#1 * 4mm, 0mm);
        \draw[black] (0mm, 0mm) -- (0mm, #2 * 4mm);
        \draw[black] (#1 * 4mm, 0mm) -- (#1 * 4mm, #2 * 4mm);
        \draw[black] (0mm, #2 * 4mm) -- (#1 * 4mm, #2 * 4mm);
    \end{tikzpicture}
}

% Einzug mit Dimension (z.B. 1cm)
\newenvironment{myindent}[1]{% 
    \parskip=6pt \parindent=0pt \raggedright 
    \def\lititem{\hangindent=#1 \hangafter=1}
}{%
    \par\ignorespaces
}

\newenvironment{hint}[2]{% 
    \begin{minipage}[t]{#2}
        \textit{#1}
    \end{minipage}
    \begin{minipage}[t]{\linewidth - #2}
}{%
    \end{minipage}
} 

%-------------------------------------------------------------------------------
% Paragraph
\setlength{\parindent}{0em}
\setlength{\parskip}{0.35em}

%-------------------------------------------------------------------------------
% Dokument
\title{Prüfung 2018}
\author{}
\date{}
\begin{document}

    %-- Page 1 -----------------------------------------------------------------
    \maketitle

    %-- Page 2 -----------------------------------------------------------------
    \newpage
    \section*{Grundlagen / Analytisch}
    \subsection*{Aufgabe 1}

    Betrachten Sie die Differentialgleichung
    \begin{align}
        \f{\pu}{\pt} + \f{\p[4] u}{\px[4]} = 0
    \end{align}
    auf dem Gebiet
    \begin{align*}
        \dom = \kq{\kl{t, x} \mid t > 0, x \in \re}
    \end{align*}
    mit den Bedingungen
    \begin{align*}
        u\kl{t, 0} &= 1, & \f{\pu}{\px}\kl{t, 0} &= 0, & \f{\p[2] u}{\px[2]}\kl{t, 0} &= 0, & \f{\p[3] u}{\px[3]}\kl{t, 0} &= 0, \\
        u\kl{0, x} &= 0
    \end{align*}
    für $x \in \re$ und $t > 0$.
    \begin{enumerate}
        \item{
            Verwenden Sie die Laplace-Transformation und finden Sie eine gewöhnliche Differentialgleichung
            samt Anfangsbedingungen, mit deren Hilfe das Problem gelöst werden kann.
        }
        \item{
            Ist das Problem gut gestellt?
        }
    \end{enumerate}
    \begin{hint}{Hinweis.}{2cm}
        Es wird nicht verlangt, dass Sie die gewöhnliche Differentialgleichung lösen.
    \end{hint}

    \writen{42}{36}

    %-- Page 3 -----------------------------------------------------------------
    \newpage

    \writen{42}{66}

    %-- Page 4 -----------------------------------------------------------------
    \newpage

    \subsection*{Aufgabe 2}

    Betrachten Sie die Differentialgleichung
    \begin{align}
                {\cos\kl{x}} \; \f{\p[2] u}{\px[2]} +
        2 \cdot {\sin\kl{x}} \; \f{\p[2]  }{\dx\dy} -
                {\cos\kl{x}} \; \f{\p[2] u}{\py[2]} = 0
    \end{align}
    auf dem Gebiet
    \begin{align}
        \dom = \kq{\kl{x, y} \mid 0 < x < 1 \; \land \; 0 < y < 1 }
    \end{align}
    mit den Randbedingungen
    \begin{align*}
        u\kl{x, 0} = f\kl{x} \qquad \text{und} \qquad u\kl{0, y} = g\kl{y}
    \end{align*}
    mit glatten Funktionen $f\kl{x}$ und $g\kl{y}$. Ist das Problem gut gestellt?

    \begin{hint}{Hinweis.}{2cm}
        Im Laufe der Lösung werden Sie auf eine gewöhnliche Differentialgleichung für $y\kl{x}$
        stossen, die relativ schwierig zu lösen ist. Versuchen Sie nicht, diese zu lösen.
        Leiten Sie vielmehr aus der Differentialgleichung eine Formel für $y'\kl{x}$ ab,
        untersuchen Sie das Vorzeichen von $y'\kl{x}$ und leiten Sie daraus ab, ob das
        Problem gut gestellt ist.
    \end{hint}

    \writen{42}{40}

    %-- Page 5 -----------------------------------------------------------------
    \newpage

    \writen{42}{66}
    
    %-- Page 6 -----------------------------------------------------------------
    \newpage
    
	\subsection*{Aufgabe 3}

    Gegeben ist die Differentialgleichung
    \begin{align}
        {\sin\kl{y}} \; \f{\pu}{\px} + \f{\pu}{\py} = 1 \label{eq:ex3}
    \end{align}
    auf dem Gebiet
    \begin{align*}
        \dom = \kq{\kl{x, y} \mid -1 < x < 1 \; \land \; 0 < y < \infty }
    \end{align*}
    mit der Randbedingung
    \begin{align*}
        u\kl{x, 0} = x^2
    \end{align*}
    \begin{enumerate}
        \item{
            Finden Sie eine Lösung der Differentialgleichung $\kl{\ref{eq:ex3}}$.
        }
        \item{
            Ist die Lösung durch die Vorgaben eindeutig bestimmt?
        }
    \end{enumerate}

    \writen{42}{40}

    %-- Page 7 -----------------------------------------------------------------
    \newpage

    \writen{42}{66}

    %-- Page 8 -----------------------------------------------------------------
    \newpage

    \subsection*{Aufgabe 4}

    Auf dem Quadrat
    \begin{align}
        \dom = \kq{\kl{x, y} \mid 0 < x < \pi \; \land \; 0 < y < \pi }
    \end{align}
    soll die Differentialgleichung
    \begin{align}
        \du = \sin\kl{x}\cos\kl{y}
    \end{align}
    mit den Randbedingungen
    \begin{align}
        u = 0 \qquad \text{auf } \p\dom
    \end{align}
    gelöst werden. Die Lösung soll in der Form
    \begin{align}
        u = u_p + u_h
    \end{align}
    gefunden werden, wobei $u_h$ eine Lösung der homogenen Gleichung $\Delta u_h = 0$ ist.
    
    \begin{enumerate}
        \item{
            Finden sie eine partikuläre Lösung $u_p$.
        }
        \item{
            Bestimmen Sie die Randbedingungen von $u_h$.
        }
        \item{
            Verwenden Sie einen Separationsansatz für $u_h$,
            stellen Sie die Differentialgleichungen \\ und Randbedinungen für die Faktoren auf.
        }
        \item{
            Lösen Sie die Differentialgleichungen.
        }
        \item{
            Führen Sie den Koeffizientenvergleich durch und bestimmen sie $u_h$.
        }
    \end{enumerate}


    \writen{42}{32}

    %-- Page 9 -----------------------------------------------------------------
    \newpage

    \writen{42}{66}

    %-- Page 10 -----------------------------------------------------------------
    \newpage
    
    \section*{Numerik}

    \subsection*{Aufgabe 5}

    Die reelle Funktion $u{\left(x,  t\right)}$ ist auf dem Streifen
    \begin{align*}
        \dom = \ks{0, 4} \times \ks{0, \infty}
    \end{align*}
    definiert. Sie erfüllt im Inneren von $\dom$ die Wellengleichung
    \begin{align*}
        u_{tt}\kl{x,  t} = 2 \cdot u_{xx}\kl{x,  t}
    \end{align*}
    und auf dem Rand von $\dom$ die Bedingung
    \begin{align*}
        u\kl{0, t} = 0 \qquad \text{und} \qquad u\kl{4, t} = 0
    \end{align*}
    sowie
    \begin{align*}
        u\kl{x, 0} = 4 \cdot x - x^2 \qquad \text{und} \qquad u_{t}\kl{x, 0} = 3
    \end{align*}
    Gesucht ist eine Approximation der Werte von $u$ in den drei Punkten von $\dom$
    \begin{align*}
        \kl{^1/_{3}}, \qquad \kl{^2/_{3}}, \qquad \kl{^3/_{3}}
    \end{align*}
    Benutzen Sie hierzu geeignete Finite Differenzen mit Schrittweiten
    \begin{align*}
        \Delta{x} = 1 \qquad \text{und} \qquad \Delta{t} = 1
    \end{align*}

    \writen{42}{35}

    %-- Page 11 -----------------------------------------------------------------
    \newpage

    \writen{42}{66}

    %-- Page 12 -----------------------------------------------------------------
    \newpage

    \subsection*{Aufgabe 6}
    Die reelle Funktion $u\kl{x, t}$ ist auf dem Streifen
    \begin{align*}
        \dom = \ks{0, 1} \times \ks{0, \infty}
    \end{align*}
    definiert. Sie erfüllt im Inneren von $\dom$ die Wärmeleitungsgleichung
    \begin{align*}
        u_{xx}\kl{x, t} = u_{t}\kl{x, t}
    \end{align*}
    und auf dem Rand von $\dom$ die Bedingung
    \begin{align*}
        u\kl{0, t} = 0 \qquad \text{und} \qquad u\kl{1, t} = 0
    \end{align*}
    sowie
    \begin{align*}
        u{\left(x,  0\right)} = 16 \cdot \kl{x - x^2}
    \end{align*}
    Gesucht ist eine Approximation der Werte von $u$ in den zwei Punkten von $\dom$
    \begin{align*}
        \begin{pmatrix}
            {^1/_3} \\ {^1/_2}
        \end{pmatrix},\quad
        \begin{pmatrix}
            {^2/_3} \\ {^1/_2}
        \end{pmatrix}
    \end{align*}
    Benutzen Sie hierzu das Verfahren von Crank-Nicolson mit Schrittweiten
    \begin{align*}
        \dx = {^1/_3} \qquad \text{und} \qquad \dt = {^1/_{4}}
    \end{align*}
    
    \writen{42}{40}

    %-- Page 13 -----------------------------------------------------------------
    \newpage

    \writen{42}{66}

    %-- Page 14 -----------------------------------------------------------------
    \newpage

    \subsection*{Aufgabe 7}
    In der Ebene ist ein kartesisches Koordinatensystem vorgegeben. \\
    Das Gebiet $\dom$ wird durch das Polygon mit Eckpunkten (in Reihenfolge)
    \begin{align*}
        \kl{0, 0}, \quad \kl{4, 0}, \quad \kl{4, 2}, \quad
        \kl{2, 2}, \quad \kl{2, 4}, \quad \kl{0, 4}, \quad \kl{0, 0}
    \end{align*}
    berandet. Die reelle Funktion $u\kl{x,y}$ ist auf $\dom$ definiert.
    Sie erfüllt im Innern von $\dom$ die Differentialgleichung
    \begin{align*}
        \du\kl{x,y} = 10
    \end{align*}
    und auf dem Rand von $\dom$ die Bedingungen
    \begin{align*}
        u\kl{x,y} = 0
    \end{align*}
    Gesucht ist eine Approximation der Werte von $u$ in den drei Punkten von $\dom$
    \begin{align*}
        \kl{1, 1}, \quad \kl{3, 1}, \quad \kl{1, 3}
    \end{align*}
    Benutzen Sie hierzu geeignete Finite Volumina.
    
    \writen{42}{38}

    %-- Page 15 -----------------------------------------------------------------
    \newpage

    \writen{42}{66}

    %-- Page 16 -----------------------------------------------------------------
    \newpage

    \subsection*{Aufgabe 8}
    Die reelle Funktion $u{\left(x\right)}$ ist auf dem Intervall
    \begin{align*}
        \dom = \ks{0, 2\pi}
    \end{align*}
    definiert. Sie erfüllt im Inneren von $\dom$ die Differentialgleichung
    \begin{align*}
        u''\kl{x} + 2 \cdot u\kl{x} = x
    \end{align*}
    und auf dem Rand von $\dom$ die Randbedingung
    \begin{align*}
        u\kl{0} = 0, \qquad u\kl{2\pi} = 0
    \end{align*}
    Berechnen Sie $a_1$, $a_2$, $a_3$ im Approximationsansatz für $u$
    \begin{align*}
        \tilde{u} = a_1 \cdot v_1\kl{x} + a_2 \cdot v_2\kl{x} + a_3 \cdot v_3\kl{x}
    \end{align*}
    mit
    \begin{align*}
        v_1\kl{x} = {\sin\kl{x}}, \quad v_2\kl{x} = {\sin\kl{2x}}, \quad v_3\kl{x} = {\sin\kl{3x}} 
    \end{align*}
    Benutzen Sie hierzu die Methode von Galerkin.

    \writen{42}{44}

    %-- Page 17 -----------------------------------------------------------------
    \newpage

    \writen{42}{66}

\end{document}