

%-------------------------------------------------------------------------------
% Dokumenten Klasse
\documentclass[
	final,
	a4paper,
	oneside,
	parskip=full,
	headings=standardclasses,
	headings=big,
	pointednumbers
]{scrartcl}

%-------------------------------------------------------------------------------
% Packete nutzen
\usepackage[T1]{fontenc}
\usepackage[utf8]{inputenc}
\usepackage[left=20mm,right=25mm,top=20mm,bottom=25mm]{geometry}
\usepackage{amsmath}
\usepackage{amssymb}
\usepackage{mathtools}

%-------------------------------------------------------------------------------
% Für enumerate
\usepackage{enumitem}
\setlist[enumerate]{
    wide=0pt,
    leftmargin=*,
    itemsep=-1ex,
    parsep=2ex,
    labelsep=1ex,
    label=\alph*)
}


%-------------------------------------------------------------------------------
% TikZ
\usepackage{tikz}
\usetikzlibrary{positioning, arrows, decorations}


%-------------------------------------------------------------------------------
% \ifthenelse
\usepackage{ifthen}

%-------------------------------------------------------------------------------
% 
\usepackage{xparse}
\NewDocumentCommand{\dx}{ O{} }{{\Delta x^{#1}}}
\NewDocumentCommand{\dy}{ O{} }{{\Delta y^{#1}}}
\NewDocumentCommand{\dt}{ O{} }{{\Delta t^{#1}}}
\NewDocumentCommand{\du}{ O{} }{{\Delta u^{#1}}}
\NewDocumentCommand{\px}{ O{} }{{\partial x^{#1}}}
\NewDocumentCommand{\py}{ O{} }{{\partial y^{#1}}}
\NewDocumentCommand{\pt}{ O{} }{{\partial t^{#1}}}
\NewDocumentCommand{\pu}{ O{} }{{\partial u^{#1}}}
\NewDocumentCommand{\p}{ O{} }{{\partial^{#1}}}
\NewDocumentCommand{\re}{ O{} }{{\mathbb{R}^{#1}}}
\NewDocumentCommand{\lap}{}{\mathcal{L}}

\newcommand{\f}[2]{\frac{#1}{#2}}
\newcommand{\fs}[2]{{\tfrac{#1}{#2}}}

% kl = ()
\newcommand{\kl}[1]{{\left( #1 \right)}}

% kq = {}
\newcommand{\kq}[1]{{\left\{ #1 \right\}}}

% ks = []
\newcommand{\ks}[1]{{\left[ #1 \right]}}

\newcommand{\dom}{{\Omega}}
\newcommand{\bound}{{\partial \Omega}}
\newcommand{\e}{\mathrm{e}}


\newcommand{\writen}[2]{%  
    \begin{tikzpicture}[]
        
        % draw vertical lines
        \foreach \x in {0,...,#1}
        {
            \draw[lightgray] (\x * 4mm, 0mm) -- (\x * 4mm, #2 * 4mm);
        }

        % draw horizontal lines
        \foreach \y in {0,...,#2}
        {
            \draw[lightgray] (0mm, \y * 4mm) -- (#1 * 4mm, \y * 4mm);
        }

        % border
        \draw[black] (0mm, 0mm) -- (#1 * 4mm, 0mm);
        \draw[black] (0mm, 0mm) -- (0mm, #2 * 4mm);
        \draw[black] (#1 * 4mm, 0mm) -- (#1 * 4mm, #2 * 4mm);
        \draw[black] (0mm, #2 * 4mm) -- (#1 * 4mm, #2 * 4mm);
    \end{tikzpicture}
}

%-------------------------------------------------------------------------------
% Paragraph
\setlength{\parindent}{0em}
\setlength{\parskip}{0.35em}

%-------------------------------------------------------------------------------
% Dokument
\title{Module exam 2019}
\author{}
\date{}
\begin{document}

    %-- Page 1 -----------------------------------------------------------------
    \maketitle

    %-- Page 2 -----------------------------------------------------------------
    \newpage

    \section*{Part 1: Theory}
	\subsection*{Exercise 1}

    For the second order partial differential equation
    \begin{align}
        -x\f{\pu}{\pt} + \f{\p[2] u}{\px[2]} = 0 \label{eq1}
    \end{align}
    on the domain
    \begin{align*}
        \dom = \kq{\kl{t, x} \in \re[2] \mid t > 0 \; \text{and} \; x > 0 }
    \end{align*}
    boundary conditions
    \begin{center}
        \vspace{-6mm}
        \begin{tabular}{ l l l l l }
            % 1
            $\begin{aligned}
                u\kl{0,x} &= 0 \\
                u\kl{t,0} &= \e^{-t}
            \end{aligned}$ &
            % 2
            $\begin{aligned}
                \\
                \text{and}
            \end{aligned}$ &
            % 3
            $\begin{aligned}
                \\
                \f{\pu}{\px}\kl{t,0} = 1
            \end{aligned}$ &
            % 4
            $\begin{aligned}
                \text{for} \\
                \text{for}
            \end{aligned}$ &
            % 5
            $\begin{aligned}
                x &> 0 \\
                t &> 0
            \end{aligned}$
        \end{tabular}
        \vspace{-6mm}
    \end{center}
    are given.

    \begin{enumerate}
        \item{
            Convert the partial differential equation $\kl{\ref{eq1}}$ into a family of
            ordinary differential equations using the Laplace transform with respect to
            the variable $t$.
        }
        \item{
            Assume that there are solutions $A\kl{x}$ and $B\kl{x}$ of the ordinary
            differential equation

            \begin{center}
                \vspace{-3mm}
                $\begin{aligned}
                    y'' - xy = 0
                \end{aligned}$
                \vspace{-3mm}
            \end{center}

            with initial conditions

            \begin{center}
                \vspace{-3mm}
                \begin{tabular}{@{} l l }
                    % 1
                    \rule{0pt}{4ex}$\begin{aligned}
                        A\kl{0} = 1 \\
                        B\kl{0} = 0
                    \end{aligned}$ &
                    % 2
                    $\begin{aligned}
                        A'\kl{0} = 0 \\
                        B'\kl{0} = 1
                    \end{aligned}$
                \end{tabular}
                \vspace{-3mm}
            \end{center}

            Show that the functions $A\kl{\sqrt{s} \cdot x}$ and $B\kl{\sqrt{s} \cdot x}$ are
            solutions of the ordinary differential equation obtained in $a)$.
        }
        \item{
            Express the Laplace transform $\lap u \kl{s,x}$ by means of the
            functions $A\kl{x}$ and $B\kl{x}$.
        }
    \end{enumerate}

    \writen{42}{22}

    %-- Page 3 -----------------------------------------------------------------
    \newpage

    \writen{42}{61}

    %-- Page 4 -----------------------------------------------------------------
    \newpage

	\subsection*{Exercise 2}

    Consider the partial differential equation
    \begin{align}
        \f{1}{2} \f{\p[2] u}{\px[2]} +
        \kl{\sin\kl{x} + \cos\kl{x}} \frac{\p[2] u}{\px \py} +
        \kl{\sin\kl{x} \cdot \cos\kl{x}} \frac{\p[2] u}{\py[2]}= 0
    \end{align}
    in the domain
    \begin{align*}
        \dom = \re[2]
    \end{align*}

    \begin{enumerate}
        \item{
            Is the differential equation elliptic, parabolic or hyperbolic?
        }
        \item{
            Do boundary conditions
            \begin{center}
                \begin{tabular}{ c c c }
                    % 1
                    $\begin{aligned}
                        u\kl{0,y} = f\kl{y}
                    \end{aligned}$ &
                    % 2
                    $\begin{aligned}
                        \text{and}
                    \end{aligned}$ &
                    % 3
                    $\begin{aligned}
                        \f{\pu}{\px}\kl{0,y} = g\kl{y}
                    \end{aligned}$
                \end{tabular}
            \end{center}

            on the boundary of the smaller domain $\dom_0 = \kq{\kl{x,y} \in \re[2] \mid x > 0}$
            uniquely determine the solution?
        }
    \end{enumerate}

    \writen{42}{40}

    %-- Page 5 -----------------------------------------------------------------
    \newpage

    \writen{42}{61}
    
    %-- Page 6 -----------------------------------------------------------------
    \newpage
	\subsection*{Exercise 3}

    Consider the partial differential equation
    \begin{align*}
        u \f{\pu}{\px} - u \frac{\pu}{\py} = 0
    \end{align*}
    on the domain
    \begin{align}
        \dom = \kq{\kl{x,y} \in \re[2] \mid y < x}
    \end{align}
    Find a solution that satisfies the boundary condition $u\kl{x,y} = {\sin\kl{x}}$ on $\bound$.

    \writen{42}{46}

    %-- Page 7 -----------------------------------------------------------------
    \newpage

    \writen{42}{61}

    %-- Page 8 -----------------------------------------------------------------
    \newpage

    \subsection*{Exercise 4}

    The air column in a wind instrument with conical bore can be described as a cone
    with center at the origin where the mouthpiece is located. The oboe fits this model
    quite well, a saxophone also has a conical bore although twisted into an S-shape.
    
    The most suitable coordinate system for this problem is spherical coordinates.

    One can safely assume that the density of the air does not depend on the angle
    $\varphi$ and $\vartheta$, so that it suffices to model the air density $\varrho$
    as a function of time and the radius alone, i.e. as a function $\varrho\kl{t,r}$.
    More precisely, $\varrho\kl{t,r}$ is the density deviation with respect to the
    density of the air surrounding the instrument. The wave equation then becomes

    \begin{align}
        \f{\p[2]}{\pt[2]} \varrho - a^2 \kl{\f{\p[2]}{\p r^2} + \f{2}{r} \f{\p}{\p r}} \varrho = 0
    \end{align}

    where $a$ is the speed of sound. The boundary conditions are

    \begin{center}
        \vspace{-3mm}
        \renewcommand{\arraystretch}{1.5}
        \begin{tabular}{ l l l }
            $ \displaystyle \f{\p \varrho}{\p r}\kl{t, 0} = 0 $ & & (instrument closed at mouthpiece) \\
            $ \displaystyle \; \varrho \kl{t, 2\pi} = 0 $ & & (density of surrounding air at the bell of the instrument)
        \end{tabular}
        \vspace{-3mm}
    \end{center}

    In addition, initial conditions at $t = 0$ are given as

    \begin{center}
        \begin{tabular}{ c c c }
            % 1
            $\varrho \kl{0, r} = f\kl{r}$ & and & $\displaystyle \f{\p}{\pt} \varrho \kl{0, r} = g\kl{r}$
        \end{tabular}
    \end{center}

    \begin{enumerate}
        \item {
            Perform separation of variables and find ordinary differential equations that can be used
            to find a solution for this partial differential equation. Solve one of those equations.
        }
        \item {
            Write the factor $R\kl{r}$ in the separation as $\f{p\kl{r}}{r}$ and use this to simplify
            the equation for $R\kl{r}$ into an equation for $p\kl{r}$.
        }
        \item {
            Solve the equation for $R\kl{r}$ and determine the possible frequencies of the oboe.
        }
    \end{enumerate}

    \writen{42}{24}

    %-- Page 9 -----------------------------------------------------------------
    \newpage

    \writen{42}{61}

    %-- Page 10 -----------------------------------------------------------------
    \newpage
    
    \section*{Numerics}

    \subsection*{Exercise 5}
    A plane Cartesian coordinate system is given. The circle $\dom$ is centered at its origin
    $\kl{0, 0}$ and has a radius $r = 5$. The function $u\kl{x,y}$ is defined on the circle $\dom$.

    The function $u\kl{x,y}$ satisfies in $\dom$ Poisson's equation
    \begin{align*}
        \Delta u\kl{x,y} + x \cdot y = 0
    \end{align*}
    and on the boundary of $\dom$ the Dirichlet condition
    \begin{align*}
        u\kl{x,y} = 0
    \end{align*}
    Determine approximate values for $u$ in the four points of $\dom$
    \begin{align*}
       \kl{3,3}, \qquad \kl{-3,3}, \qquad \kl{-3,-3}, \qquad \kl{3,-3}
    \end{align*}
    by the method of finite differences.

    \writen{42}{40}

    %-- Page 11 -----------------------------------------------------------------
    \newpage

    \writen{42}{61}

    %-- Page 12 -----------------------------------------------------------------
    \newpage

    \subsection*{Exercise 6}
    The function $u\kl{x,t}$ is defined on the strip
    \begin{align*}
        \Omega = \ks{0, \pi} \times \left[ 0, \infty \right)
    \end{align*}
    The function $u\kl{x,t}$ satisfies in $\dom$ the wave equation
    \begin{align*}
        u_{xx}\kl{x,  t} = u_{tt}\kl{x,  t}
    \end{align*}
    and on the boundary of $\dom$ the conditions
    \begin{align*}
        u\kl{0,  t} = 0 \qquad \text{and} \qquad u\kl{\pi,  t} = 0
    \end{align*}
    as well as
    \begin{align*}
        u\kl{x,  0} = \sin\kl{x}^4 \qquad \text{and} \qquad u_t\kl{x,  0} = \sin\kl{x}^2
    \end{align*}
    Determine approximate values for $u$ in the three points for $\dom$
    \begin{align*}
        \kl{\fs{\pi}{4},2}, \qquad \kl{\fs{\pi}{2},2}, \qquad \kl{\fs{3\pi}{4},2}
    \end{align*}
    by the method of finite differences. Use the leapfrog method with
    \begin{align*}
        \dx = \f{\pi}{4} \qquad \text{and} \qquad \dt = 1
    \end{align*}
    
    \writen{42}{36}

    %-- Page 13 -----------------------------------------------------------------
    \newpage

    \writen{42}{61}

    %-- Page 14 -----------------------------------------------------------------
    \newpage

    \subsection*{Exercise 7}
    A plane Cartesian coordinate system is given. The triangle $\dom$ is defined by its vertices
    \begin{align*}
        \kl{0,0}, \qquad \kl{8,0}, \qquad \kl{0,8}
    \end{align*}
    The function $u\kl{x,y}$ is defined on the triangle $\dom$. The function $u\kl{x,y}$
    satisfies in $\dom$ Poisson's equation
    \begin{align*}
        \Delta u\kl{x,y} + 1 = 0
    \end{align*}
    and on the boundary of $\dom$ the Dirichlet condition
    \begin{align*}
        u\kl{x,y} = 0
    \end{align*}
    Determine approximate values for $u$ in the four points of $\dom$
    \begin{align*}
        \kl{1,5}, \qquad \kl{1,3}, \qquad \kl{3,1}, \qquad \kl{5,1}
    \end{align*}
    by the method of finite volumes à la Voronoi.
    
    \writen{42}{40}

    %-- Page 15 -----------------------------------------------------------------
    \newpage

    \writen{42}{61}

    %-- Page 16 -----------------------------------------------------------------
    \newpage

    \subsection*{Exercise 8}
    The real function $u\kl{x}$ is defined on the interval
    \begin{align*}
        \dom = \ks{0, 1}
    \end{align*}
    The function satisfies in $\dom$ the differential equation
    \begin{align*}
        u''{\left(x\right)} + 2 = 0
    \end{align*}
    and on the boundary of $\dom$ the conditions
    \begin{align*}
        u\kl{0} = 1 \qquad \text{and} \qquad u\kl{1} = 1
    \end{align*}
    Determine an approximation function $\widetilde{u}\kl{x}$ for $u\kl{x}$.
    
    Apply the finite elements method.

    Use meshes
    \begin{align*}
        \ks{0,0.25}, \qquad \ks{0.25,0.75}, \qquad \ks{0.75,1}
    \end{align*}
    and linear shape functions.

    \writen{42}{40}

    %-- Page 17 -----------------------------------------------------------------
    \newpage

    \writen{42}{61}

\end{document}