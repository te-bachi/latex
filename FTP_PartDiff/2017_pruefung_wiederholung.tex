

%-------------------------------------------------------------------------------
% Dokumenten Klasse
\documentclass[
	final,
	a4paper,
	oneside,
	parskip=full,
	headings=standardclasses,
	headings=big,
	pointednumbers
]{scrartcl}

%-------------------------------------------------------------------------------
% Packete nutzen
\usepackage[T1]{fontenc}
\usepackage[utf8]{inputenc}
\usepackage[ngerman]{babel}
\usepackage[left=20mm,right=20mm,top=10mm,bottom=25mm]{geometry}
\usepackage{amsmath}
\usepackage{amssymb}
\usepackage{mathtools}

%-------------------------------------------------------------------------------
% Für enumerate
\usepackage{enumitem}
\setlist[enumerate]{
    wide=0pt,
    leftmargin=*,
    itemsep=-1ex,
    parsep=2ex,
    labelsep=1ex,
    label=\alph*)
}

%-------------------------------------------------------------------------------
% TikZ
\usepackage{tikz}
\usetikzlibrary{positioning, arrows, decorations}

%-------------------------------------------------------------------------------
% \ifthenelse
\usepackage{ifthen}

%-------------------------------------------------------------------------------
% Links
\usepackage{hyperref}
\hypersetup{
    colorlinks=true,
    linkcolor=red,
    filecolor=magenta,      
    urlcolor=cyan,
}

%-------------------------------------------------------------------------------
% 
\usepackage{xparse}
\NewDocumentCommand{\dx}{ O{} }{{\Delta x^{#1}}}
\NewDocumentCommand{\dy}{ O{} }{{\Delta y^{#1}}}
\NewDocumentCommand{\dt}{ O{} }{{\Delta t^{#1}}}
\NewDocumentCommand{\du}{ O{} }{{\Delta u^{#1}}}
\NewDocumentCommand{\px}{ O{} }{{\partial x^{#1}}}
\NewDocumentCommand{\py}{ O{} }{{\partial y^{#1}}}
\NewDocumentCommand{\pt}{ O{} }{{\partial t^{#1}}}
\NewDocumentCommand{\pu}{ O{} }{{\partial u^{#1}}}
\NewDocumentCommand{\p}{ O{} }{{\partial^{#1}}}
\NewDocumentCommand{\re}{ O{} }{{\mathbb{R}^{#1}}}
\NewDocumentCommand{\lap}{}{\mathcal{L}}

\newcommand{\f}[2]{\frac{#1}{#2}}
\newcommand{\fs}[2]{{\tfrac{#1}{#2}}}

% kl = ()
\newcommand{\kl}[1]{{\left( #1 \right)}}

% kq = {}
\newcommand{\kq}[1]{{\left\{ #1 \right\}}}

% ks = []
\newcommand{\ks}[1]{{\left[ #1 \right]}}

\newcommand{\dom}{{\Omega}}
\newcommand{\bound}{{\partial \Omega}}
\newcommand{\e}{\mathrm{e}}


\newcommand{\writen}[2]{%  
    \begin{tikzpicture}[]
        
        % draw vertical lines
        \foreach \x in {0,...,#1}
        {
            \draw[lightgray] (\x * 4mm, 0mm) -- (\x * 4mm, #2 * 4mm);
        }

        % draw horizontal lines
        \foreach \y in {0,...,#2}
        {
            \draw[lightgray] (0mm, \y * 4mm) -- (#1 * 4mm, \y * 4mm);
        }

        % border
        \draw[black] (0mm, 0mm) -- (#1 * 4mm, 0mm);
        \draw[black] (0mm, 0mm) -- (0mm, #2 * 4mm);
        \draw[black] (#1 * 4mm, 0mm) -- (#1 * 4mm, #2 * 4mm);
        \draw[black] (0mm, #2 * 4mm) -- (#1 * 4mm, #2 * 4mm);
    \end{tikzpicture}
}

% Einzug mit Dimension (z.B. 1cm)
\newenvironment{myindent}[1]{% 
    \parskip=6pt \parindent=0pt \raggedright 
    \def\lititem{\hangindent=#1 \hangafter=1}
}{%
    \par\ignorespaces
}

\newenvironment{hint}[2]{% 
    \begin{minipage}[t]{#2}
        \textit{#1}
    \end{minipage}
    \begin{minipage}[t]{\linewidth - #2}
}{%
    \end{minipage}
} 

%-------------------------------------------------------------------------------
% Paragraph
\setlength{\parindent}{0em}
\setlength{\parskip}{0.35em}

%-------------------------------------------------------------------------------
% Dokument
\title{Wiederholungsprüfung 2017}
\author{}
\date{}
\begin{document}

    %-- Page 1 -----------------------------------------------------------------
    \maketitle

    %-- Page 2 -----------------------------------------------------------------
    \newpage
    \section*{Grundlagen / Analytisch}
    \subsection*{Aufgabe 1}

    Gegeben ist die Differentialgleichung
    \begin{align}
        \frac{\p[4] u}{\pt[4]} + \frac{\pu}{\px} = 0
    \end{align}
    auf dem Gebiet
    \begin{align*}
        \Omega = \kq{\kl{x, t} \mid t > 0 \land x > 0}
    \end{align*}
    mit Randbedingungen
    \begin{align}
        u\kl{x, 0} &= 0, & \f{\pu}{\pt}\kl{x, 0} &= 0, & \f{\p[2] u}{\pt[2]}\kl{x, 0} &= 0, & \f{\p[3] u}{\pt[3]}\kl{x, 0} &= 1,\\
        u\kl{0, t} &= \e^{-t} \nonumber
    \end{align}
    \begin{enumerate}
        \item{
            Führen Sie Laplace-Transformation nach der Variablen $t$ durch und stellen Sie eine
            Differentialgleichung für $\lap u \kl{x, s}$ samt Anfangsbedingungen auf.
        }
        \item{
            Bestimmen Sie die Funktion $\lap u \kl{x, s}$.
        }
    \end{enumerate}
    \begin{hint}{Hinweis.}{2cm}
        Die gewöhnliche Differentialgleichung $y' + ay = b$ mit Anfangsbedingung
        $y\kl{0} = c$ hat die allgemeine Lösung
    \end{hint}
    \begin{align*}
        y\kl{x} = \kl{c - \f{b}{a}} \cdot \e^{-a x} + \f{b}{a}
    \end{align*}

    \writen{42}{36}

    %-- Page 3 -----------------------------------------------------------------
    \newpage

    \writen{42}{66}

    %-- Page 4 -----------------------------------------------------------------
    \newpage

    \subsection*{Aufgabe 2}

    Sei $0 < y_0 < 1$ gegeben und $\dom$ sei das Gebiet
    \begin{align}
        \dom = \kq{\kl{x, y} \mid 0 < x < 1 \land y_0 < y < 1 }
    \end{align}
    Die Funktion $u$ erfüllt in $\dom$ die partielle Differentialgleichung
    \begin{align}
        \f{\pu}{\px} + \f{1}{y^2}\f{\pu}{\py} = 1
    \end{align}
    und die Randbedingung
    \begin{align*}
        u\kl{x, y_0} = \cos\kl{x}, \qquad 0 < x < 1.
    \end{align*}
    \begin{enumerate}
        \item{
            Finden Sie eine Funktion $u$ mit diesen Eigenschaften.
        }
        \item{
            Ist die Lösung $u$ durch die gegebenen Randbedingungen eindeutig bestimmt?
        }
    \end{enumerate}

    \writen{42}{48}

    %-- Page 5 -----------------------------------------------------------------
    \newpage

    \writen{42}{66}
    
    %-- Page 6 -----------------------------------------------------------------
    \newpage
    
	\subsection*{Aufgabe 3}

    Auf dem Gebiet
    \begin{align*}
        \Omega = \left\{\left(x, y\right) \mid 1 < x^2 + y^2 \right\}
    \end{align*}
    soll die folgende partielle Differentialgleichung in Polarkoordinaten
    \begin{align}
        \frac{\partial^2 u}{\partial r^2} - \frac{\partial^2 u}{\partial \varphi^2} = 0
    \end{align}
    gelöst werden. Auf welchem Teil des Randes $x^2 + y^2 = 1$ müssen Randwerte vorgegeben werden,
    damit die Lösung $u{\left(x,  y\right)}$ für Punkte mit $x > 0$ und $x^2 + y^2 = 4$ eindeutig bestimmt ist?

    \writen{42}{50}

    %-- Page 7 -----------------------------------------------------------------
    \newpage

    \writen{42}{66}

    %-- Page 8 -----------------------------------------------------------------
    \newpage

    \subsection*{Aufgabe 4}

    Die biharmonische Gleichung
    \begin{align}
        \Delta \du = 0
    \end{align}
    soll gelöst werden. Man kann sie auch als
    \begin{align}
        \f{\p[4]u}{\px[4]} + 2 \f{\p[4]u}{\px[2]\py[2]} + \f{\p[4]u}{\py[4]} = 0 \label{eq:biharm}
    \end{align}
    schreiben. Verwenden Sie einen Separationsansatz der Form $u\kl{x,y} = X\kl{x} \cdot Y\kl{y}$
    und leiten Sie gewöhnliche Differentialgleichungen für $X(x)$ und $Y(y)$ her,
    mit denen das Problem gelöst werden kann.
    
    \begin{hint}{Hinweis.}{2cm}
        Unmittelbar nach Einsetzen des Separationsansatzes lässt sich die Differentialgleichung $\kl{\ref{eq:biharm}}$
        noch nicht separieren. Schreiben Sie sie in der Form
        \begin{align*}
            E\kl{x} + \ks{F\kl{x} \cdot G\kl{y}} + H\kl{y} = 0
        \end{align*}
        und leiten Sie nach $x$ und nach $y$ ab. Zeigen Sie dann, dass mindestens einer der Terme
        $F\kl{x}$ und $G\kl{y}$ konstant sein muss, schreiben Sie $F\kl{x} = \lambda_1$ im
        ersten Fall und $G\kl{y} = \lambda_2$ im zweiten Fall. In jedem dieser zwei Fälle
        wird die Separation möglich.
        
        \vspace{2mm}
        Die Differentialgleichungen müssen nicht gelöst werden.
    \end{hint}


    \writen{42}{40}

    %-- Page 9 -----------------------------------------------------------------
    \newpage

    \writen{42}{66}

    %-- Page 10 -----------------------------------------------------------------
    \newpage
    
    \section*{Numerik}

    \subsection*{Aufgabe 5}
    In der Ebene ist ein kartesisches Koordinatensystem vorgegeben. \\
    Die reelle Funktion $u\kl{x,y}$ ist auf dem Quadrat
    \begin{align*}
        \Omega = \ks{0, 1} \times \ks{0, 1}
    \end{align*}
    definiert. Sie erfüllt im Inneren von $\Omega$ die Differentialgleichung
    \begin{align*}
        \du \kl{x,  y} = 0
    \end{align*}
    und auf dem Rand von $\Omega$ die Bedingung
    \begin{align*}
        u{\left(x,  0\right)} = 0, \quad
        u{\left(0,  y\right)} = 0, \quad
        u{\left(x,  1\right)} = 0
    \end{align*}
    sowie
    \begin{align*}
        \f{\pu}{\p n}\kl{1,  y} = 1
    \end{align*}
    Gesucht ist eine Approximation der Werte von $u$ in den vier Punkten von $\Omega$
    \begin{align*}
        \begin{pmatrix}
            {^1/_3} \\ {^1/_3}
        \end{pmatrix},\quad
        \begin{pmatrix}
            {^2/_3} \\ {^1/_3}
        \end{pmatrix},\quad
        \begin{pmatrix}
            {^1/_3} \\ {^2/_3}
        \end{pmatrix},\quad
        \begin{pmatrix}
            {^2/_3} \\ {^2/_3}
        \end{pmatrix}
    \end{align*}
    Benutzen Sie hierzu geeignete Finite Differenzen.

    \writen{42}{35}

    %-- Page 11 -----------------------------------------------------------------
    \newpage

    \writen{42}{66}

    %-- Page 12 -----------------------------------------------------------------
    \newpage

    \subsection*{Aufgabe 6}
    Die reelle Funktion $u{\left(x,  t\right)}$ ist auf dem Streifen
    \begin{align*}
        \Omega = \left[ 0, 4 \right] \times \left[ 0, \infty \right)
    \end{align*}
    definiert. Sie erfüllt im Inneren von $\Omega$ die Wärmeleitungsgleichung
    \begin{align*}
        u_{xx}\kl{x,  t} = u_{t}\kl{x,  t}
    \end{align*}
    und auf dem Rand von $\Omega$ die Bedingung
    \begin{align*}
        u{\left(0,  t\right)} = 0 \qquad \text{und} \qquad u{\left(4,  t\right)} = 0
    \end{align*}
    sowie
    \begin{align*}
        u{\left(x,  0\right)} = x^2 \cdot \kl{4 - x}^2
    \end{align*}
    Gesucht ist eine Approximation der Werte von $u$ in den drei Punkten von $\Omega$
    \begin{align*}
        \kl{^1/_{2}}, \qquad \kl{^2/_{2}}, \qquad \kl{^3/_{2}}
    \end{align*}
    Benutzen Sie hierzu das Verfahren von Richardson mit Schrittweiten
    \begin{align*}
        \Delta{x} = 1 \qquad \text{und} \qquad \Delta{t} = 1
    \end{align*}
    
    \writen{42}{40}

    %-- Page 13 -----------------------------------------------------------------
    \newpage

    \writen{42}{66}

    %-- Page 14 -----------------------------------------------------------------
    \newpage

    \subsection*{Aufgabe 7}
    In der Ebene ist ein kartesisches Koordinatensystem vorgegeben. \\
    Das Gebiet $\Omega$ wird durch das Polygon mit Eckpunkten (in Reihenfolge)
    \begin{align*}
        \kl{0, 0}, \quad \kl{6, 0}, \quad \kl{6, 2}, \quad
        \kl{4, 2}, \quad \kl{4, 4}, \quad \kl{0, 4}, \quad \kl{0, 0}
    \end{align*}
    berandet. Die reelle Funktion $u\kl{x,y}$ ist auf $\Omega$ definiert.
    Sie erfüllt im Innern von $\Omega$ die Differentialgleichung
    \begin{align*}
        \du\kl{x,y} = 0
    \end{align*}
    und auf dem Rand von $\Omega$ die Bedingungen
    \begin{align*}
        u\kl{x,y} = 0 \quad \text{falls} \quad y \neq 0.
    \end{align*}
    Weiter sind folgende Werte von $u$ bekannt:
    \begin{align*}
        u\kl{1,0} = 10, \quad u\kl{3,0} = 10, \quad u\kl{5,0} = 10
    \end{align*}
    Gesucht ist eine Approximation der Werte von $u$ in den fünf
    Punkten von $\Omega$
    \begin{align*}
        \kl{1, 1}, \quad \kl{3, 1}, \quad \kl{5, 1}, \quad \kl{1, 3}, \quad \kl{3, 3}
    \end{align*}
    Benutzen Sie hierzu geeignete Finite Volumina.
    
    \writen{42}{38}

    %-- Page 15 -----------------------------------------------------------------
    \newpage

    \writen{42}{66}

    %-- Page 16 -----------------------------------------------------------------
    \newpage

    \subsection*{Aufgabe 8}
    Die reelle Funktion $u{\left(x\right)}$ ist auf dem Intervall
    \begin{align*}
        \Omega = \left[ 0, 1 \right]
    \end{align*}
    definiert. Sie erfüllt im Inneren von $\Omega$ die Differentialgleichung
    \begin{align*}
        -u''\kl{x} = 20
    \end{align*}
    und auf dem Rand von $\Omega$ die Randbedingung
    \begin{align*}
        u\kl{0} = 0, \qquad u\kl{1} = 0
    \end{align*}
    Gesucht ist eine Approximationsfunktion $\tilde{u}\kl{x}$ für $u\kl{x}$. \\
    Benutzen Sie das Verfahren von Gauss. Benutzen Sie die Masche $\ks{0,1}$
    und den Approximationsansatz
    \begin{align*}
        v\kl{x} = a \cdot \kl{x - x^2}
    \end{align*}

    \writen{42}{44}

    %-- Page 17 -----------------------------------------------------------------
    \newpage

    \writen{42}{66}

\end{document}