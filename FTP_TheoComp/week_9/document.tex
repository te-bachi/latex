
%-------------------------------------------------------------------------------
% Dokumenten Klasse
\documentclass[
    final,
    a4paper,
    oneside,
    parskip=full,
    headings=standardclasses,
    headings=big,
    pointednumbers
]{scrartcl}



%-------------------------------------------------------------------------------
% Dokument Sprache
\usepackage[ngerman]{babel}

%-------------------------------------------------------------------------------
% Dokument Sprache
\usepackage[utf8]{inputenc}
\usepackage[T1]{fontenc}

%-------------------------------------------------------------------------------
% Individuelle Kopf- und Fusszeilen

\usepackage{scrlayer-scrpage}

% Bisherige Einstellungen für Kopf- und Fußzeilen löschen:
\clearpairofpagestyles

% Zentriert auf linken Seiten die aktuelle Kapitelüberschrift,
% auf rechten Seiten die Überschrift des aktuellen Abschnitts ausgeben:
\chead{\headmark}

% Zentriert die Seitenzahl ausgeben (auch beim Seitenstil "scrplain"):
\cfoot*{\pagemark}

\pagestyle{scrheadings}

%-------------------------------------------------------------------------------
% Text mit Unterstrich
\usepackage{ulem}

%-------------------------------------------------------------------------------
% Seitenränder und Anmerkungen 
\usepackage[marginparwidth=3cm,
            left=5cm,
            textwidth=12cm]{geometry}
%\usepackage{marginnote}
\reversemarginpar

%\usepackage{showframe}
%\usepackage{layout}

%\usepackage{fourier} 
%\usepackage{mparhack}

%-------------------------------------------------------------------------------
% AMS
\usepackage{amsmath}
\usepackage{amssymb}

%\usepackage{multirow}

%-------------------------------------------------------------------------------
% 
\newcommand{\anmerkung}[2]{%
    \marginline{%
        \vspace{#1} \rule{\marginparwidth}{0.4pt} #2
    }
}

\newcommand{\mytitle}[1]{%
    {\LARGE \bfseries #1}
}

\makeatletter
\newcommand*{\rom}[1]{\expandafter\@slowromancap\romannumeral #1@}
\makeatother

%-------------------------------------------------------------------------------
% Blinktext
%\usepackage{blindtext}

\usepackage{csquotes}

\usepackage{epigraph}
%\epigraphfontsize{\small\itshape}
\setlength\epigraphwidth{8cm}
%\setlength\epigraphwidth{.8\textwidth}
\setlength\epigraphrule{1pt}

\renewcommand*{\dictumwidth}{\textwidth}
\renewcommand*{\raggeddictum}{\raggedright}
\renewcommand*{\raggeddictumtext}{\raggedright}
\renewcommand*{\dictumrule}{}
\renewcommand*{\dictumauthorformat}[1]{--- \textup{#1}}

%-------------------------------------------------------------------------------
% Dokument
\begin{document}
    %\layout
    %\blindtext
    
    \mytitle{Komplexitästheorie}
    
    Aufgaben $\rightarrow$ Wie Lösen?
    
    Skalierbarkeit / Aufwand
    
    \section{Einführung}
    algebraische Komplexitätstheorie (KT)
    
    \subsection{Rechnen mit Matrizen}
    
    \begin{tabular}{lll}
        Gegeben: & $A = 
                   \begin{pmatrix}
                      a_{11} & a_{12} \\
                      a_{21} & a_{22}
                   \end{pmatrix}_{2\mathrm{x}2}$ &
                   $B =
                   \begin{pmatrix}
                   b_{11} & b_{12} \\
                   b_{21} & b_{22}
                   \end{pmatrix}_{2\mathrm{x}2}$ \\
                 & & \\
        Frage:   & \multicolumn{2}{l}{Wieviele Multiplikationen benötigen wir um $A \cdot B$ zu berechnen?} \\
                 & & \\
        Aufwand: & \multicolumn{2}{l}{Addition, $2\mathrm{x}2$-Matrix: $n^2$ } \\
                 & $\#mult \leq n^3$ &
    \end{tabular}

    \enquote{Gauss-Algorithm is not optimal!}, Volker Strassen, 1968
    
    %\epigraphfontsize{\small\itshape}
    \epigraph{``Begin at the beginning," the King said gravely, ``and go on till you
        come to the end: then stop."}{--- \textup{Lewis Carroll}, Alice in Wonderland}
    
    \dictum[Lewis Carroll, 1968]{
        ``Begin at the beginning, and go on till you come to the end: then stop.''}
    
    \anmerkung{0.35cm}{Bemerkung}
    $n^2 \leq \#mult\left( A  \cdot B \right) \leq n^3$
    
    \anmerkung{0.2cm}{Notation} 
    $m\left( A \cdot B \right) = \min\left\{ \#mult\left( A  \cdot B \right) \right\}$
    
    $m\left( A \cdot B \right) \leq 7$ ?
    
    Sei $ C = A \cdot B = \begin{pmatrix}
        c_{11} & c_{12} \\
        c_{21} & c_{22}
    \end{pmatrix} $
    
    \begin{tabular}{cl|l}
        \rom{1} & $ \left(a_{11} + a_{22} \right) \left(b_{11} + b_{22} \right) $
                & $ C_{11} = \rom{1} + \rom{4} - \rom{5} + \rom{7} $ \\
        \rom{2} & $ \left(a_{21} + a_{22} \right) b_{11} $ 
                & $ C_{21} = \rom{2} + \rom{4} $ \\
        \rom{3} & \\
        \rom{4} & \\
        \rom{5} & \\
        \rom{6} & \\
        \rom{7} &
    \end{tabular}
    
    Gehts noch schneller?
    
    Winograd (1971): $m\left( A \cdot B \right) = 7$ 
    
\end{document}