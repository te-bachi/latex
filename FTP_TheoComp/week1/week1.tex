

%-------------------------------------------------------------------------------
% Dokumenten Klasse
\documentclass[
	final,
	a4paper,
	oneside,
	parskip=full,
	headings=standardclasses,
	headings=big,
	pointednumbers
]{scrartcl}

%-------------------------------------------------------------------------------
% Packete nutzen
\usepackage{ngerman,palatino,setspace}
\usepackage[T1]{fontenc}
\usepackage[utf8]{inputenc}
\usepackage[left=45mm,right=35mm,top=25mm,bottom=25mm]{geometry}
%\usepackage{fullpage}
\usepackage{graphicx}
\usepackage{scrpage2}
\usepackage{listings}
\usepackage[usenames,dvipsnames,svgnames]{xcolor}
\usepackage[hidelinks]{hyperref}
\usepackage{amsmath}
\usepackage{amssymb}
\usepackage{mathtools}
\usepackage{caption}
\usepackage{mdframed}
\usepackage{dashbox}
\usepackage{enumitem}
\usepackage{ulem}
\usepackage{tikz}
\usepackage{array}

\usetikzlibrary{shapes,snakes}

%\usepackage{tabularx}
%\newcolumntype{L}[1]{>{\raggedright\arraybackslash}p{#1}}
%\newcolumntype{C}[1]{>{\centering\arraybackslash}p{#1}}
%\newcolumntype{R}[1]{>{\raggedleft\arraybackslash}p{#1}}

%-------------------------------------------------------------------------------
% myident
% mysep
% myrule
\def\myident{3cm}
\def\mysep{0pt}
\def\myrule{0pt}

%-------------------------------------------------------------------------------
% MyDef

\newsavebox{\mybox}
\newenvironment{MyDef}
[2]
{%
    \setlength{\fboxsep}{\mysep}
    \setlength{\fboxrule}{\myrule}
    \hspace{-\myident}\fbox{\begin{minipage}[t]{\myident}\vspace{-0.5cm}\par\rule{\textwidth}{0.4pt}\par\hfill\textbf{#2}\enskip\,\end{minipage}}\begin{lrbox}{\mybox}\begin{minipage}[t]{\textwidth}\vspace{#1}
}
{%
    \end{minipage}\end{lrbox}\fbox{\usebox{\mybox}}
}

\newenvironment{MyBsp}
[2]
{%
    \setlength{\fboxsep}{\mysep}
    \setlength{\fboxrule}{\myrule}
    \hspace{-\myident}\fbox{\begin{minipage}[t]{\myident}\hfill\textbf{#2}\enskip\,\end{minipage}}\begin{lrbox}{\mybox}\begin{minipage}[t]{\textwidth}\vspace{#1}
}
{%
    \end{minipage}\end{lrbox}\fbox{\usebox{\mybox}}
}

%-------------------------------------------------------------------------------
% mysubsection
\newcommand{\mysubsection}
[1]
{
    \setlength{\fboxsep}{\mysep}
    \setlength{\fboxrule}{\myrule}
    \subsection*{#1}\vspace{-1cm}
    \par
    \fbox{\rule{\textwidth}{0.4pt}}
    \par
    \vspace{-0.9cm}
    \fbox{\rule{\textwidth}{0.4pt}}
}

%-------------------------------------------------------------------------------
% mycircle
\newcommand{\mycircle}[1]{%
    \tikz[baseline=(char.base)]\node[rectangle, rounded corners, draw=red, inner sep=2pt](char){#1} ;}

%-------------------------------------------------------------------------------
% myrect
\newcommand{\myrect}[1]{%
\tikz[baseline=(char.base)]\node[rectangle, draw=blue, inner sep=2pt](char){#1} ;}

%-------------------------------------------------------------------------------
% mdframed
\mdfsetup{%
    leftmargin=0cm,
    skipabove=0.2cm,
    linecolor=black,
    backgroundcolor=gray!10,
    linewidth=0.50pt,
    innerleftmargin=0.2cm,
    innerrightmargin=0.0cm,
    innertopmargin=-0.2cm,
    innerbottommargin=0.1cm
}

\newmdenv[
    innerleftmargin=0.2cm,
    innerrightmargin=0.0cm,
    innertopmargin=-0.2cm,
    innerbottommargin=0.1cm
]{mdmath}

\newmdenv[
    font=\small,
    innerleftmargin=0.2cm,
    innerrightmargin=0.2cm,
    innertopmargin=0.2cm,
    innerbottommargin=0.2cm
]{mdtext}

%-------------------------------------------------------------------------------
% Dokument
\begin{document}
    \section*{Woche 1}
    
    %=== Sprache ===========================================================
    \mysubsection{Sprache}
    
    %--- Menge / Folge -------------------------------------------------------
    \begin{MyDef}{-0.26cm}{Def.}
        \textbf{Menge:}
        
        ungeordnet, keine Doppeleinträge
    \end{MyDef}

    \begin{MyBsp}{-0.88cm}{Bsp.}
        \begin{flalign*}
            M_1                 &= \{ a, b, c, d\} & \\
            M_2                 &= \{ c, d, e, f\} & \\
            M_1 \cap M_2        &= \{ c, d\} & \\
            M_1 \cup M_2        &= \{ a, b, c, d, e, f\} & \\
            M_1 \setminus M_2   &= \{ a, b \} & \\
            a                   & \in M_1 \\
            b                   & \notin M_2
        \end{flalign*}
    \end{MyBsp}
    
    \begin{MyDef}{-0.26cm}{Def.}
        \textbf{Folge:}
        
        geordnet, kann auch Doppeleinträge enthalten
    \end{MyDef}

    \begin{MyBsp}{-0.88cm}{Bsp.}
        \begin{flalign*}
        M &= \{ a, b, c, d\} & \\
        F &= (a,b,a,b,d,d,c,c) & 
        \end{flalign*}
    \end{MyBsp}
    
    %--- Alphabet ------------------------------------------------------------
    \begin{MyDef}{-0.26cm}{Def.}
        \textbf{Alphabet:}
        
        nicht-leere endliche \mycircle{Menge} (von Zeichen)
    \end{MyDef}
    
    \begin{MyBsp}{-0.36cm}{Bsp.}
        \hspace{-0.2cm}
        \begin{tabular}{ll}
            $\Sigma_1 := \left\{ a, b, c, d \right\}$ & \color{Green}{// 'a'} \\
            $\Sigma_2 := \left\{ 0 \right\} $         &
        \end{tabular}
    \end{MyBsp}

    \begin{MyBsp}{-0.36cm}{Nicht Bsp.}
            \hspace{-0.2cm}
            \begin{tabular}{ll}
                $\emptyset \not\subseteq \Sigma $                          & \color{Green}{// leere Menge} \\
                $\left\{ a, b, c, d, \ldots \right\} \not\subseteq \Sigma$ & \color{Green}{// unendliche Menge}
            \end{tabular}
    \end{MyBsp}

    %--- Wort ------------------------------------------------------------
    \begin{MyDef}{-0.26cm}{Def.}
        \textbf{Wort/String:}
        
        über Alphabet $\Sigma$ ist eine endliche \mycircle{Folge} von Zeichen aus $\Sigma$,
        eventuell leer
    \end{MyDef}

    \begin{MyBsp}{-0.36cm}{Bsp.}
        \hspace{-0.2cm}
        \begin{tabular}{lll}
            $acdcd$ & Wort über $\Sigma_1$ & \color{Green}{// ``acdcd``} \\
            $0000 $ & Wort über $\Sigma_2$ &
        \end{tabular}
    \end{MyBsp}

    \begin{MyDef}{-0.26cm}{Def.}
        \textbf{Leeres Wort:}
        
        \hspace{-0.3cm}
        \begin{tabular}{lll}
            $\varepsilon$ über \uline{jedes} Alphabet         & \color{Green}{// `` ``} \\
            \multicolumn{2}{l}{$\varepsilon$ \uline{nicht} Element vom Alphabet, sondern ein Meta-Zeichen } \\
            & \\
            \color{Blue}{Meta-Sprache}   & Über Programmiersprache sprechen \\
            \color{Blue}{Objekt-Sprache} & Programmiersprache
        \end{tabular}
    \end{MyDef}

    \begin{MyDef}{-0.26cm}{Def.}
        \textbf{$\mathbf{\Sigma^*}$}:
        
        Menge \uline{aller} Wörter über $\Sigma$, immer unendlich gross
    \end{MyDef}

    \begin{MyBsp}{-0.36cm}{Bsp.}
        \begin{tabular}{ll}
            $\Sigma_1^* = \left( a, aa, aaa, \ldots, ab, aba, abab, \ldots \right)$ & über $\Sigma_1$ \\
            $\Sigma_2^* =\left( 0, 00, 000 \ldots \right)$ & über $\Sigma_2$
        \end{tabular}
    \end{MyBsp}

    \begin{MyDef}{-0.26cm}{Def.}
        Konkatenation von Wörtern $x$ und $y$ über $\Sigma$ \\
        
        \hspace{-0.3cm}
        \begin{tabular}{llll}
            $x = x_1x_2 \ldots x_n$, & $x_i \in \Sigma$, & $n \geq 0 $ & \color{Green}{// $ x + y $ } \\
            $y = y_1y_2 \ldots y_m$, & $y_i \in \Sigma$, & $m \geq 0 $ & \color{Green}{// ``acdcd`` + ``000`` }  \\
        \end{tabular} \\
        
        $ xy = x \cdot y =  x_1x_2 \ldots x_ny_1y_2 \ldots y_m $ \\
    \end{MyDef}

    %=== Algebra ===========================================================
    \mysubsection{Algebra}
    
    %---Monoid ------------------------------------------------------------
    \begin{MyDef}{-0.26cm}{Def.}
        \textbf{Monoid:}
        
        In der abstrakten Algebra ist ein Monoid eine algebraische Struktur bestehend
        aus einer {\color{Blue}Menge} mit einer {\color{Blue}inneren Verknüpfung} und einem {\color{Blue}neutralen Element}. \\
        
        Sei $M$ eine Menge und $\circ: M \times M \to M$ eine Operation. Das Paar $\left(M, \circ:\right)$ heisst Monoid, falls: \\
        
        \hspace{-0.3cm}
        \begin{tabular}{llll}
            1) & Assoziativität der Verknüpfung & $\forall a,b,c \in M:$ & $(a \circ b) \circ c = a \circ (b \circ c)$ \\
            2) & $e$ ist neutrales Element      & $\forall a\in M:$      & $e \circ a = a \circ e = a$ \\
               &                                &                        & \\
            \multicolumn{4}{l}{Es gilt aber \uline{kein} Kommunikativgesetz für Monoide} \\
            3) & Kommutativität der Verknüpfung & $\forall a,b \in M:$   & $a \circ = b \circ a$ \\
               &                                &                        & \\
            \multicolumn{4}{l}{und nicht notwendigerweise ein inverse Elemente} \\
            4) & $a^{-1}$ ist inverses Element  & $a, a^{-1} \in M:$     & $ a \circ a^{-1} = a^{-1} \circ a = e $ \\
               &                                &                        & \\
        \end{tabular} \\
        
        Ein Monoid ist also eine Halbgruppe mit neutralem Element, hat aber nicht
        notwendigerweise ein inverses Element. Wenn ein inverses Element existiert,
        gehört es zu einer Gruppe.
    \end{MyDef}
    
    \begin{MyBsp}{-0.26cm}{}
        Die natürlichen Zahlen sind nicht genau definiert.
        Sie kann als die positiven ganzen Zahlen (also ohne die 0) \\
        
        $ \mathbb{N} = \{1, 2, 3, \ldots\} $ \\
        
        oder als die nichtnegativen ganzen Zahlen (also inklusive der 0) \\
        
        $ \mathbb{N} = \{0, 1, 2, 3, \ldots\} $ \\
        
        aufgefasst werden. Wir benutzen folgende Definition.
        \begin{flalign*}
            \mathbb{N}    & = \left\{ 0, 1, 2, 3, \ldots \right\} & \\
            \mathbb{N^+}  & = \left\{ 1, 2, 3, \ldots \right\} &
        \end{flalign*}
    \end{MyBsp}
    
    \begin{MyBsp}{-0.36cm}{Bsp.}
        \hspace{-0.3cm}
        \begin{tabular}{lll}
            1) & $ \left(\mathbb{N}, +\right) $       & Monoid mit $e = 0$ \\
            2) & $ \left(\mathbb{N}^+, +\right) $     & Kein Monoid, $0$ fehlt \\
            3) & $ \left(\mathbb{N}^+, \cdot\right) $ & Monoid mit $e = 1$ \\
            4) & $ \left(\Sigma^*, \cdot\right) $     & Monoid mit $e = \varepsilon$ \\
        \end{tabular}
    \end{MyBsp}
    
    %--- Potenzen ------------------------------------------------------------
    \begin{MyDef}{-0.26cm}{Def.}
        \textbf{Potenzen:}
        
        Sei $\left( M, \circ \right)$ ein Monoid mit neuralem Element $e$.
        
        \hspace{-0.3cm}
        \begin{tabular}{lll}
            Sei $a \in M$ & $ a^0 = e $ &  \\
                          & $ a^n = a \circ a^{n-1} $, & $ n \geq 1 $ \\
                          & $ n^n = \underbrace{
                                        a \circ a \circ a \circ \ldots \circ a
                                    }_{
                                        n \textrm{ mal } a, \; (n - 1) \textrm{ mal } \circ
                                    } {\color{Blue} \circ \; e}$ &
        \end{tabular}
    \end{MyDef}

    %=== Formale Sprache ===========================================================
    \mysubsection{Formale Sprache}
    
    
    %--- Sprache ------------------------------------------------------------
    \begin{MyDef}{-0.26cm}{Def.}
        \textbf{Sprache:}
        
        über das Alphabet $\Sigma$: \mycircle{Menge} von Wörtern über $\Sigma$, also $ \subseteq \Sigma^*$. \\
        Kann unendlich sein!
    \end{MyDef}
    
    \begin{MyBsp}{-0.36cm}{Bsp.}
        \hspace{-0.3cm}
        \begin{tabular}{lll}
            $ \left\{ \; \right\} $                             & kleinste Sprache   & \\
            $ \left\{ \varepsilon \right\} $                    & \multicolumn{2}{l}{ein Element: das leere Wort}             \\
            $ \; \Sigma^* $                                     & \multicolumn{2}{l}{grösste Sprache $\to$ unendliche gross!} \\
            $ \left\{ a, aa, ab, ac, dddd \right\}$             & über $\Sigma_1$    & \\
            $ \left\{ 0, 00, 000, \ldots \right\}$              & über $\Sigma_2$    & {\color{Green}// codiert $\mathbb{N^+}$ } \\
            $ \left\{ \varepsilon, 0, 00, 000, \ldots \right\}$ & über $\Sigma_2$    & {\color{Green}// codiert $\mathbb{N}$} \\
        \end{tabular}
    \end{MyBsp}
    
    \begin{MyBsp}{-0.36cm}{Bem.}
        Wort hat hier \uline{keine} Bedeutung!
    \end{MyBsp}

    %--- Sprache ------------------------------------------------------------
    \begin{MyDef}{-0.26cm}{Def.}
        \textbf{Operationen auf Sprachen:}
        
        Seien $L_1$, $L_2$ Sprachen über $\Sigma$
        
        \hspace{-0.3cm}
        \begin{tabular}{lll}
            1) & $ L_1 \cup L_2 $: & normale Vereinigungsmenge \\
            2) & \multicolumn{2}{l}{$ \underbrace{L_1 \cdot L_2}_{Konkat.} =
                \left\{ \; \omega_1 \cdot \omega_2 \mid \omega_1 \in L_1, \; \omega_2 \in L_2 \; \right\} $} \\
               & & \\
               & \multicolumn{2}{l}{Konkatenation auf Sprachen ist Monoid mit $\left\{ \varepsilon \right\} $ als neutrales Element} \\
               & \multicolumn{2}{l}{$\to L^n$ also definiert! } \\
            3) & $L^* = L^0 \cup L^1 \cup L^2 \cup \ldots$ & Kleenescher Stern
        \end{tabular}
    \end{MyDef}
    
    \begin{MyBsp}{-0.87cm}{Bsp.}
        \begin{flalign*}
            \Sigma        & = \left\{ \; a, b, \ldots, z \; \right\} & \\
            L_1           & = \left\{ \; \textrm{good}, \textrm{bad} \; \right\} \quad L_2 = \left\{ \; \textrm{dog}, \textrm{cat} \; \right\}& \\
            L_1 \cup L_2  & = \left\{ \; \textrm{good}, \textrm{bad}, \textrm{dog}, \textrm{cat} \; \right\} & \\
            L_1 \cdot L_2 & = \left\{ \; \textrm{gooddog}, \textrm{goodcat}, \textrm{baddog}, \textrm{badcat} \; \right\} & \\
            L_2^0         & = \left\{ \; \varepsilon \; \right\} & \\
            L_2^1         & = \left\{ \; \textrm{dog}, \textrm{cat} \; \right\} & \\
            L_2^2         & = \left\{ \; \textrm{dogdog}, \textrm{dogcat}, \textrm{catdog}, \textrm{catcat} \; \right\} & \\
            L_2^3         & = L_2 \cdot L_2 \cdot L_2 & \\
                          & = \left\{ \; \textrm{dog}, \textrm{cat} \; \right\} \cdot \left\{ \; \textrm{dog}, \textrm{cat} \; \right\} \cdot \left\{ \; \textrm{dog}, \textrm{cat} \; \right\} & \\
                          & = \left\{ \; \textrm{dogdogdog}, \textrm{dogdogcat}, \textrm{dogcatdog}, \textrm{dogcatcat},  \ldots \; \right\} & \\
            L_2^*         & = L_2^0 \cup L_2^1 \cup L_2^2 \cup \ldots & \\
                          & = \left\{ \; \underbrace{\quad \varepsilon \quad}_{L_0},
                                         \underbrace{\textrm{dog}, \textrm{cat}}_{L_1},
                                         \underbrace{\textrm{dogdog}, \textrm{dogcat}, \ldots }_{L_2} \; \right\} & \\
        \end{flalign*} \\
        
        Ein Wort hat endlich viele Zeichen von einem endlichen Alphabet.\\
        Eine Sprache kann unendlich viele Wörter besitzen.
    \end{MyBsp}
    
    
    %=== Endlicher Automat ===========================================================
    \mysubsection{Endlicher Automat}
    
    Wort $\to$ Rater-Rater $\to$ gehört zu Sprache oder nicht
    
    
    %--- Zustandsübergangsdiagramm ------------------------------------------------------------
    \begin{MyDef}{-0.26cm}{Def.}
        \textbf{Zustandsübergangsdiagramm:}
        
        \vspace{0.5cm}
        \begin{tikzpicture}[scale=0.2]
            \tikzstyle{every node}+=[inner sep=0pt]
            \draw [black] (15.6,-9.8) circle (3);
            \draw (15.6,-9.8) node {$q_1$};
            \draw [black] (41.5,-9.8) circle (3);
            \draw (41.5,-9.8) node {$q_3$};
            \draw [black] (28.8,-9.8) circle (3);
            \draw (28.8,-9.8) node {$q_2$};
            \draw [black] (28.8,-9.8) circle (2.4);
            \draw [black] (7.6,-9.8) -- (12.6,-9.8);
            \fill [black] (12.6,-9.8) -- (11.8,-9.3) -- (11.8,-10.3);
            \draw [black] (18.6,-9.8) -- (25.8,-9.8);
            \fill [black] (25.8,-9.8) -- (25,-9.3) -- (25,-10.3);
            \draw (22.2,-10.3) node [below] {$1$};
            \draw [black] (31.571,-8.667) arc (105.72959:74.27041:13.202);
            \fill [black] (38.73,-8.67) -- (38.09,-7.97) -- (37.82,-8.93);
            \draw (35.15,-7.67) node [above] {$0$};
            \draw [black] (14.277,-7.12) arc (234:-54:2.25);
            \draw (15.6,-2.55) node [above] {$0$};
            \fill [black] (16.92,-7.12) -- (17.8,-6.77) -- (16.99,-6.18);
            \draw [black] (27.477,-7.12) arc (234:-54:2.25);
            \draw (28.8,-2.55) node [above] {$1$};
            \fill [black] (30.12,-7.12) -- (31,-6.77) -- (30.19,-6.18);
            \draw [black] (38.696,-10.85) arc (-75.52313:-104.47687:14.184);
            \fill [black] (31.6,-10.85) -- (32.25,-11.53) -- (32.5,-10.57);
            \draw (35.15,-11.8) node [below] {$0,1$};
        \end{tikzpicture}
        \vspace{0.5cm}
        
        \hspace{-0.3cm}
        \begin{tabular}{lm{4.5cm}m{2cm}}
            1) & Zustand                    & 
                \begin{tikzpicture}[scale=1.0]
                    \tikzstyle{every node}+=[inner sep=0pt]
                    \draw [black] (0,0) circle (0.5);
                    \draw (0,0) node {$q_1$};
                \end{tikzpicture} \\
            2) & Akzeptierender Zustand    & 
                \begin{tikzpicture}[scale=1.0]
                    \tikzstyle{every node}+=[inner sep=0pt]
                    \draw [black] (0,0) circle (0.5);
                    \draw [black] (0,0) circle (0.6);
                    \draw (0,0) node {$q_2$};
                \end{tikzpicture} \\
            3) & Transition                & 
                \begin{tikzpicture}[scale=0.2]
                    \draw [black] (7.6,-9.8) -- (12.6,-9.8);
                    \fill [black] (12.6,-9.8) -- (11.8,-9.3) -- (11.8,-10.3);
                    \draw (10,-9.8) node [below] {$1$};
                \end{tikzpicture} \\
            4) & Zustände                  & $q_1$, $q_2$, $q_3$ \\
            5) & Startzustand              & $q_1$ \\
            6) & Akzeptierende Zustände    & $q_2$
        \end{tabular}
    \end{MyDef}
    
    \begin{MyBsp}{-0.26cm}{Bsp.}
        \textbf{Bedeutung:}
        
        Berechne Eingabe : $1101 \in \Sigma^*$ \\
        
        \hspace{-0.3cm}
        \begin{tabular}{llll}
            1) & Start      &                            & $q_1$ \\
            2) & Lese       & \myrect{$1$} $1$ $0$ $1$   & $q_1 \xrightarrow{\;1\;} q_2$ \\
            3) & Lese       & $1$ \myrect{$1$} $0$ $1$   & $q_2 \xrightarrow{\;1\;} q_2$ \quad (self) \\
            4) & Lese       & $1$ $1$ \myrect{$0$} $1$   & $q_2 \xrightarrow{\;0\;} q_3$ \\
            5) & Lese       & $1$ $1$ $0$ \myrect{$1$}   & $q_3 \xrightarrow{\;1\;} q_2$ \\
            6) & \multicolumn{3}{l}{Akzeptiere 1101, weil Endzustand $q_2$ akzeptieren ist.}\\
        \end{tabular}
    \end{MyBsp}
    
    %--- DFA ------------------------------------------------------------
    \begin{MyDef}{-0.26cm}{Def.}
        \textbf{Deterministic Finite Automaton (DFA):}
        
        Ein DFA ist ein 5-Tupel $(Q,\Sigma,\delta, s, F)$ wobei \\
        
        \hspace{-0.3cm}
        \begin{tabular}{lll}
            1) & $Q$                                             & endliche \mycircle{Menge} von Zuständen \\
            2) & $\Sigma$                                        & Eingangsalphabet \\
            3) & $\delta: Q \times \Sigma \xrightarrow{\;\;} Q$: & \uline{einen} nachfolgenden Zustand \\
            4) & $s \in Q$                                       & Startzustand \\
            5) & $F \subseteq Q$                                 & \mycircle{Menge} akzeptierender Zustände
        \end{tabular}
    \end{MyDef}
    
    %--- Unser Beispiel -------------------------------------------------------
    \begin{MyBsp}{-0.26cm}{Bsp.}
        \textbf{Unser Beispiel:} \\
        
        $Q = \left\{ \; q_1, q_2, q_3 \; \right\} $ \quad
        $\Sigma = \left\{ \; 0, 1 \; \right\} $ \\
        
        \hspace{-0.3cm}
        \begin{minipage}[t]{0.3\textwidth}
            \begin{tabular}{ll}
                $ \delta(q_1,0) $ & $= q_1$ \\
                $ \delta(q_1,1) $ & $= q_2$ \\
                \hline
                $ \delta(q_2,0) $ & $= q_3$ \\
                $ \delta(q_2,1) $ & $= q_2$ \\
                \hline
                $ \delta(q_3,0) $ & $= q_2$ \\
                $ \delta(q_3,1) $ & $= q_2$ \\
            \end{tabular}
        \end{minipage} \begin{minipage}[t]{0.3\textwidth}
            \begin{tabular}{l|ll}
                $ \delta $ & 0     & 1     \\
                \hline
                $ q_1 $    & $q_1$ & $q_2$ \\
                $ q_2 $    & $q_3$ & $q_2$ \\
                $ q_3 $    & $q_2$ & $q_2$
            \end{tabular}
        \end{minipage} \\
        
        $s = q_1 $ \quad
        $F = \left\{ \; q_2 \; \right\} $ \\
    \end{MyBsp}
    
    %--- Berechne  ------------------------------------------------------------
    \begin{MyBsp}{-0.26cm}{}
        \textbf{Berechne (Semantik):} \\
        
        Sei $M = (Q,\Sigma,\delta, s, F)$ eine DFA\\
        Sei $\omega = \omega_1\omega_2\ldots\omega_n$ ein Wort aus $\Sigma^*$ mit $w_i \in \Sigma$, $n \geq 0$ \\
        
        $M$ akzeptiert $\omega$, falls eine \mycircle{Folge} von Zuständen existiert: \\
        $ r_0, r_1, \ldots, r_n$ in $Q$ mit \\
        
        \hspace{-0.3cm}
        \begin{tabular}{ll}
            1) & $ r_0 = s $ \\
            2) & $ r_i = \delta\left( r_{i-1}, \omega_i \right) $ \\
            3) & $ r_n \in F $
        \end{tabular} \\
        \\
        Andernfalls wird $\omega$ \uline{verworfen}.
    \end{MyBsp}
    
    %--- Erkennen ------------------------------------------------------------
    \begin{MyDef}{-0.26cm}{Def.}
        $M$ erkennt Sprache $A$ falls \\
        
        $ A = \left\{ \; \omega \in \Sigma^* \mid M \textrm{ akzeptiert } \omega \; \right\}$
    \end{MyDef}
    
    %--- Bsp 1  ------------------------------------------------------------
    \begin{MyBsp}{-0.26cm}{Bem.}
        \begin{tikzpicture}
            \node (M) at (0,0) {$M$};
            \node (A)  [text width=3cm]       at (3,0.5)    {akzeptiert Wort};
            \node (A2) [text width=3cm, blue] at (6.5,0.5)  {accept};
            \node (B)  [text width=3cm]       at (3.0,-0.5) {erkennt Sprache};
            \node (B2) [text width=3cm, blue] at (6.5,-0.5) {recognize};
            \draw[->,thick] (M) to [in = 180, out = +40] (A);
            \draw[->,thick] (M) to [in = 180, out = -40] (B);
        \end{tikzpicture}
    \end{MyBsp}

    %---Regulär ------------------------------------------------------------
    \begin{MyDef}{-0.26cm}{Def.}
        Eine Sprache $A$ heisst \uline{regulär}, falls ein DFA existiert,
        der die Sprache erkennt.
    \end{MyDef}
    
    %--- Bsp 1  ------------------------------------------------------------
    \begin{MyBsp}{-0.26cm}{Bsp.}
        
        $Q = \left\{ \; q_0 \; \right\} $ \quad
        $\Sigma = \left\{ \; 0, 1 \; \right\} $ \quad
        $s = q_0 $ \quad
        $F = \left\{ \;  \; \right\} $ \\
        
        \begin{tikzpicture}[scale=0.2]
            \tikzstyle{every node}+=[inner sep=0pt]
            \draw [black] (15.6,-9.8) circle (3);
            \draw (15.6,-9.8) node {$q_0$};
            \draw [black] (7.6,-9.8) -- (12.6,-9.8);
            \fill [black] (12.6,-9.8) -- (11.8,-9.3) -- (11.8,-10.3);
            \draw [black] (14.277,-7.12) arc (234:-54:2.25);
            \draw (15.6,-2.55) node [above] {$0,1$};
            \fill [black] (16.92,-7.12) -- (17.8,-6.77) -- (16.99,-6.18);
        \end{tikzpicture} \\
        
        \hspace{-0.3cm}
        \begin{minipage}[t]{0.3\textwidth}
            \begin{tabular}{ll}
                $ \delta(q_0,0) $ & $= q_0$ \\
                $ \delta(q_0,1) $ & $= q_0$ \\
            \end{tabular}
        \end{minipage} \begin{minipage}[t]{0.3\textwidth}
            \begin{tabular}{l|ll}
                $ \delta $ & 0     & 1     \\
                \hline
                $ q_0 $    & $q_0$ & $q_0$
            \end{tabular}
        \end{minipage} \\
        \\
        
        \begin{tabular}{ll}
            \uline{akzeptiert} & kein einziges Wort\\
            \uline{erkennt}    & $\emptyset$ \\
        \end{tabular}
    \end{MyBsp}
    
    %--- Bsp 2  ------------------------------------------------------------
    \begin{MyBsp}{-0.26cm}{Bsp.}
        
        $Q = \left\{ \; q_0 \; \right\} $ \quad
        $\Sigma = \left\{ \; 0, 1 \; \right\} $ \quad
        $s = q_0 $ \quad
        $F = \left\{ \; q_0 \; \right\} $ \\
        
        \begin{tikzpicture}[scale=0.2]
            \tikzstyle{every node}+=[inner sep=0pt]
            \draw [black] (15.6,-9.8) circle (3);
            \draw (15.6,-9.8) node {$q_0$};
            \draw [black] (15.6,-9.8) circle (2.4);
            \draw [black] (7.6,-9.8) -- (12.6,-9.8);
            \fill [black] (12.6,-9.8) -- (11.8,-9.3) -- (11.8,-10.3);
            \draw [black] (14.277,-7.12) arc (234:-54:2.25);
            \draw (15.6,-2.55) node [above] {$0,1$};
            \fill [black] (16.92,-7.12) -- (17.8,-6.77) -- (16.99,-6.18);
        \end{tikzpicture} \\
        
        \hspace{-0.3cm}
        \begin{minipage}[t]{0.3\textwidth}
            \begin{tabular}{ll}
                $ \delta(q_0,0) $ & $= q_0$ \\
                $ \delta(q_0,1) $ & $= q_0$ \\
            \end{tabular}
        \end{minipage} \begin{minipage}[t]{0.3\textwidth}
            \begin{tabular}{l|ll}
                $ \delta $ & 0     & 1     \\
                \hline
                $ q_0 $    & $q_0$ & $q_0$
            \end{tabular}
        \end{minipage} \\
        \\
        
        \begin{tabular}{ll}
            \uline{akzeptiert} & jedes Wort \\
            \uline{erkennt}    & $\Sigma^*$ \\
        \end{tabular}
    \end{MyBsp}
\end{document}
