
%-------------------------------------------------------------------------------
% Dokumenten Klasse
\documentclass[
    final,
    a4paper,
    oneside,
    parskip=full,
    headings=standardclasses,
    headings=big,
    pointednumbers
]{scrartcl}



%-------------------------------------------------------------------------------
% Dokument Sprache
\usepackage[ngerman]{babel}

%-------------------------------------------------------------------------------
% Dokument Sprache
\usepackage[utf8]{inputenc}
\usepackage[T1]{fontenc}

%-------------------------------------------------------------------------------
% Individuelle Kopf- und Fusszeilen

\usepackage{scrlayer-scrpage}

% Bisherige Einstellungen für Kopf- und Fußzeilen löschen:
\clearpairofpagestyles

% Zentriert auf linken Seiten die aktuelle Kapitelüberschrift,
% auf rechten Seiten die Überschrift des aktuellen Abschnitts ausgeben:
\chead{\headmark}

% Zentriert die Seitenzahl ausgeben (auch beim Seitenstil "scrplain"):
\cfoot*{\pagemark}

\pagestyle{scrheadings}

%-------------------------------------------------------------------------------
% Text mit Unterstrich
\usepackage{ulem}

%-------------------------------------------------------------------------------
% Seitenränder und Anmerkungen 
\usepackage[marginparwidth=3cm,
            left=5cm,
            textwidth=12cm]{geometry}
%\usepackage{marginnote}
\reversemarginpar

%\usepackage{showframe}
%\usepackage{layout}

%\usepackage{fourier} 
%\usepackage{mparhack}

%-------------------------------------------------------------------------------
% AMS
\usepackage{amsmath}
\usepackage{amssymb}

%\usepackage{multirow}

%-------------------------------------------------------------------------------
% 
\newcommand{\anmerkung}[2]{%
    \marginline{%
        \vspace{#1} \rule{\marginparwidth}{0.4pt} #2
    }
}

\newcommand{\mytitle}[1]{%
    {\LARGE \bfseries #1}
}

\makeatletter
\newcommand*{\rom}[1]{\expandafter\@slowromancap\romannumeral #1@}
\makeatother

%-------------------------------------------------------------------------------
% Blinktext
%\usepackage{blindtext}

\usepackage{csquotes}

\usepackage{epigraph}
%\epigraphfontsize{\small\itshape}
\setlength\epigraphwidth{8cm}
%\setlength\epigraphwidth{.8\textwidth}
\setlength\epigraphrule{1pt}

\renewcommand*{\dictumwidth}{\textwidth}
\renewcommand*{\raggeddictum}{\raggedright}
\renewcommand*{\raggeddictumtext}{\raggedright}
\renewcommand*{\dictumrule}{}
\renewcommand*{\dictumauthorformat}[1]{--- \textup{#1}}

%-------------------------------------------------------------------------------

\usepackage{tikz}

\usetikzlibrary{shapes,snakes,calc,arrows}

%-------------------------------------------------------------------------------
% Dokument
\begin{document}
    %\layout
    %\blindtext
    
    \mytitle{Komplexitästheorie}
    
    \section{Polyzeit-Reduktion}
    
    \begin{itemize}
    	\setlength{\itemindent}{-12pt}
    	\setlength{\itemsep}{-10pt}
    	%\itemsep-10pt
    	\renewcommand{\labelitemi}{$-$}
    	\item Beispiele zu Reduktionen
    	\item Definition davon verstehen
    	\item Intuitiv Verstehen was eine Reduktion ist
    	\item Viele schwarze Punkt
    	\item Jeder Punkt ist ein Entscheidungsproblem resp. eine formale Sprache
    	\item Eine Reduktion ist eine Verbindung zwischen zwei Problemen
    \end{itemize}

    \begin{tikzpicture}[every node/.style={fill, circle, inner sep = 1pt}]
	    \pgfmathsetseed{1}
    	\foreach \x in {1,...,8} {
    		\foreach \y in {1,...,8} {
			    %\node[label=above:$\x \y$] (A\x\y) at (\x + rand, rand) {};
			    \node (A\x\y) at (\x + rand, rand) {};
			}
		}
	
		\iffalse
			\draw[red,thick,dashed] (A11) -- (A12);
			\draw[blue,thick,dashed] (A11) -- (A13);
			\draw[green,thick,dashed] (A11) -- (A14);
			\draw (A11) -- (A15);
			\draw (A11) -- (A16);
			\draw (A11) -- (A17);
			\draw (A11) -- (A18);
			
			\draw[red,thick,dashed] (A21) -- (A22);
			\draw[blue,thick,dashed] (A21) -- (A23);
			\draw[green,thick,dashed] (A21) -- (A24);
			\draw[red,thick,dashed] (A21) -- (A25);
			\draw[blue,thick,dashed] (A21) -- (A26);
			\draw[green,thick,dashed] (A21) -- (A27);
			\draw (A21) -- (A28);
		\fi
		
		\draw (A14) -- (A28);
    \end{tikzpicture}
    
    \iffalse
    \begin{tikzpicture}
		\node (M) at (0,0) {$M$};
		\node (A)  [text width=3cm]       at (3,0.5)    {akzeptiert Wort};
		\node (A2) [text width=3cm, blue] at (6.5,0.5)  {accept};
		\node (B)  [text width=3cm]       at (3.0,-0.5) {erkennt Sprache};
		\node (B2) [text width=3cm, blue] at (6.5,-0.5) {recognize};
		\draw[->,thick] (M) to [in = 180, out = +40] (A);
		\draw[->,thick] (M) to [in = 180, out = -40] (B);
	\end{tikzpicture}
	\fi

	Denn es kommt häufig vor, dass man irgendwie erkennt: 2 Probleme sind ziemlich ähnlich.
	
	Oft sieht man \underline{ein} Problem ist nur ein \underline{anderes} Problem mit ein
	bisschen anderer Verkleidung.
	
	Oder man stellt fest: wenn ich das \underline{eine} Problem lösen kann, dann kann ich mit dem
	Entscheidungsverfahren für dieses auch das \underline{andere} ganz einfach lösen.
	
	Wenn man so etwas gefunden hat, dann ist es eine Reduktion, von dem \underline{einen} auf das
	\underline{andere} Problem.
	
	\begin{tikzpicture}[every node/.style={draw, circle, inner sep = 5pt}]
		\node (A) at (0, 0) [circle] {$A$};
		\node (B) at (3, 0) [circle] {$B$};
		\draw[line width=1pt, arrows=-angle 60] (A) to (B);
	\end{tikzpicture}
	
	Wenn wir Problem $B$ lösen können und eine Reduktion von $A$ auf $B$, dann kann man
	$A$ auch lösen. Es ist wie eine Übersetzung von einem in ein anderes Problem.
	
	Ihr kennt das sicher, wenn ihr code schreibt, denkt ihr, moment, so etwas habe ich
	gleich schon mal geschrieben.
	
	Wenn ihr feststellt, Code, den man wiederverwenden kann. Denn er löst ein Problem, dass,
	wenn man dieses Problem lösen kann, kann man mit diesem Problem andere Probleme einfacher
	lösen.
    
    \subsection{Rechnen mit Matrizen}
    
    \begin{flalign*}
    	L_1  & = \left\{ \omega \mid \textrm{ Länge } \vert w\vert_b \textrm{ lässt bei Division durch $3$ den Rest $1$ } \right\} & \\
    	L_2  & = \left\{ \omega \mid \textrm{ Länge } \vert w\vert_b \textrm{ ist durch $3$ teilbar } \right\} &
    \end{flalign*}
    
\end{document}