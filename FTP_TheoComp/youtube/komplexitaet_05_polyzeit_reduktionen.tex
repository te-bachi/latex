
%-------------------------------------------------------------------------------
% Dokumenten Klasse
\documentclass[
    final,
    a4paper,
    oneside,
    parskip=full,
    headings=standardclasses,
    headings=big,
    pointednumbers
]{scrartcl}



%-------------------------------------------------------------------------------
% Dokument Sprache
\usepackage[ngerman]{babel}

%-------------------------------------------------------------------------------
% Dokument Sprache
\usepackage[utf8]{inputenc}
\usepackage[T1]{fontenc}

%-------------------------------------------------------------------------------
% Individuelle Kopf- und Fusszeilen

\usepackage{scrlayer-scrpage}

% Bisherige Einstellungen für Kopf- und Fußzeilen löschen:
\clearpairofpagestyles

% Zentriert auf linken Seiten die aktuelle Kapitelüberschrift,
% auf rechten Seiten die Überschrift des aktuellen Abschnitts ausgeben:
\chead{\headmark}

% Zentriert die Seitenzahl ausgeben (auch beim Seitenstil "scrplain"):
\cfoot*{\pagemark}

\pagestyle{scrheadings}

%-------------------------------------------------------------------------------
% Text mit Unterstrich
\usepackage{ulem}

%-------------------------------------------------------------------------------
% Seitenränder und Anmerkungen 
\usepackage[marginparwidth=3cm,
            left=5cm,
            textwidth=12cm]{geometry}
%\usepackage{marginnote}
\reversemarginpar

%\usepackage{showframe}
%\usepackage{layout}

%\usepackage{fourier} 
%\usepackage{mparhack}

%-------------------------------------------------------------------------------
% AMS
\usepackage{amsmath}
\usepackage{amssymb}

%\usepackage{multirow}

%-------------------------------------------------------------------------------
% 
\newcommand{\anmerkung}[2]{%
    \marginline{%
        \vspace{#1} \rule{\marginparwidth}{0.4pt} #2
    }
}

\newcommand{\mytitle}[1]{%
    {\LARGE \bfseries #1}
}

\makeatletter
\newcommand*{\rom}[1]{\expandafter\@slowromancap\romannumeral #1@}
\makeatother

%-------------------------------------------------------------------------------
% Blinktext
%\usepackage{blindtext}

\usepackage{csquotes}

\usepackage{epigraph}
%\epigraphfontsize{\small\itshape}
\setlength\epigraphwidth{8cm}
%\setlength\epigraphwidth{.8\textwidth}
\setlength\epigraphrule{1pt}

\renewcommand*{\dictumwidth}{\textwidth}
\renewcommand*{\raggeddictum}{\raggedright}
\renewcommand*{\raggeddictumtext}{\raggedright}
\renewcommand*{\dictumrule}{}
\renewcommand*{\dictumauthorformat}[1]{--- \textup{#1}}

%-------------------------------------------------------------------------------
% Listings
\usepackage{listings}

\lstset{
	language=Java,
	breaklines=true,
	numbers=left,
	numberstyle=\tiny,
	numbersep=5pt,
	captionpos=b,
	basicstyle=\footnotesize\ttfamily,
	stringstyle=\color{magenta},
	identifierstyle=\color{black},
	keywordstyle=\color{blue}, 
	commentstyle=\color{DarkGreen}
}

\newcommand{\mylisting}[2][]{%
	\lstinputlisting[caption={\texttt{\detokenize{#2}}},#1]{#2}%
}

%-------------------------------------------------------------------------------

\usepackage{tikz}

\usetikzlibrary{shapes,snakes,calc,arrows}

%-------------------------------------------------------------------------------
% Dokument
\begin{document}
    %\layout
    %\blindtext
    
    \mytitle{Komplexitästheorie}
    
    \section{Polyzeit-Reduktion}
    
    \begin{itemize}
    	\setlength{\itemindent}{-12pt}
    	\setlength{\itemsep}{-10pt}
    	%\itemsep-10pt
    	\renewcommand{\labelitemi}{$-$}
    	\item Beispiele zu Reduktionen
    	\item Definition davon verstehen
    	\item Intuitiv Verstehen was eine Reduktion ist
    	\item Viele schwarze Punkt
    	\item Jeder Punkt ist ein Entscheidungsproblem resp. eine formale Sprache
    	\item Eine Reduktion ist eine Verbindung zwischen zwei Problemen
    \end{itemize}

    \begin{tikzpicture}[every node/.style={fill, circle, inner sep = 1pt}]
	    \pgfmathsetseed{1}
    	\foreach \x in {1,...,8} {
    		\foreach \y in {1,...,8} {
			    %\node[label=above:$\x \y$] (A\x\y) at (\x + rand, rand) {};
			    \node (A\x\y) at (\x + rand, rand) {};
			}
		}
	
		\iffalse
			\draw[red,thick,dashed] (A11) -- (A12);
			\draw[blue,thick,dashed] (A11) -- (A13);
			\draw[green,thick,dashed] (A11) -- (A14);
			\draw (A11) -- (A15);
			\draw (A11) -- (A16);
			\draw (A11) -- (A17);
			\draw (A11) -- (A18);
			
			\draw[red,thick,dashed] (A21) -- (A22);
			\draw[blue,thick,dashed] (A21) -- (A23);
			\draw[green,thick,dashed] (A21) -- (A24);
			\draw[red,thick,dashed] (A21) -- (A25);
			\draw[blue,thick,dashed] (A21) -- (A26);
			\draw[green,thick,dashed] (A21) -- (A27);
			\draw (A21) -- (A28);
		\fi
		
		\draw (A14) -- (A28);
    \end{tikzpicture}
    
    \iffalse
    \begin{tikzpicture}
		\node (M) at (0,0) {$M$};
		\node (A)  [text width=3cm]       at (3,0.5)    {akzeptiert Wort};
		\node (A2) [text width=3cm, blue] at (6.5,0.5)  {accept};
		\node (B)  [text width=3cm]       at (3.0,-0.5) {erkennt Sprache};
		\node (B2) [text width=3cm, blue] at (6.5,-0.5) {recognize};
		\draw[->,thick] (M) to [in = 180, out = +40] (A);
		\draw[->,thick] (M) to [in = 180, out = -40] (B);
	\end{tikzpicture}
	\fi

	Denn es kommt häufig vor, dass man irgendwie erkennt: 2 Probleme sind ziemlich ähnlich.
	
	Oft sieht man \underline{ein} Problem ist nur ein \underline{anderes} Problem mit ein
	bisschen anderer Verkleidung.
	
	Oder man stellt fest: wenn ich das \underline{eine} Problem lösen kann, dann kann ich mit dem
	Entscheidungsverfahren für dieses auch das \underline{andere} ganz einfach lösen.
	
	Wenn man so etwas gefunden hat, dann ist es eine Reduktion, von dem \underline{einen} auf das
	\underline{andere} Problem.
	
	\begin{tikzpicture}[every node/.style={draw, circle, inner sep = 5pt}]
		\node (A) at (0, 0) [circle] {$A$};
		\node (B) at (3, 0) [circle] {$B$};
		\draw[line width=1pt, arrows=-angle 60] (A) to (B);
	\end{tikzpicture}
	
	Wenn wir Problem $B$ lösen können und eine Reduktion von $A$ auf $B$, dann kann man
	$A$ auch lösen. Es ist wie eine Übersetzung von einem in ein anderes Problem.
	
	Ihr kennt das sicher, wenn ihr code schreibt, denkt ihr, moment, so etwas habe ich
	gleich schon mal geschrieben.
	
	Wenn ihr feststellt, Code, den man wiederverwenden kann. Denn er löst ein Problem, dass,
	wenn man dieses Problem lösen kann, kann man mit diesem Problem andere Probleme einfacher
	lösen. Wenn $A$ und $B$ ganz ähnliche Probleme sind, und wir möchten den Code dafür nicht
	zwei Mal schreiben sondern wir schreiben einmal wie man Problem B löst und dann schreiben
	wir, wie übersetzt man Problem B in Problem A. Und dann können wir das Problem A ganz einfach
	lösen indem wir das Problem B übersetzen und das Problem A lösen.
	
	Problem $A$:
	\mylisting[frame=single]{solveL1.java}
	
	Problem $B$:
	\mylisting[frame=single]{solveL2.java}
	
    Diese Reduktion übersetzt jedes Wort $\omega$ in der Sprache $L_1$ enthalten ist also wo die
    Antwort \textbf{<<ja>>} lautet. Und auch jedes Wort $\omega$ wo die Antwort \textbf{<<nein>>}
    lauten würde weil $L_1$ übersetzt es in ein Wort $\omega_2$ so dass die Antwort hierauf
    \textbf{<<nein>>} lautet. So dafür machen wir jetzt Mal ein paar Beispiele.
    
    \subsection{Beispiel mit formalen Sprachen}
    
    Machen wir Mal ein Beispiel mit formalen Sprachen.
    
    \begin{flalign*}
	    L_1  & = \left\{ \omega \mid \textrm{ Länge } \vert w\vert_b \textrm{ lässt bei Division durch $3$ den Rest $1$ } \right\} & \\
	    L_2  & = \left\{ \omega \mid \textrm{ Länge } \vert w\vert_b \textrm{ ist durch $3$ teilbar } \right\} &
    \end{flalign*}
    
    Also sagen wir Mal: für diese Sprache $L_2$
    hier, das ist nämlich alle Worte bei denen die Anzahl der b's durch drei Teilbar ist.
    Sagen wir Mal dafür haben wir schon einen Algorithmus gefunden. Wenn man so einen 
    Algorithmus, dem geben wir ein Wort, und der gibt uns zurück einen \texttt{boolean}
    und zwar gibt der \texttt{true} wenn die Anzahl der b's in diesem Wort durch drei Teilbar
    ist und \texttt{false} falls es nicht durch drei Teilbar ist.
    
    Jetzt wollen wir das Problem $L_1$ lösen und zwar sind das all die Worte bei denen
    die Anzahl der b's bei der Division durch drei den Rest 1 lässt und das wirkt irgendwie
    ganz ähnlich wie Problem $L_2$ und tatsächlich können wir $L_1$ auf $L_2$ reduzieren.
    Wir können wann immer wir ein Wort bekommen und wir wollen wissen, ist das in $L_1$,
    können wir es auf eine ganz einfache Art übersetzen in eine Eingabe für $L_2$ und dann
    führen wir einfach den Algorithmus auf diesem neue Wort. Führen wir einfach den Algorithmus
    aus. Der checkt einfach nur, ist die Anzahl der b's in diesem Wort durch drei Teilbar.
    \begin{center}
    	\begin{tikzpicture}[every node/.style={inner sep = 5pt}]
    	\node (L1) at (0, 0.5) {$L_1$:};
    	\node (A) at (0, 0) {$\omega$};
    	\node (L2) at (4, 0.5) {$L_2$:};
    	\node (B) at (4, 0) {$\omega bb$};
    	\node (C) at (1.8, 0.5) {Reduktion};
    	\draw[->,decorate,decoration={snake,amplitude=0.8mm,segment length=2mm,post length=1mm}] (A) to (B);
    	\end{tikzpicture}
    \end{center}
    
    Was müssen wir da machen um das zu übersetzen? Wir können folgendes tun: man würde irgend
    ein Wort $\omega$ bekommen, dann
    
    sd
    
\end{document}