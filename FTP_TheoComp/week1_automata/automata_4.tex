

%-------------------------------------------------------------------------------
% Dokumenten Klasse
\documentclass[
	final,
	a4paper,
	oneside,
	parskip=full,
	headings=standardclasses,
	headings=big,
	pointednumbers
]{scrartcl}

%-------------------------------------------------------------------------------
% Packete nutzen
\usepackage{ngerman,palatino,setspace}
\usepackage[T1]{fontenc}
\usepackage[utf8]{inputenc}
\usepackage[left=20mm,right=20mm,top=25mm,bottom=25mm]{geometry}
\usepackage[svgnames]{xcolor}
\usepackage{amsmath}
\usepackage{mathtools}
\usepackage{tikz}

\usetikzlibrary{automata, positioning, arrows, shapes, calc}

%{
%\tikzset{
%    ->, % makes the edges directed
%    >=stealth, % makes the arrow heads bold
%    node distance=2cm, % specifies the minimum distance between two nodes. Change if necessary.
%    every state/.style={thick, fill=gray!10}, % sets the properties for each ’state’ node
%    every edge/.append style={line width=0.25mm}, % sets the properties for each ’state’ node
%    initial text=$ $, % sets the text that appears on the start arrow
%}
%}

\tikzset{
    node distance=2cm, % Minimum distance between two nodes. Change if necessary.
    every state/.style={ % Sets the properties for each state
        semithick,
        fill=gray!10
    },
    initial text={}, % No label on start arrow
    double distance=2pt, % Adjust appearance of accept states
    every edge/.style={ % Sets the properties for each transition
        draw,
        ->,>=stealth, % Makes edges directed with bold arrowheads
        auto,
        semithick
    }
}

%-------------------------------------------------------------------------------
% New Font package
%\usepackage{bm}
%\usepackage[sc]{mathpazo}

\def\boldeps{\boldsymbol{\varepsilon}}

%-------------------------------------------------------------------------------
\usepackage{multirow}

%-------------------------------------------------------------------------------
% uline
\usepackage{ulem}

%-------------------------------------------------------------------------------
% Anderer Font
\usepackage{mathrsfs}
\usepackage[mathcal]{euscript}

%-------------------------------------------------------------------------------
% blank symbol \fgeupbracket
\usepackage{fge}


%-------------------------------------------------------------------------------
% besseres tabular
\usepackage{array}

%-------------------------------------------------------------------------------
% Square brackets
\usepackage{stmaryrd}

\newcommand{\blank}{\fgeupbracket}
\newcommand{\red}[1]{{\color{red}#1}}
\newcommand{\blue}[1]{{\color{blue}#1}}
\newcommand{\green}[1]{{\color{Green}#1}}

\newcommand{\newState}[4]{\node[state,#3](#1)[#4]{#2};}
\newcommand{\newTransition}[4]{\path[->] (#1) edge [#4] node {#3} (#2);} 

%-------------------------------------------------------------------------------
% mycircle
\newcommand{\mc}[1]{%
    \tikz[baseline=(char.base)]{
        \node[circle, draw=red, inner sep=1pt](char){#1};
        \draw[->,>=stealth, draw=red] (char) -- (0,-0.5);
    }
}



%\newcommand{\mc}[1]{%
%    \tikz[baseline=(q.base)]
%    \node[circle, draw=red, inner sep=1pt](q){#1};
%    \draw (q) -- (1,0);}

%-------------------------------------------------------------------------------
% Dokument
\begin{document}

    
    %--- Page 1 --------------------------------------------------------------------
    \newpage

    {
        \renewcommand{\arraystretch}{1}
        \newcolumntype{x}{>{\centering\arraybackslash}p{0.5cm}}
        \begin{tabular}{lxxxxxxx}
                & \mc{$q_1$} &            &            &            &            &          &          \\
            1.  & $a$        & $a$        & $a$        & $b$        & $a$        & $b$      & $\blank$ \\
            %--------------------------------------------------------------------------------------------
                &            & \mc{$q_2$} &            &            &            &          &          \\
            2.  & $\blank$   & $a$        & $a$        & $b$        & $a$        & $b$      & $\blank$ \\
            %--------------------------------------------------------------------------------------------
                &            &            & \mc{$q_3$} &            &            &          &          \\
            3.  & $\blank$   & $X$        & $a$        & $b$        & $a$        & $b$      & $\blank$ \\
            %--------------------------------------------------------------------------------------------
                &            &            &            & \mc{$q_3$} &            &          &          \\
            4.  & $\blank$   & $X$        & $a$        & $b$        & $a$        & $b$      & $\blank$ \\
            %--------------------------------------------------------------------------------------------
                &            &            & \mc{$q_6$} &            &            &          &          \\
            5.  & $\blank$   & $X$        & $a$        & $X$        & $a$        & $b$      & $\blank$ \\
            %--------------------------------------------------------------------------------------------
                &            & \mc{$q_6$} &            &            &            &          &          \\
            6.  & $\blank$   & $X$        & $a$        & $X$        & $a$        & $b$      & $\blank$ \\
            %--------------------------------------------------------------------------------------------
                & \mc{$q_6$} &            &            &            &            &          &          \\
            7.  & $\blank$   & $X$        & $a$        & $X$        & $a$        & $b$      & $\blank$ \\
            %--------------------------------------------------------------------------------------------
                &            & \mc{$q_1$} &            &            &            &          &          \\
            8.  & $\blank$   & $X$        & $a$        & $X$        & $a$        & $b$      & $\blank$ \\
            %--------------------------------------------------------------------------------------------
                &            &            & \mc{$q_1$} &            &            &          &          \\
            9.  & $\blank$   & $X$        & $a$        & $X$        & $a$        & $b$      & $\blank$ \\
            %--------------------------------------------------------------------------------------------
                &            &            &            & \mc{$q_2$} &            &          &          \\
            10. & $\blank$   & $X$        & $\blank$   & $X$        & $a$        & $b$      & $\blank$ \\
            %--------------------------------------------------------------------------------------------
                &            &            &            &            & \mc{$q_3$} &          &          \\
            11. & $\blank$   & $X$        & $X$        & $X$        & $\blank$   & $\blank$ & $\blank$ \\
            %--------------------------------------------------------------------------------------------
        \end{tabular}
    }
\end{document}
