\documentclass{article}
\usepackage{blindtext}
\usepackage{geometry}
\usepackage{booktabs,
            makecell, % for second example
            tabularx}
\renewcommand\theadfont{\itshape\normalsize}% for two lines column headers
\renewcommand\theadgape{}
\usepackage{siunitx}
\usepackage[table]{xcolor}
\usepackage{ragged2e}

\renewcommand\tabularxcolumn[1]{>{\Centering}m{#1}}

\newcommand\TEXT{%
I want to draw the following table in Latex. In the first column the text should be left aligned. The text in all other cells should be centered.}% only for demo

\begin{document}

    \begin{table}[h]
        \centering
        \sisetup{table-align-text-post=false,
                 table-space-text-post={\,\%},
                 group-four-digits}
        \caption[Material for SDS-PAGE]{Buffer compositions for SDS-PAGE.}
        \label{tab:8899}
        \begin{tabularx}{0.8\textwidth}{>{\raggedright}X
                S[table-format=3.2]
                S[table-format=3.2]
                S[table-format=3.5]
                l
            }
            \toprule &
            {\thead{Resolving\\ gel}} & {\thead{Stacking\\ gel}} & 
            {\thead{Laemmli\\ sample buffer}} & \thead{Running\\ buffer} \\

            \midrule
            Acrylamide                  & 10 \%     & 5 \%      &  {--}     &  {--} \\
            Tris-HCl (1.5 M, pH 8.8)    & 25 \%     &  {--}     &  {--}     &  {--} \\
            \addlinespace
            Tris-HCl (1.0 M, pH 6.8)    &  {--}     & 12.5 \%   & 6.25 \%   &  {--} \\
            SDS                         & 0.1 \%    & 0.1 \%    &  {--}     &  {--} \\
            \addlinespace
            Ammonium persulfate         & 0.1 \%    & 0.1 \%    &  {--}     &  {--} \\
            TEMED                       & 0.04 \%   & 0.1 \%    &  {--}     &  {--} \\
            \addlinespace
            2-Mercaptoethanol           &  {--}     &  {--}     & 0.1 \%    &  {--} \\
            Bromophenol blue            &  {--}     &  {--}     & 0.0005 \% &  {--} \\
            \addlinespace
            Glycerol                    &  {--}     &  {--}     & 10 \%     &  {--} \\
            TRIS                        &  {--}     &  {--}     &  {--}     & 0.25 M\\
            \addlinespace
            Glycine                     &  {--}     &  {--}     &  {--}     & 1.92 M\\
            pH                          &  {--}     &  {--}     &  {--}     & 8.3   \\
            \addlinespace
            H\textsubscript{2}O         & {Solvent} & {Solvent} & {Solvent} & Solvent\\
            \bottomrule
        \end{tabularx}
    \end{table}

    \begin{table}[h]
        \centering
        \begin{tabular}{l c S[table-format=10.9] S[retain-zero-exponent=true]}
            \toprule
            \multicolumn{4}{c}{SI Prefixes} \\
            \addlinespace %\midrule
            Prefix & Symbol & {Multiplication Factor} & {\dots\ in Scientific Notation} \\
            \midrule
            giga  & \si{\giga} & 1000000000 & e9 \\
            mega  & \si{\mega} & 1000000    & e6 \\ 
            kilo  & \si{\kilo} & 1000       & e3 \\
            deca  & \si{\deka} & 10         & e1 \\
            \rowcolor{gray!20}  -- & -- & 1 & e0 \\
            deci  & \si{\deci} & 0.1        & e-1 \\
            centi & \si{\centi}& 0.01       & e-2 \\
            milli & \si{\milli}& 0.001      & e-3 \\
            micro & \si{\micro}& 0.000001   & e-6 \\
            nano  & \si{\nano} & 0.000000001& e-9 \\
            \bottomrule
        \end{tabular}
    \end{table}


    \newpage
    
    \begin{table}[h]
        \centering
        \begin{tabularx}{\textwidth}{|m{4cm}*4{|X}|}\hline
                       &  A  &   B &   C & D         \\\hline
            \TEXT\TEXT & 123 & 123 & 123 & \TEXT\\\hline
        \end{tabularx}
    \end{table}

    \begin{table}[h]
        \begin{tabular}{|lp{2cm}m{2cm}b{2cm}r|}
          \toprule
          foo & long text in the first column
          & long text in the second column
          & even longer text in the third and second-last column
          & bar \\
          \bottomrule
        \end{tabular}
    \end{table}

    \begin{table}[h]
        \renewcommand\tabularxcolumn[1]{m{#1}}
        \centering
        \begin{tabularx}{\textwidth}{|>{\RaggedRight}p{4cm}*{5}{|>{\Centering}X}|}
            \hline
                       &  A & B & C & D\\
            \hline
            \blindtext & 123 & 123 & 123 & 123\\
            \hline
        \end{tabularx}
    \end{table}
\end{document}