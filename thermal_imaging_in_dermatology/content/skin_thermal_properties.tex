%!TEX root = ../main.tex

\section*{Skin Thermal Properties}

In the following paragraphs, we will briefly review the skin structure, its
thermoregulation mechanisms, and the different models proposed in the literature
to simulate its thermal behavior.

\subsection*{Skin Structure}

The skin is the biggest organ of the body, with a mean surface of 2 square meters
and a weight of 4-10 kg in adults (ie, around 8\% of the body mass). Skin has four
main functions: protection, sensation-connection, thermoregulation, and metabolism.
The skin preserves the hydration of the body thanks to the \textit{stratum cornea} that
limits the loss of water. Moreover, it protects the body from mechanical injuries,
chemical injuries, temperatures variations, ultraviolet (UV) radiations, and
microorganisms [15]. Its thickness is highly variable, conferring an adapted
flexibility or mechanical protection according to the needs of the different parts
of the body. The skin is stratified in three main layers, ie, the epidermis, the
dermis, and the hypodermis (Fig. 31.1A).

\subsubsection*{Epidermis}

The epidermis is the outer cellular layer of the skin. The thickness of epidermis
varies with the location; for example, the epidermis of the eyelid measures around
$30-40 \mu m$, compared with $140 \mu m$ for the buttocks, or more than $600 \mu m$
for the palms or soles [16]. \textit{Keratinocytes} are the most important contingent of
epidermal cells. As these cells differentiate, they move upward to the surface and
their shape and their content change and form the successive layers of the
epidermis. The last layer is the \textit{stratum corneum}, or horny layer, made of dead
\textit{keratinocytes} filled with keratin, a sulfur-rich protein, and coated with lipids.
Its very cohesive structure plays a major role in limiting the water loss and in
the global protective barrier function of the skin. \textit{Melanocytes} are located
in the basal layer of the epidermis and connected to several layers of \textit{keratinocytes},
thanks to their dendritic morphology. These cells secrete a pigment, the \textit{melanin},
and transfer the melanin granules to \textit{keratinocytes} by their dendritic
processes. \textit{Langerhans cells} are professional antigen-presenting cells. They
play a pivotal role in the immune defense of the body. \textit{Merkel cells} (initially
called \textit{touch cells} by Merkel) are connected to sensory nerve endings and
involved in the sensitive discrimination.

\subsubsection*{Dermis}

The dermis is a very dense layer that joins the epidermis at a thin basement membrane
zone. Its thickness can vary from $0.4 mm$ (eyelids, prepuce) to $1 cm$ on the back.
Dermis is a fibrous connective tissue that contains the blood supply for the skin.
Through the very developed vascularization (around $1 m$ of capillaries per square
centimeter of skin) the dermis brings nutrients to the epidermis, because it has no
direct blood supply. The \textit{adnexae}, ie, \textit{sebaceous glands} that secrete the
\textit{sebum}, \textit{apocrine}, and \textit{eccrine glands} that secrete the sweat,
and the \textit{hair follicles}, are located in this skin layer (see Fig. 31.1A). The
basis of the dermis is a supporting matrix made of \textit{mucopolysaccharides}, ie,
\textit{macromolecules} that retain the water like sponges. Within this matrix, two
kinds of fiber confer the strength properties of the skin: a great tensile strength
from the \textit{collagen fibers} and the elasticity from the \textit{elastic fibers}.
\textit{Dermal cells} are principally \textit{fibroblasts} that produce the
\textit{collagen} and the \textit{elastic fibers}. Other cells like \textit{histiocytes}
and \textit{mast cells} are involved in the immune system.

\subsubsection*{Hypodermis}

Hypodermis is also called \textit{subcutaneous fat}, though it is part of the skin.
It is its innermost and thickest layer, attached to the dermis by \textit{collagen}
and \textit{elastic fibers}. It is constituted of \textit{adipocytes} organized in
lobules and connective tissue containing vessels and nerves (see 31.1.A). This deep
layer of the skin can be extremely thin in some parts of the body (< $1 mm$ on eyelids)
or extremely thick in others (several centimeters in abdomen or buttocks), with huge
variations from person to person according to the body mass index.

\subsection*{Skin Thermal Modeling}

As discussed later in this capture thermal imaging achieves its full potential only
in combination with a proper thermal modeling of the sample under investigation. The
building of such thermal models is a requirement for the extraction of quantitative
parameters from the experimental data. Heat transfer phenomena taking place in the
different skin layers are a mix of heat conduction processes coupled with
physiological mechanisms that include blood perfusion, metabolic heat generation,
and sweating. The skin is an active medium that is regulating the bodily temperature.
Absorption and emission of visible, UV, or IR radiation depends on the thickness and
the pigmentation of the skin. Hypodermis and hairs isolate the body from the cold.
Thermal regulation can also be achieved by the variations of the diameter of the skin
vessels. Indeed, the skin is highly vascularized, containing about 10\% of the vessels
of the all body. Thermoreceptors located in the skin detect levels and variations
of temperature. If skin temperature drops, neuronal signals are sent to trigger
vasoconstriction of dermal and hypodermal arterioles, limiting thermal loss by
reducing the blood flow exposed to peripheral low temperatures. If the internal
temperature rises, vasodilation allows heat transfers to external environment by
the means of radiation, conduction, and convection. When ambient temperature is high,
evaporation of sweat produced by eccrine and apocrine glands induces a lowering of the
body temperature. In case of extreme temperature the metabolism will eventually slow
down. Skin tissue is a complex active, nonhomogeneous and anisotropic medium; as a
result, the building of a realistic thermal skin model remains challenging.

Numerous heat-transfer skin models have been proposed in the literature (see, for
example, Ref. [17] for a review on the subject). They can be classified into four
categories: continuum models, vascular models, hybrid models, and models based on
porous media theory. Continuum models are widely used because of their simplicity
and because they can be either solved analytically or using finite elements or
finite difference solvers. They are based on the Penne's bioheat equation describing
the influence of blood perfusion on the skin temperature distribution in terms of
volumetrically distributed heat sources [18].

\begin{equation} 
	\rho C \frac{\partial T}{\partial t} + \omega \rho_b C_b \left ( T - T_b \right ) - Q = k \nabla^2 T
\end{equation}

where $C, k, \rho, \omega$ and $Q$ are respectively the specific heat, the heat
conductivity, the density, the blood perfusion rate, and the metabolic heat
generation. $C_b, \rho_b$ and $T_b$ denote the blood specific heat, the blood
density, and the blood temperature usually set to $C$. $T$ represents the local
tissue temperature, $t$ denotes the time variable, and $\nabla^2$ is the Laplace
operator. Eq. 31.1 states that the rate of change of thermal energy contained in
a unit volume is equal to the sum of the rates at which the thermal energy enters
or leaves the volume by conduction, perfusion, and metabolic heat generation. The
term $\omega \rho_b C_b \left ( T - T_b \right )$ describes the volumetrically
distributed heat sources or heat sinks, depending on whether the local tissue
temperature is above or under blood temperature.

For a simple one-layer model where the skin is considered as a semiinfinite,
homogenous medium, closed-form analytical solutions of Eq. (31.1) can be
obtained for different boundary conditions [19, 20]. Nonetheless, to achieve a
more realistic description, skin tissue should be considered as composed of
different layers, each layer exhibiting specific thermophysical properties.
Beside, vasoconstriction and vasodilation mechanisms should be added to the
model by making tissue perfusion a function of the local temperature. The
complexity of such a system requires the use of numerical methods.

Fig 31.1B represents the idealized model usually adopted to describe the skin
structure [21]. Three different layers are considered: epidermis, dermis, and
fat tissue. The size of each layer has to be adapted depending on the bodily
location. Some models split the dermis into reticular and papillary dermis or
take into account an additional muscle layer [22,23]. Table 31.1 summarizes
the thermophysical properties of each of these layers taken from the
literature.

Eq. (31.1) is numerically solved imposing appropriate boundary conditions at
the domain boaders and continuity conditions for the temperature and heat flux
at each interface between the different tissue layers. The bottom surface of
the skin is usually fixed at the core temperature $T_C = T_b = 37 grad C$
(see Fig. 31.1B), whereas the boundary condition for the heat transfer
occurring at the skin surface is generally composed of three parts, ie,
convection, radiation and evaporation [21]:
\begin{equation} 
	-k \nabla^2 T_s = h \left( T_s - T_{amb} \right) + \epsilon \sigma \left( T_{s}^4 - T_{amb}^4 \right) + Q_e
\end{equation}

where $T_s$ is the skin surface temperature, $T_{amb}$ the ambient room
temperature, $h$ the convective heat transfer coefficient (natural or forced),
$\epsilon$ the skin emissivity, $\sigma$ the Stefan-Boltzmann constant, and
$Q_e$ the evaporative heat losses caused by sweating.

In skin thermal imaging, an IR camera measures the radiative component of
Eq. (31.2). Even neglecting the atmospheric absorption happening between the
skin and the camera detector, the ambient temperature, as well as the skin
surface emissivity, should be known to calculate the absolute skin surface
temperature $T_s$. This well-known difficulty in thermal imaging is discussed
more in details in the ``Thermal Radiation Characteristics'' section.

It follows from the skin thermal model that only pathological states affecting
one or several of the thermophysical parameters presented in Table 31.1, or
affecting the thickness of the different skin layers, can induce a potentially
measureable variation of the skin surface temperature $T_s$. More restricting is
the nonspecificity of thermal imaging. From static skin thermograms it is difficult
to differentiate the origin of thermal signals. A way to overcome this limitation
is to perform active or dynamic thermography measurements where the skin surface
is monitored in a transient state. For example, the convection term of Eq. (31.2)
can be periodically modulated by varying the ambient temperature or the heat
transfer coefficient $h$. We suggested that monitoring the skin surface temperature
response obtained for different modulation frequencies allows differentiating
thermal signals originating from perfusion variations [24].

By specifically designing an active thermography experiment, together with the
development of a heat transfer model, thermal imaging has the ability to retreive
quantitative information about distinct skin thermophysical properties. Different
active thermography methods are presented in the ``Measurement Procedures'' section.

