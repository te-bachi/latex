%!TEX root = ../main.tex

\section*{Skin Thermal Properties}

In the following paragraphs, we will briefly review the skin structure, its
thermoregulation mechanisms, and the different models proposed in the literature
to simulate its thermal behavior.

\subsection*{Skin Structure}

The skin is the biggest organ of the body, with a mean surface of 2 square meters
and a weight of 4-10 kg in adults (ie, around 8\% of the body mass). Skin has four
main functions: protection, sensation-connection, thermoregulation, and metabolism.
The skin preserves the hydration of the body thanks to the \textit{stratum cornea} that
limits the loss of water. Moreover, it protects the body from mechanical injuries,
chemical injuries, temperatures variations, ultraviolet (UV) radiations, and
microorganisms [15]. Its thickness is highly variable, conferring an adapted
flexibility or mechanical protection according to the needs of the different parts
of the body. The skin is stratified in three main layers, ie, the epidermis, the
dermis, and the hypodermis (Fig. 31.1A).

\subsubsection*{Epidermis}

The epidermis is the outer cellular layer of the skin. The thickness of epidermis
varies with the location; for example, the epidermis of the eyelid measures around
$30-40 \mu m$, compared with $140 \mu m$ for the buttocks, or more than $600 \mu m$
for the palms or soles [16]. \textit{Keratinocytes} are the most important contingent of
epidermal cells. As these cells differentiate, they move upward to the surface and
their shape and their content change and form the successive layers of the
epidermis. The last layer is the \textit{stratum corneum}, or horny layer, made of dead
\textit{keratinocytes} filled with keratin, a sulfur-rich protein, and coated with lipids.
Its very cohesive structure plays a major role in limiting the water loss and in
the global protective barrier function of the skin. \textit{Melanocytes} are located
in the basal layer of the epidermis and connected to several layers of \textit{keratinocytes},
thanks to their dendritic morphology. These cells secrete a pigment, the \textit{melanin},
and transfer the melanin granules to \textit{keratinocytes} by their dendritic
processes. \textit{Langerhans cells} are professional antigen-presenting cells. They
play a pivotal role in the immune defense of the body. \textit{Merkel cells} (initially
called \textit{touch cells} by Merkel) are connected to sensory nerve endings and
involved in the sensitive discrimination.
