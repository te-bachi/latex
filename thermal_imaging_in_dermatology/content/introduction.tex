%!TEX root = ../main.tex

\section*{Introduction}

Thermal imaging, or thermography, consists of measuring and imaging the thermal radiation
emitted by every object above the absolute zero temperature [1]. Because this radiation is
temperature-dependent, the infrared (IR) images recorded can be converted into temperature
maps, or thermograms, allowing retrieving valuable information about the object under
investigation. Thermal imaging has been known since the middle of the 20th century; recent
technological achievements concerning IR imaging devices, together with the development of
new procedures based on transient thermal emission measurements have revolutionized the
field. Nowadays thermography is a method with advantages that are undisputed in engineering.

It is routinely used for the nondestructive testing of materials [2], to investigate
electronic components [3], or in the photovoltaic industry to detect defects in solar
cells [4].

Despite an early interest for potential medical applications [5-7], thermal imaging is
rarely used in the clinic. The main reason is probably the initial disappointing results
obtained solely with static measurement procedures, where the sample is investigated in
its steady state and using unhandy and performance-limited first-generation IR cameras.
Fortunately those early studies have been put into perspective and medical thermal
imaging is experiencing a renaissance since the late 1990s [8,9].

Among the numerous potential medical applications of thermal imaging (such as in
neurology [10], oncology [11], ophthalmology [12], surgery [13], or dentistry [14]),
dermatology is one of the most promising application fields. Despite its low specificity,
static thermal imaging is a powerful tool to detect and characterize problems affecting
the skin physiology. Indeed, abnormalities such as malignancies, inflammation, and
infection usually cause localized increases in temperature that can be identified as hot
spots or as asymmetrical patterns in a skin thermogram. In combination with skin thermal
models, new active procedures drastically extend the capabilities of thermal imaging by
allowing the retrieval of quantitative physiological information.

The goal of this chapter is to demonstrate the potential of thermal imaging for
dermatological applications and to review the main investigations accomplished so far.
Skin thermal properties will be briefly reviewed together with the main heat transfer
models used to extract physiological parameters from experimental data. The basics of
thermal radiation and thermal-imaging device technology will be presented with the
different experimental procedures that have been reported in the literature. Rather
than giving an exhaustive description of thermography, we aim to familiarize the
reader with key concepts that will allow designing a proper experimental setup with an
optimal IR camera depending on the specific application. As an illustration, two examples
of the utilization of thermal imaging in skin cancer detection and burn depth evaluation
will be presented. Other promising applications will be briefly outlined in the last
paragraph.
