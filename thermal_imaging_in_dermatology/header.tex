
%-------------------------------------------------------------------------------
% Dokumenten Klasse
\documentclass[
	final,
	a4paper,
	twoside,
	parskip=full,
	headings=standardclasses,
	headings=big
]{scrartcl}

%-------------------------------------------------------------------------------
% Packete nutzen
\usepackage{ngerman,palatino,setspace}
\usepackage[T1]{fontenc}
\usepackage[utf8]{inputenc}
\usepackage[left=35mm,right=35mm,top=25mm,bottom=25mm]{geometry}
\usepackage{graphicx}
\usepackage{scrpage2}
\usepackage{listings}
\usepackage[usenames,dvipsnames,svgnames,table]{xcolor}
\usepackage[hidelinks]{hyperref}
\usepackage{amsmath}
\usepackage{caption}

%-------------------------------------------------------------------------------
% Kopf- und Fusszeile

\pagestyle{scrheadings}
\clearscrheadfoot
\chead{Diseases \& Projects > About Leishmaniasis}
\cfoot[Seite \thepage]{Seite \thepage}

%-------------------------------------------------------------------------------
%
\title{Diseases \& Projects > About Leishmaniasis}
\author{Andreas Bachmann \\ \href{mailto:bachman0@students.zhaw.ch}{bachman0@students.zhaw.ch}}
\date{\today}

%-------------------------------------------------------------------------------
% Dokumenten Einstellungen

% \section Abstände
\RedeclareSectionCommand[
beforeskip=-1\baselineskip,
afterskip=0.001\baselineskip
]{section}

% \subsection Abstände
\RedeclareSectionCommand[
beforeskip=0\baselineskip,
afterskip=0.001\baselineskip
]{subsection}

% Formeln
\DeclareCaptionType{mycapequ}[aa][bb]
\captionsetup[mycapequ]{labelformat=empty}


%-------------------------------------------------------------------------------
% Anführungszeichen/Gänsefüsschen mit Leerzeichen am Ende

% Deutsches doppeltes unteres/oberes Anführungszeichen
\newcommand{\gqq}[1]{
	\glqq{}#1\grqq{}
}

% Deutsches einfaches unteres/oberes Anführungszeichen
\newcommand{\gq}[1]{
	\glq{}#1\grq{}
}

% Französisches doppeltes Anführungszeichen
\newcommand{\fqq}[1]{
	\flqq{}#1\frqq{}
}

% Französisches einfaches Anführungszeichen
\newcommand{\fq}[1]{
	\flq{}#1\frq{}
}


%-------------------------------------------------------------------------------
% Listings
\lstset{
	language=Matlab,
	breaklines=true,
	numbers=left,
	numberstyle=\tiny,
	numbersep=5pt,
	captionpos=b,
	basicstyle=\footnotesize\ttfamily,
	stringstyle=\color{magenta},
	identifierstyle=\color{black},
	keywordstyle=\color{blue}, 
	commentstyle=\color{DarkGreen}
}

\newcommand{\mylisting}[2][]{%
	\lstinputlisting[caption={\texttt{\detokenize{#2}}},#1]{#2}%
}
