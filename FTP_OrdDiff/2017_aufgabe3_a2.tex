

%-------------------------------------------------------------------------------
% Dokumenten Klasse
\documentclass[
	final,
	oneside,
	parskip=full,
	headings=standardclasses,
	headings=big,
	pointednumbers
]{scrartcl}

%-------------------------------------------------------------------------------
% Packete nutzen
\usepackage{ngerman,palatino,setspace}
\usepackage[T1]{fontenc}
\usepackage[utf8]{inputenc}
\usepackage[left=3mm,right=6mm,top=5mm,bottom=5mm,papersize={14cm,34cm}]{geometry}
\usepackage{graphicx}
\usepackage{scrpage2}
\usepackage{listings}
\usepackage[usenames,dvipsnames,svgnames]{xcolor}
\usepackage[hidelinks]{hyperref}
\usepackage{amsmath}
\usepackage{amssymb}
\usepackage{mathtools}
\usepackage{caption}
\usepackage[framemethod=tikz]{mdframed}
\usepackage{dashbox}
\usepackage{enumitem}
\usepackage{ulem}
\usepackage{multicol}
\usepackage{svg}

\mdfsetup{%
	leftmargin=0cm,
	skipabove=0.2cm,
	linecolor=black,
	backgroundcolor=gray!10,
	linewidth=0.50pt,
	innerleftmargin=0.2cm,
	innerrightmargin=0.0cm,
	innertopmargin=-0.2cm,
	innerbottommargin=0.1cm
}

\newmdenv[
	font=\small,
	innerleftmargin=0.2cm,
	innerrightmargin=0.0cm,
	innertopmargin=-0.2cm,
	innerbottommargin=0.1cm
]{mdmath}

\newmdenv[
	font=\footnotesize,
	innerleftmargin=0.2cm,
	innerrightmargin=0.0cm,
	innertopmargin=-0.2cm,
	innerbottommargin=0.1cm
]{mdmathss}

\newmdenv[
	font=\small,
	innerleftmargin=0.2cm,
	innerrightmargin=0.2cm,
	innertopmargin=0.2cm,
	innerbottommargin=0.2cm
]{mdtext}

% \draw[option] (x,y) (w,h)
\newcommand{\sepline}[2]{
	\vspace{#1}
	\par\noindent\makebox[\linewidth][l]{%
		\hspace*{-\mdflength{innerleftmargin}}%
		\tikz\draw[thick,dashed,gray!60] (0,0) --%
		(\textwidth+\the\mdflength{innerleftmargin}+\the\mdflength{innerrightmargin},0);
	}\par\nobreak
	\vspace{#2}
}

%-------------------------------------------------------------------------------
% Array Spalten und Zeilen-Abstand
\arraycolsep=1.5pt\def\arraystretch{1.2}

%-------------------------------------------------------------------------------
% Difference
\newcommand*\diff{\mathop{}\!\mathrm{d}}
\newcommand*\Diff[1]{\mathop{}\!\mathrm{d^#1}}


%-------------------------------------------------------------------------------
% Absolute Value + Norm
\DeclarePairedDelimiter\abs{\lvert}{\rvert}%
\DeclarePairedDelimiter\norm{\lVert}{\rVert}%

%-------------------------------------------------------------------------------
% Euler
\newcommand*\e[1]{\mathrm{e}^{#1}}

%-------------------------------------------------------------------------------
% Font Small
\newcommand*\fs[1]{{\scriptstyle#1}}

%-------------------------------------------------------------------------------
% Plus-Minus Small
\newcommand*\pms{{\scriptstyle \pm } \,}

%-------------------------------------------------------------------------------
% Limes to Infinity
\let\limintold\limint
\renewcommand*\liminf[1]{\lim\limits_{#1 \to \infty}}

%-------------------------------------------------------------------------------
% Eigene Figure Umgebung
\newenvironment{Figure}
	{\par\noindent\minipage{\linewidth}}
	{\endminipage\par}

%-------------------------------------------------------------------------------
% Enge Align
\apptocmd\normalsize{% 
	\setlength\abovedisplayshortskip{0cm}% 
	\setlength\belowdisplayshortskip{0cm}% 
	\setlength\abovedisplayskip{0.2cm}% 
	\setlength\belowdisplayskip{0.25cm}% 
}{}{\undefined} 

%-------------------------------------------------------------------------------
% Römische Zahlen
\newcommand{\RNum}[1]{\uppercase\expandafter{\romannumeral #1\relax}}

%-------------------------------------------------------------------------------
% Dokument
\begin{document}
	\begin{minipage}{0.5\linewidth}
		\textbf{Aufgabe 3} \; Wir betrachten das vom Parameter $a \in \mathbb{R}$
		abhängige dynamische System
		\vspace{-0.25cm}
		\begin{flalign*}
			\boldsymbol{\dot{x}} = \left(
			\arraycolsep=3pt\def\arraystretch{1}
			\begin{array}{rr}
			-\alpha & -2 \\
			\alpha  & 1
			\end{array}
			\right) \boldsymbol{x} &
		\end{flalign*}
		\vspace{-0.75cm}
		\renewcommand{\labelenumi}{\alph{enumi})}
		\begin{enumerate}[leftmargin=0.5cm]
			\item
				Klassifizieren Sie den Fixpunkt $ \boldsymbol{x^*} = \boldsymbol{0} $
				in Abhängigkeit von $\alpha$.
		\end{enumerate}
	\end{minipage}
	\begin{mdmath}[leftmargin=0.5cm]
		\begin{flalign*}
			\arraycolsep=8pt\def\arraystretch{1.2}
			\begin{array}{lllllll}
				\text{\RNum{1}} & \Delta < 0 & \alpha < 0 &         \\
				\hline
				\text{\RNum{2}} & \Delta > 0 & \alpha > 0 & \tau < 0 & \alpha > 1 & \tau^2 - 4 \Delta \ge 0 & 3 + \sqrt{8} \le \alpha    \\
				\cline{6-7}
				\text{\RNum{3}} &            &            &          &            & \tau^2 - 4 \Delta <   0 & 1 < \alpha < 3 + \sqrt{8} \\
				\cline{4-7}
				\text{\RNum{4}} &            &            & \tau > 0 & \alpha < 1 & \tau^2 - 4 \Delta \ge 0 & 0 < \alpha \le 3 - \sqrt{8} \\
				\cline{6-7}
				\text{\RNum{5}} &            &            &          &            & \tau^2 - 4 \Delta <   0 & 3 - \sqrt{8} < \alpha < 1 \\
				\cline{4-7}
				\text{\RNum{6}} &            &            & \tau = 0 & \alpha = 1 & \\
				\hline
				\text{\RNum{7}} & \Delta = 0 & \alpha = 0
			\end{array} & &
		\end{flalign*}
		\sepline{-1cm}{-1cm}
		\begin{flalign*}
			\arraycolsep=8pt\def\arraystretch{1.2}
			\begin{array}{lll}
				\text{\RNum{1}} & \alpha < 0                  & \text{saddle point} \\
				\text{\RNum{2}} & 3 + \sqrt{8} \le \alpha     & \text{stable node} \\
				\text{\RNum{3}} & 1 < \alpha < 3 + \sqrt{8}   & \text{stable spiral} \\
				\text{\RNum{4}} & 0 < \alpha \le 3 - \sqrt{8} & \text{unstable node} \\
				\text{\RNum{5}} & 3 - \sqrt{8} < \alpha < 1   & \text{unstable spiral} \\
				\text{\RNum{6}} & \alpha = 1                  & \text{center, ellipse} \\
				\text{\RNum{7}} & \alpha = 0                  & \text{not isolate} 
			\end{array} & &
		\end{flalign*}
	\end{mdmath}
\end{document}
