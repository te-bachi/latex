

%-------------------------------------------------------------------------------
% Dokumenten Klasse
\documentclass[
	final,
	oneside,
	parskip=full,
	headings=standardclasses,
	headings=big,
	pointednumbers
]{scrartcl}

%-------------------------------------------------------------------------------
% Packete nutzen
\usepackage{ngerman,palatino,setspace}
\usepackage[T1]{fontenc}
\usepackage[utf8]{inputenc}
\usepackage[left=3mm,right=6mm,top=5mm,bottom=5mm,papersize={8cm,34cm}]{geometry}
\usepackage{graphicx}
\usepackage{scrpage2}
\usepackage{listings}
\usepackage[usenames,dvipsnames,svgnames]{xcolor}
\usepackage[hidelinks]{hyperref}
\usepackage{amsmath}
\usepackage{mathtools}
\usepackage{caption}
\usepackage[framemethod=tikz]{mdframed}
\usepackage{dashbox}
\usepackage{enumitem}
\usepackage{ulem}

\mdfsetup{%
	leftmargin=0cm,
	skipabove=0.2cm,
	linecolor=black,
	backgroundcolor=gray!10,
	linewidth=0.50pt,
	innerleftmargin=0.2cm,
	innerrightmargin=0.0cm,
	innertopmargin=-0.2cm,
	innerbottommargin=0.1cm
}

\newmdenv[
	font=\small,
	innerleftmargin=0.2cm,
	innerrightmargin=0.0cm,
	innertopmargin=-0.2cm,
	innerbottommargin=0.1cm
]{mdmath}

\newmdenv[
	font=\footnotesize,
	innerleftmargin=0.2cm,
	innerrightmargin=0.0cm,
	innertopmargin=-0.2cm,
	innerbottommargin=0.1cm
]{mdmathss}

\newmdenv[
	font=\small,
	innerleftmargin=0.2cm,
	innerrightmargin=0.2cm,
	innertopmargin=0.2cm,
	innerbottommargin=0.2cm
]{mdtext}

% \draw[option] (x,y) (w,h)
\newcommand{\sepline}[2]{
	\vspace{#1}
	\par\noindent\makebox[\linewidth][l]{%
		\hspace*{-\mdflength{innerleftmargin}}%
		\tikz\draw[thick,dashed,gray!60] (0,0) --%
		(\textwidth+\the\mdflength{innerleftmargin}+\the\mdflength{innerrightmargin},0);
	}\par\nobreak
	\vspace{#2}
}

%-------------------------------------------------------------------------------
% Array Spalten und Zeilen-Abstand
\arraycolsep=1.5pt\def\arraystretch{1.2}

%-------------------------------------------------------------------------------
% Difference
\newcommand*\diff{\mathop{}\!\mathrm{d}}
\newcommand*\Diff[1]{\mathop{}\!\mathrm{d^#1}}


%-------------------------------------------------------------------------------
% Absolute Value + Norm
\DeclarePairedDelimiter\abs{\lvert}{\rvert}%
\DeclarePairedDelimiter\norm{\lVert}{\rVert}%

%-------------------------------------------------------------------------------
% Euler
\newcommand*\e[1]{\mathrm{e}^{#1}}

%-------------------------------------------------------------------------------
% Font Small
\newcommand*\fs[1]{{\scriptstyle#1}}

%-------------------------------------------------------------------------------
% Plus-Minus Small
\newcommand*\pms{{\scriptstyle \pm } \,}

%-------------------------------------------------------------------------------
% Limes to Infinity
\let\limintold\limint
\renewcommand*\liminf[1]{\lim\limits_{#1 \to \infty}}

%-------------------------------------------------------------------------------
% Enge Align
\apptocmd\normalsize{% 
	\setlength\abovedisplayshortskip{0cm}% 
	\setlength\belowdisplayshortskip{0cm}% 
	\setlength\abovedisplayskip{0.2cm}% 
	\setlength\belowdisplayskip{0.25cm}% 
}{}{\undefined} 

%-------------------------------------------------------------------------------
% Dokument
\begin{document}
	\textbf{Aufgabe 2} \; Wir betrachten für beliebige $\omega, \delta > 0$ die
	Differentialgleichung
	% Float-Left align (& am Ende um es Links-Bündig zu machen)
	\begin{flalign*}
		y'' + \omega^2y &=\e{-\delta x} &
	\end{flalign*}
	\vspace{-0.75cm}
	
	\begin{mdmath}
		\begin{flalign*}
			y  &= C \cdot f(x) & \fs{\text{(Faktorregel)\quad}} & \\
			y' &= C \cdot f'(x)
		\end{flalign*}
		\sepline{-1.3cm}{-1.2cm}
		\begin{flalign*}
			y  &= f_1(x) + f_2(x) & \fs{\text{(Summenregel)\quad}} & \\
			y' &= f_1'(x) + f_2'(x)
		\end{flalign*}
		\sepline{-1.3cm}{-1.2cm}
		\begin{flalign*}
			y' &= F'(u) \cdot u'(x) & \fs{\text{(Kettenregel)\quad}} &
		\end{flalign*}
	\end{mdmath}
	
	\renewcommand{\labelenumi}{\alph{enumi})}
	\begin{enumerate}[leftmargin=0.5cm]
		\item
			Bestimmen Sie die allgemeine Lösung der Differentialgleichung.
			\begin{mdmath}
				\begin{flalign*}
					y'' + \omega^2y      &= 0 & \\
					\lambda^2 + \omega^2 &= 0 \\
					\lambda^2            &= -\omega^2 \\
				\end{flalign*}
				\vspace{-1.4cm}
				\begin{flalign*}
					\lambda_{1,2}        &= \pms \sqrt{\fs{-\omega^2}} = \pms \sqrt{\fs{-1}}  \sqrt{\fs{\omega^2}} & \\
					\lambda_{1,2}        &= \pms \omega i \hspace{0.5cm} \fs{(\lambda_{1,2} = \alpha \pms \beta i)} \\
					y_1                  &= \fs{ C_1 \e{\omega i} \; = \; K_1 \cdot \e{0} \cos(\omega x) + K_2 \cdot \e{0}  \sin(\omega x) } \\
					y_2                  &= \fs{ C_2 \e{-\omega i} \; = \; K_3 \cdot \cos(-\omega x) + K_4 \cdot \sin(-\omega x) } \\
					                     &= \fs{ K_3 \cdot \cos(\omega x) + K_4^* \cdot \sin(\omega x) } \\
					y_1                  &= y_2 \\
					& \hspace{-0.34cm} \uuline{y_h = K_1 \cdot \cos(\omega x) + K_2 \cdot \sin(\omega x)}
				\end{flalign*}
				\sepline{-1.2cm}{-1.2cm}
				\begin{flalign*}
					y_p                  &= A \cdot \e{-\delta x} & \\
					y_p'                 &= -\delta \cdot A \cdot \e{-\delta x} \\
					y_p''                &= \delta^2 \cdot A \cdot \e{-\delta x}
				\end{flalign*}
				\sepline{-1.3cm}{-1.2cm}
				\begin{flalign*}
					\begin{array}{lll}
						y_p''                                 &+ \; \omega^2 \cdot y_p                       &= \e{-\delta x} \\
						\delta^2 \cdot A \cdot \e{-\delta x}  &+ \; \omega^2 \cdot A \cdot \e{-\delta x}     &= \e{-\delta x} \\
						\multicolumn{2}{r}{ A \cdot \left( \delta^2 + \omega^2 \right) \cdot \e{-\delta x} } &= \e{-\delta x} \\
						\e{-\delta x}: & \multicolumn{1}{r}{ A \cdot \left( \delta^2 + \omega^2 \right) }    &= 1 \\
						\multicolumn{2}{r}{ A }                                                              &= \tfrac{1}{\delta^2 + \omega^2}
					\end{array} & &
				\end{flalign*}
				\sepline{-1.2cm}{-1.2cm}
				\begin{flalign*}
					y_p  &= A \cdot \e{-\delta x} & \\
					& \hspace{-0.34cm} \uuline{y_p = \tfrac{1}{\delta^2 + \omega^2} \cdot \e{-\delta x}}
				\end{flalign*}
				\sepline{-1.2cm}{-1.2cm}
				\begin{flalign*}
					y &= y_h + y_p & \\
					& \hspace{-0.2cm} \uuline{y = \fs{K_1 \cdot \cos(\omega x) + K_2 \cdot \sin(\omega x) + \tfrac{1}{\delta^2 + \omega^2} \cdot \e{-\delta x}}}
				\end{flalign*}
			\end{mdmath}
		\item 
			Lösen Sie (analytisch) Anfangswertproblem zu den Anfangswerten \\ $y(0) = 0$, $y'(0) = 0$.
			\begin{mdmath}
				\begin{flalign*}
					\fs{y \;}  & \fs{ = \; K_1 \cdot \cos(\omega x) + K_2 \cdot \sin(\omega x) + \tfrac{1}{\delta^2 + \omega^2} \cdot \e{-\delta x}} & \\
					\fs{y'}    & \fs{= \;  - K_1 \cdot \omega \cdot \sin(\omega x) + K_2 \cdot \omega \cdot \cos(\omega x) - \tfrac{\delta}{\delta^2 + \omega^2} \cdot \e{-\delta x}}
				\end{flalign*}
				\sepline{-1.3cm}{-1.2cm}
				\begin{flalign*}
					\fs{y(0)}  &= \fs{0} & \\
					\fs{0}     & = \fs{ K_1 \cdot \underbrace{\fs{\cos(0)}}_{=1} + K_2 \cdot \underbrace{\fs{\sin(0)}}_{=0} + \tfrac{1}{\delta^2 + \omega^2} \cdot \e{0} } \\
					           & \hspace{0.4cm} K_1 + \tfrac{1}{\delta^2 + \omega^2} = 0  \\
					           & \hspace{1.74cm} K_1  = - \tfrac{1}{\delta^2 + \omega^2}  \\
					\fs{y'(0)} &= \fs{0} \\
					\fs{0}     &= \fs{ - K_1 \cdot \omega \cdot \underbrace{\fs{\sin(0)}}_{=0} + K_2 \cdot \omega \cdot \underbrace{\fs{\cos(0)}}_{=1} - \tfrac{\delta}{\delta^2 + \omega^2} \cdot \e{0} } & \\
					           & \hspace{0.4cm} K_2 \cdot \omega - \tfrac{\delta}{\delta^2 + \omega^2} = 0  \\
					           & \hspace{2.21cm} K_2  = \tfrac{\delta}{\omega} \cdot \tfrac{1}{\delta^2 + \omega^2}
				\end{flalign*}
				\sepline{-1.3cm}{-1.2cm}
				\begin{flalign*}
					y \;= \; & - \tfrac{1}{\delta^2 + \omega^2} \cdot \cos(\omega x) \; + & \\
					    & \tfrac{\delta}{\omega} \cdot \tfrac{1}{\delta^2 + \omega^2} \cdot \sin(\omega x) \; + \\
					    & \tfrac{1}{\delta^2 + \omega^2} \cdot \e{-\delta x}
				\end{flalign*}
			\end{mdmath}
	\end{enumerate}
\end{document}
