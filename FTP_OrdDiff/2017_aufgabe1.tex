

%-------------------------------------------------------------------------------
% Dokumenten Klasse
\documentclass[
	final,
	oneside,
	parskip=full,
	headings=standardclasses,
	headings=big,
	pointednumbers
]{scrartcl}

%-------------------------------------------------------------------------------
% Packete nutzen
\usepackage{ngerman,palatino,setspace}
\usepackage[T1]{fontenc}
\usepackage[utf8]{inputenc}
\usepackage[left=3mm,right=6mm,top=5mm,bottom=5mm,papersize={8cm,28.5cm}]{geometry}
\usepackage{graphicx}
\usepackage{scrpage2}
\usepackage{listings}
\usepackage[usenames,dvipsnames,svgnames]{xcolor}
\usepackage[hidelinks]{hyperref}
\usepackage{amsmath}
\usepackage{mathtools}
\usepackage{caption}
\usepackage{mdframed}
\usepackage{dashbox}
\usepackage{enumitem}
\usepackage{ulem}

\mdfsetup{%
	leftmargin=0cm,
	skipabove=0.2cm,
	linecolor=black,
	backgroundcolor=gray!10,
	linewidth=0.50pt,
	innerleftmargin=0.2cm,
	innerrightmargin=0.0cm,
	innertopmargin=-0.2cm,
	innerbottommargin=0.1cm
}

\newmdenv[
	innerleftmargin=0.2cm,
	innerrightmargin=0.0cm,
	innertopmargin=-0.2cm,
	innerbottommargin=0.1cm
]{mdmath}

\newmdenv[
	font=\small,
	innerleftmargin=0.2cm,
	innerrightmargin=0.2cm,
	innertopmargin=0.2cm,
	innerbottommargin=0.2cm
]{mdtext}

%-------------------------------------------------------------------------------
% Difference
\newcommand*\diff{\mathop{}\!\mathrm{d}}
\newcommand*\Diff[1]{\mathop{}\!\mathrm{d^#1}}


%-------------------------------------------------------------------------------
% Absolute Value + Norm
\DeclarePairedDelimiter\abs{\lvert}{\rvert}%
\DeclarePairedDelimiter\norm{\lVert}{\rVert}%

%-------------------------------------------------------------------------------
% Euler
\newcommand*\e[1]{\mathrm{e}^{#1}}

%-------------------------------------------------------------------------------
% Font Small
\newcommand*\fs[1]{{\scriptstyle#1}}

%-------------------------------------------------------------------------------
% Plus-Minus Small
\newcommand*\pms{{\scriptstyle \pm } \,}

%-------------------------------------------------------------------------------
% Limes to Infinity
\let\limintold\limint
\renewcommand*\liminf[1]{\lim\limits_{#1 \to \infty}}

%-------------------------------------------------------------------------------
% Enge Align
\apptocmd\normalsize{% 
	\setlength\abovedisplayshortskip{0cm}% 
	\setlength\belowdisplayshortskip{0cm}% 
	\setlength\abovedisplayskip{0.2cm}% 
	\setlength\belowdisplayskip{0.25cm}% 
}{}{\undefined} 

%-------------------------------------------------------------------------------
% Dokument
\begin{document}
	\textbf{Aufgabe 1} \; Eine Bakterienkultur mit beschränktem Nahrungsangebot
	wächst nach dem Gesetz
	% Float-Left align (& am Ende um es Links-Bündig zu machen)
	\begin{flalign*}
		\dot{a}(t) &= k \cdot \left( N - a(t) \right) &
	\end{flalign*}
	Dabei sind $k > 0$ und $N > 0$ gegebene Konstanten, und $a \left( t \right)$
	bedeutet die Zahl der Bakterien zum Zeitpunkt $t$.
	\renewcommand{\labelenumi}{\alph{enumi})}
	\begin{enumerate}[leftmargin=0.5cm]
		\item
			Bestimmen und klassifizieren Sie den/""die Fixpunkt(e) dieses Systems
			in Abhängigkeit von Parameter $k$, $N$.
			\begin{mdmath}
				\begin{flalign*}
					\dot{a}(t) = f(a)    &= k \cdot \left( N - a(t) \right) & \\
					              f'(a)  &= -k \\
					             f(a^*)  &= kN - ka^* = 0\\
					             ka^*    &= kN \\
					             a^*     &= \tfrac{kN}{k} \hspace{0.7cm} \longrightarrow \uuline{a^* = N} \\
					             f'(a^*) &= -k < 0 \longrightarrow \uuline{stable}
				\end{flalign*}
			\end{mdmath}
		\item 
			Lösen Sie das Anfangswertproblem zum Anfangswert $a(0) = \frac{N}{2}$.
			\begin{mdmath}
				\begin{flalign*}
					\tfrac{\diff a}{\diff t}       &= k \cdot \left( N - a \right) & \\
					\int{\tfrac{1}{N - a}} \diff a &= k \int{1 \diff t} \\
					- \ln(\fs{\abs{N - a}})        &= kt + C_1 \\
					\ln(\fs{\abs{N - a}})          &= -kt - C_1 \\
					\abs{N - a}                    &= \e{-kt - C_1} \\
					N - a                          &= {\scriptstyle \pm } \, \e{-C_1} \e{-kt} = K \e{-kt} \\
					a                              &= N - K \e{-kt} \\
					a(0) = \tfrac{N}{2}            &= N - K \e{0} = N - K \\
					K                              &= N - \tfrac{N}{2} = \tfrac{2N - N}{2} = \tfrac{N}{2} \\
					& \hspace{-0.2cm} \uuline{a = N - \tfrac{N}{2} \cdot \e{-kt} }
				\end{flalign*}
			\end{mdmath}
		\item
			Bestimmen Sie $\liminf{t} a(t)$.
			\begin{mdmath}
				\begin{flalign*}
					\liminf{t} \e{-kt} &= 0 & \\
					a                  &= N - \tfrac{N}{2} \cdot 0 \\
					& \hspace{-0.2cm} \uuline{a = N }
				\end{flalign*}
			\end{mdmath}
		\item 
			Erklären Sie, wie die Resultate der Teile a) und c) zusammenhängen.
			\begin{mdtext}
				Wenn der Fixpunkt stabil ist, zieht sich die Kurve
				in $t \to \infty$ dem Fixpunkt an. Wenn instabil, dann
				stösst die Kurve sich vom Fixpunkt ab.
			\end{mdtext}
	\end{enumerate}
\end{document}
