

%-------------------------------------------------------------------------------
% Dokumenten Klasse
\documentclass[
	final,
	oneside,
	parskip=full,
	headings=standardclasses,
	headings=big,
	pointednumbers
]{scrartcl}

%-------------------------------------------------------------------------------
% Packete nutzen
\usepackage{ngerman,palatino,setspace}
\usepackage[T1]{fontenc}
\usepackage[utf8]{inputenc}
\usepackage[left=3mm,right=6mm,top=5mm,bottom=5mm,papersize={8cm,70cm}]{geometry}
\usepackage{graphicx}
\usepackage{scrpage2}
\usepackage{listings}
\usepackage[usenames,dvipsnames,svgnames]{xcolor}
\usepackage[hidelinks]{hyperref}
\usepackage{amsmath}
\usepackage{amssymb}
\usepackage{mathtools}
\usepackage{caption}
\usepackage[framemethod=tikz]{mdframed}
\usepackage{dashbox}
\usepackage{enumitem}
\usepackage{ulem}
\usepackage{multicol}
\usepackage{svg}
\usepackage{array}

\mdfsetup{%
	leftmargin=0cm,
	skipabove=0.2cm,
	linecolor=black,
	backgroundcolor=gray!10,
	linewidth=0.50pt,
	innerleftmargin=0.2cm,
	innerrightmargin=0.0cm,
	innertopmargin=-0.2cm,
	innerbottommargin=0.1cm
}

\newmdenv[
	font=\small,
	innerleftmargin=0.2cm,
	innerrightmargin=0.0cm,
	innertopmargin=-0.2cm,
	innerbottommargin=0.1cm
]{mdmath}

\newmdenv[
	font=\footnotesize,
	innerleftmargin=0.2cm,
	innerrightmargin=0.0cm,
	innertopmargin=-0.2cm,
	innerbottommargin=0.1cm
]{mdmathss}

\newmdenv[
	font=\small,
	innerleftmargin=0.2cm,
	innerrightmargin=0.2cm,
	innertopmargin=0.2cm,
	innerbottommargin=0.2cm
]{mdtext}

% \draw[option] (x,y) (w,h)
\newcommand{\sepline}[2]{
	\vspace{#1}
	\par\noindent\makebox[\linewidth][l]{%
		\hspace*{-\mdflength{innerleftmargin}}%
		\tikz\draw[thick,dashed,gray!60] (0,0) --%
		(\textwidth+\the\mdflength{innerleftmargin}+\the\mdflength{innerrightmargin},0);
	}\par\nobreak
	\vspace{#2}
}

%-------------------------------------------------------------------------------
% Array Spalten und Zeilen-Abstand
\arraycolsep=1.5pt\def\arraystretch{1.2}

%-------------------------------------------------------------------------------
% Difference
\newcommand*\diff{\mathop{}\!\mathrm{d}}
\newcommand*\Diff[1]{\mathop{}\!\mathrm{d^#1}}


%-------------------------------------------------------------------------------
% Absolute Value + Norm
\DeclarePairedDelimiter\abs{\lvert}{\rvert}%
\DeclarePairedDelimiter\norm{\lVert}{\rVert}%

%-------------------------------------------------------------------------------
% Euler
\newcommand*\e[1]{\mathrm{e}^{#1}}

%-------------------------------------------------------------------------------
% Font Small
\newcommand*\fs[1]{{\scriptstyle#1}}

%-------------------------------------------------------------------------------
% Plus-Minus Small
\newcommand*\pms{{\scriptstyle \pm } \,}

%-------------------------------------------------------------------------------
% Limes to Infinity
\let\limintold\limint
\renewcommand*\liminf[1]{\lim\limits_{#1 \to \infty}}

%-------------------------------------------------------------------------------
% Eigene Figure Umgebung
\newenvironment{Figure}
	{\par\noindent\minipage{\linewidth}}
	{\endminipage\par}

%-------------------------------------------------------------------------------
% Enge Align
\apptocmd\normalsize{% 
	\setlength\abovedisplayshortskip{0cm}% 
	\setlength\belowdisplayshortskip{0cm}% 
	\setlength\abovedisplayskip{0.2cm}% 
	\setlength\belowdisplayskip{0.25cm}% 
}{}{\undefined} 

%-------------------------------------------------------------------------------
% Römische Zahlen
\newcommand{\RNum}[1]{\uppercase\expandafter{\romannumeral #1\relax}}

%-------------------------------------------------------------------------------
% Dokument
\begin{document}
	\textbf{Aufgabe 3} \; Wir betrachten das vom Parameter $a \in \mathbb{R}$
	abhängige dynamische System
	% Float-Left align (& am Ende um es Links-Bündig zu machen)
	\vspace{-0.25cm}
	\begin{flalign*}
		\boldsymbol{\dot{x}} = \left(
		\arraycolsep=3pt\def\arraystretch{1}
		\begin{array}{rr}
			-\alpha & -2 \\
			\alpha  & 1
		\end{array}
		\right) \boldsymbol{x} &
	\end{flalign*}
	\vspace{-0.75cm}
	
	\renewcommand{\labelenumi}{\alph{enumi})}
	\begin{enumerate}[leftmargin=0.5cm]
		\item
			Klassifizieren Sie den Fixpunkt $ \boldsymbol{x^*} = \boldsymbol{0} $
			in Abhängigkeit von $\alpha$.
			\begin{mdmath}
				\begin{flalign*}
					\tau              &= -\alpha + 1 = \uuline{1 - \alpha} & \\
					\Delta            &= -\alpha + 2\alpha = 2\alpha - \alpha = \uuline{\alpha} \\
					\tau^2 - 4 \Delta &= \left( 1 - \alpha \right)^2 - 4\alpha \\
					                  &= 1 - 2\alpha + \alpha^2 - 4\alpha \\
					0                 &= \alpha^2 - 6\alpha + 1 \\
					\alpha_{1,2}      &= \tfrac{-b \pms \sqrt{b^2 - 4ac}}{2a} \\
					                  &= \tfrac{1}{2} \left( 6 \pms \sqrt{36 - 4} \right) \\
					                  &= 3 \pms \sqrt{\tfrac{32}{4}} \\
					                  &= \uuline{3 \pms \sqrt{8}} \\
					                  & \hspace{0.5cm} \alpha_{1} \approx 0.17 \quad \alpha_{2} \approx 5.83
				\end{flalign*}
				\vspace{-0.75cm}
				\begin{Figure}
					\centering
					\includegraphics[width=5cm]{2017_aufgabe3_img.pdf}
				\end{Figure}
				\sepline{-0.1cm}{-0.4cm}
				\text{Siehe Beilageblatt}
				\sepline{-0.6cm}{-1.2cm}
				\begin{flalign*}
					\arraycolsep=6pt\def\arraystretch{1.2}
					\begin{array}{lll}
						\fs{\text{\RNum{1}}} & \alpha < 0                  & \fs{\text{saddle point}} \\
						\fs{\text{\RNum{2}}} & 3 + \sqrt{8} \le \alpha     & \fs{\text{stable node}} \\
						\fs{\text{\RNum{3}}} & 1 < \alpha < 3 + \sqrt{8}   & \fs{\text{stable spiral}} \\
						\fs{\text{\RNum{4}}} & 0 < \alpha \le 3 - \sqrt{8} & \fs{\text{unstable node}} \\
						\fs{\text{\RNum{5}}} & 3 - \sqrt{8} < \alpha < 1   & \fs{\text{unstable spiral}} \\
						\fs{\text{\RNum{6}}} & \alpha = 1                  & \fs{\text{center, ellipse}} \\
						\fs{\text{\RNum{7}}} & \alpha = 0                  & \fs{\text{not isolate}}
					\end{array} & &
				\end{flalign*}
			\end{mdmath}
		\item
			Bestimmen Sie im Fall $\alpha = -1$ (analytisch) die Lösung des Anfangswertproblems
			zum Anfangswert \\ $\boldsymbol{x}(0) = \left(
			\begin{array}{lll}
				1 \\
				1
			\end{array} \right) $
			
			\begin{mdmath}
				\begin{flalign*}
					\alpha = -1 & & \boldsymbol{\dot{x}} = \left(
					\arraycolsep=4pt\def\arraystretch{1}
					\begin{array}{rr}
						 1 & -2 \\
						-1 &  1
					\end{array} \right) \boldsymbol{x} \hspace{1cm}
				\end{flalign*}
				\vspace{-0.5cm}
				\begin{flalign*}
					\left|
					\arraycolsep=4pt\def\arraystretch{1}
					\begin{array}{cc}
						 1 - \lambda & -2 \\
						-1           &  1 - \lambda
					\end{array}
					\right|                               &= \boldsymbol{0} & \\
					(1 - \lambda)(1 - \lambda) - 2        &= 0 \\
					1 - \lambda - \lambda + \lambda^2 - 2 &= 0 \\
					\lambda^2 - 2\lambda - 1              &= 0 \\
					\lambda_{1,2}                         &= \tfrac{-b \pms \sqrt{b^2 - 4ac}}{2a} \\
					\lambda_{1,2}                         &= \tfrac{1}{2} \left( 2 \pms \sqrt{4 + 4} \right) \\
					                                      &= 1 \pms \sqrt{\tfrac{8}{4}} \\
					                                      & \hspace{-0.6cm} \uuline{\lambda_{1,2} = 1 \pms \sqrt{2}} \\
				\end{flalign*}
				\vspace{-1.6cm}
				\begin{flalign*}
					\fs{\mathtt{cPolyRoots((1-a)(1-a)-2, a)}} & &
				\end{flalign*}
				\sepline{-1.3cm}{-1.0cm}
				\begin{flalign*}
					\fs{\mathtt{eigvec2 \left(
					\left[
					\arraycolsep=3pt\def\arraystretch{1}
					\begin{array}{rr}
						\fs{\mathtt{1}}  & \fs{\mathtt{-2}} \\
						\fs{\mathtt{-1}} & \fs{\mathtt{1}}
					\end{array}
					\right], \left\{ 1 + \sqrt{2}, \; 1 - \sqrt{2} \right\}
					\right)}} & &
				\end{flalign*}
				\sepline{-1.0cm}{-1.2cm}
				%--- lambda 1 ---------------------------------------------
				\begin{flalign*}
					\boldsymbol{\lambda_1} & \boldsymbol{= 1 + \sqrt{2}} & 
				\end{flalign*}
				\vspace{-0.75cm}
				\begin{flalign*}
					& \left(
						\arraycolsep=3pt\def\arraystretch{1}
						\begin{array}{cc}
							-\sqrt{2} & -2 \\
							-1        & -\sqrt{2}
						\end{array}
					\right) & \\ %----------------------------
					& \left(
						\arraycolsep=3pt\def\arraystretch{1}
						\begin{array}{cc}
							1   & \sqrt{2} \\
							0   & 0
						\end{array}
					\right) 
					\left(
						\arraycolsep=3pt\def\arraystretch{1}
						\begin{array}{cc}
							v_1 \\
							v_2
						\end{array}
					\right) =
					\left(
					\arraycolsep=3pt\def\arraystretch{1}
					\begin{array}{cc}
					0 \\
					0
					\end{array}
					\right) &
				\end{flalign*}
				\vspace{-0.5cm}
				\begin{flalign*}
					1 \cdot v_1 + \sqrt{2} \cdot v_2 &= 0 & \\
					1 \cdot v_1                      &= -\sqrt{2} \cdot v_2 \\
					v_2                              &= 1 \\
					v_1                              &= -\sqrt{2} \\
					v & =
					\left(
						\arraycolsep=0pt\def\arraystretch{1}
						\begin{array}{cc}
							\fs{-\sqrt{2}} \\
							\fs{1}
						\end{array}
					\right) &
				\end{flalign*}
				\sepline{-1.0cm}{-1.0cm}
				%--- lambda 2 ---------------------------------------------
				\begin{flalign*}
					\boldsymbol{\lambda_2} & \boldsymbol{= 1 - \sqrt{2}} & 
				\end{flalign*}
				\vspace{-0.75cm}
				\begin{flalign*}
					& \left(
					\arraycolsep=3pt\def\arraystretch{1}
					\begin{array}{cc}
					\sqrt{2} & -2 \\
					-1       & \sqrt{2}
					\end{array}
					\right) & \\ %----------------------------
					& \left(
					\arraycolsep=3pt\def\arraystretch{1}
					\begin{array}{cc}
					1   & -\sqrt{2} \\
					0   & 0
					\end{array}
					\right) 
					\left(
					\arraycolsep=3pt\def\arraystretch{1}
					\begin{array}{cc}
					w_1 \\
					w_2
					\end{array}
					\right) =
					\left(
					\arraycolsep=3pt\def\arraystretch{1}
					\begin{array}{cc}
					0 \\
					0
					\end{array}
					\right) &
				\end{flalign*}
				\vspace{-0.5cm}
				\begin{flalign*}
					1 \cdot w_1 - \sqrt{2} \cdot w_2 &= 0 & \\
					1 \cdot w_1                      &= \sqrt{2} \cdot w_2 \\
					w_2                              &= 1 \\
					w_1                              &= \sqrt{2} \\
					w & =
					\left(
					\arraycolsep=0pt\def\arraystretch{1}
					\begin{array}{cc}
					\fs{\sqrt{2}} \\
					\fs{1}
					\end{array}
					\right) &
				\end{flalign*}
				\sepline{-1.0cm}{-0.4cm}
				\text{Allgemeine Lösung}
				\vspace{-0.2cm}
				\begin{flalign*}
					\boldsymbol{x} &= C_1 \cdot \e{(1 + \sqrt{2})t} \left(
					\arraycolsep=0pt\def\arraystretch{1}
					\begin{array}{cc}
						\fs{-\sqrt{2}} \\
						\fs{1}
					\end{array}
					\right) + & \\
					& \hspace{0.46cm} C_2 \cdot \e{(1 - \sqrt{2})t} \left(
					\arraycolsep=0pt\def\arraystretch{1}
					\begin{array}{cc}
						\fs{\sqrt{2}} \\
						\fs{1}
					\end{array}
					\right)
				\end{flalign*}
				\sepline{-1.0cm}{-0.4cm}
				\text{Anfangsbedingung}
				\vspace{-0.2cm}
				\begin{flalign*}
					%---------
					% 1. Zeile
					\boldsymbol{x}(0) &= \left(
					\begin{array}{lll}
						1 \\
						1
					\end{array} \right) & \\
					%---------
					% 2. Zeile
					\boldsymbol{x}(0) &= C_1 \left(
					\arraycolsep=0pt\def\arraystretch{1}
					\begin{array}{cc}
						\fs{-\sqrt{2}} \\
						\fs{1}
					\end{array}
					\right) + C_2 \left(
					\arraycolsep=0pt\def\arraystretch{1}
					\begin{array}{cc}
						\fs{\sqrt{2}} \\
						\fs{1}
					\end{array}
					\right) =  \left(
					\begin{array}{lll}
						1 \\
						1
					\end{array} \right) \\
					%---------
					% 3. Zeile
					& \hspace{-0.2cm} \left|
					\arraycolsep=3pt\def\arraystretch{1}
					\begin{array}{llll}
						-\sqrt{2} C_1 & + \sqrt{2} C_2 & = & 1 \\
						C_1           & + C_2          & = & 1
					\end{array}
					\right|
				\end{flalign*}
				\vspace{-0.5cm}
				\begin{flalign*}
					C_1 + C_2                                      & = 1 & \\
					C_2                                            & = 1 - C_1 \\
					-\sqrt{2} C_1 + \sqrt{2} C_2                   & = 1 \\
					-\sqrt{2} C_1 + \sqrt{2} \left(1 - C_1 \right) & = 1 \\
					-\sqrt{2} C_1 + \sqrt{2} - \sqrt{2} C_1        & = 1 \\
					-2 \sqrt{2} C_1                                & = 1 - \sqrt{2}  \\
					2 \sqrt{2} C_1                                 & = \sqrt{2} - 1  \\
				\end{flalign*}
				\vspace{-1.5cm}
				\begin{align*}
					\setlength\extrarowheight{6pt}
					\arraycolsep=8pt\def\arraystretch{1}
					\begin{array}{ll}
						%-- C_1 -------------------------------
						C_1 = \tfrac{\sqrt{2} - 1}{2 \sqrt{2}}              & \\
						C_1 = \tfrac{\sqrt{2} \left(\sqrt{2} - 1 \right)}
						            {2 \sqrt{2}\sqrt{2}}                    & C_1 = \tfrac{\sqrt{2}}{2 \sqrt{2}} - \tfrac{1}{2 \sqrt{2}} \\
						\uuline{C_1 = \tfrac{2 - \sqrt{2}}{4}}              & \uuline{C_1 = \tfrac{1}{2} - \tfrac{1}{2 \sqrt{2}}} \\
						%-- C_2 -------------------------------
						C_2 = 1 - C_1                                       & \\
						C_2 = 1 - \tfrac{2 - \sqrt{2}}{4}                   & C_2 = 1 - \fs{\left(\tfrac{1}{2} - \tfrac{1}{2 \sqrt{2}}\right)} \\
						C_2 =\tfrac{4 - \left(2 - \sqrt{2}\right)}{4}       & C_2 = 1 - \tfrac{1}{2} + \tfrac{1}{2 \sqrt{2}} \\
						\uuline{C_2 = \tfrac{2 + \sqrt{2}}{4}}              & \uuline{C_2 = \tfrac{1}{2} + \tfrac{1}{2 \sqrt{2}}}
					\end{array}
				\end{align*}
				\vspace{-0.5cm}
				\begin{flalign*}
					\boldsymbol{x} &= \tfrac{2 - \sqrt{2}}{4} \cdot \e{(1 + \sqrt{2})t} \left(
					\arraycolsep=0pt\def\arraystretch{1}
					\begin{array}{cc}
						\fs{-\sqrt{2}} \\
						\fs{1}
					\end{array}
					\right) + & \\
					& \hspace{0.46cm} \tfrac{2 + \sqrt{2}}{4} \cdot \e{(1 - \sqrt{2})t} \left(
					\arraycolsep=0pt\def\arraystretch{1}
					\begin{array}{cc}
						\fs{\sqrt{2}} \\
						\fs{1}
					\end{array}
					\right)
				\end{flalign*}
			\end{mdmath}
			
	\end{enumerate}
\end{document}
