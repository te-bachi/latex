

%-------------------------------------------------------------------------------
% Dokumenten Klasse
\documentclass[
	final,
	a4paper,
	oneside,
	parskip=full,
	headings=standardclasses,
	headings=big,
	pointednumbers
]{scrartcl}

%-------------------------------------------------------------------------------
% Packete nutzen
\usepackage{ngerman,palatino,setspace}
\usepackage[T1]{fontenc}
\usepackage[latin9]{inputenc}
\usepackage[left=35mm,right=35mm,top=25mm,bottom=25mm]{geometry}
\usepackage{graphicx}
\usepackage{scrpage2}
\usepackage{listings}
\usepackage[usenames,dvipsnames,svgnames]{xcolor}
\usepackage[hidelinks]{hyperref}
\usepackage{amsmath}
\usepackage{caption}
\usepackage{mdframed}
\usepackage{dashbox}

\usepackage{multirow}% http://ctan.org/pkg/multirow
\usepackage{hhline}% http://ctan.org/pkg/hhline

%-------------------------------------------------------------------------------
% Hintergrundfarbe von Packet 'framed' mit Kommando 'shaded' nutzen
\colorlet{shadecolor}{gray!25}

%-------------------------------------------------------------------------------
% F�r "align" Umgebung eine Zelle nach links/rechzs verschieben
\makeatletter
\newcommand{\pushright}[1]{\ifmeasuring@#1\else\omit\hfill$\displaystyle#1$\fi\ignorespaces}
\newcommand{\pushleft}[1]{\ifmeasuring@#1\else\omit$\displaystyle#1$\hfill\fi\ignorespaces}
\makeatother

%-------------------------------------------------------------------------------
% Kopf- und Fusszeile

\pagestyle{scrheadings}
\clearscrheadfoot
\cfoot[Seite \thepage]{Seite \thepage}

%-------------------------------------------------------------------------------
%
\date{\today}

%-------------------------------------------------------------------------------
% Dokumenten Einstellungen

% Section Abst�nde
\RedeclareSectionCommand[
	beforeskip=-1\baselineskip,
	afterskip=0.001\baselineskip
]{section}
\RedeclareSectionCommand[
	beforeskip=0\baselineskip,
	afterskip=0.001\baselineskip
]{subsection}

% Formeln
\DeclareCaptionType{mycapequ}[aa][bb]
\captionsetup[mycapequ]{labelformat=empty}

\def\changemargin#1#2{\list{}{\rightmargin#2\leftmargin#1}\item[]}
\let\endchangemargin=\endlist 

%-------------------------------------------------------------------------------
% Listings
\lstset{
	language=Matlab,
	breaklines=true,
	numbers=left,
	numberstyle=\tiny,
	numbersep=5pt,
	captionpos=b,
	basicstyle=\footnotesize\ttfamily,
	stringstyle=\color{magenta},
	identifierstyle=\color{black},
	keywordstyle=\color{blue}, 
	commentstyle=\color{DarkGreen}
}

\newcommand{\mylisting}[2][]{%
	\lstinputlisting[caption={\texttt{\detokenize{#2}}},#1]{#2}%
}

%-------------------------------------------------------------------------------
% Dokument
\begin{document}
	
	\section*{L�sen einer DGL 1. Ordnung}
	Die DGL ist wie folgt:
	\begin{mycapequ}[!ht]
		\vspace{-0.75cm}
		\begin{align*}
			a\left(x\right) \cdot y\mkern3mu' + b\left(x\right) \cdot y & = g\left(x\right) \\
			y\mkern3mu' + f\left(x\right) \cdot y & = g\left(x\right) \\
			% Linie
			\cline{1-2}
			L \cdot i\mkern3mu' + R \cdot i & = u_{0} \cdot \sin \left( \omega \cdot t \right) \\
			i\mkern3mu' + \frac{R}{L} \cdot i & = \frac{u_{0}}{L} \cdot \sin \left( \omega \cdot t \right)
		\end{align*}
		\vspace{-1.0cm}
	\end{mycapequ}
	
	\subsection*{Homogene L�sung: Trennung der Variablen}
	Finde ein $y$, welches die homogene Differentialgleichung (DGL) l�st.
	\begin{mycapequ}[!ht]
		\vspace{-0.25cm}
		\begin{align*}
		i' + \frac{R}{L} \cdot i & = 0 \\
		\frac{\mathrm{d}i}{\mathrm{d}t} & = -\frac{R}{L} \cdot i \\
		\int{\frac{1}{i} \cdot \mathrm{d}i} & = -\frac{R}{L} \int{1 \cdot \mathrm{d}t} \\
		\ln{\left(\vert i \vert \right)} & = -\frac{R}{L} \cdot t + K \\
		\vert i \vert  & = \mathrm{e}^{-\frac{R}{L} \, \cdot \, t\, + \,K} \\
		i & = \pm\mathrm{e}^{K} \cdot \mathrm{e}^{-\frac{R}{L} \, \cdot \, t} \\
		i_{h} \left( t \right) & = C \cdot \mathrm{e}^{-\frac{R}{L} \, \cdot \, t} \\
		\end{align*}
		\vspace{-1.75cm}
	\end{mycapequ}

	\begin{mycapequ}[!ht]
		\begin{mdframed}[leftmargin=0cm,
			skipabove=1cm,    
			linecolor=black,
			backgroundcolor=gray!10,
			linewidth=0.75pt,
			innerleftmargin=0em,
			innerrightmargin=0em,
			innertopmargin=0.75em,
			innerbottommargin=0.5em,
			]
			\begin{addmargin}[1em]{2em}% 1em left, 2em right
				\textbf{Integration einer homogenen linearen Differentialgleichung 1. Ordnung}
				
				\vspace{-0.50cm}
				\begin{align*}
					y\mkern3mu' + f\left(x\right) \cdot y & = 0 \\
				\end{align*}
				\vspace{-1.00cm}
				
				Durch \textit{Trennen der Variablen}. Die \textit{allgemeine L�sung} ist
				
				\vspace{-0.50cm}
				\begin{align*}
					y & = C \cdot \mathrm{e}^{-\int f\left(x\right)\; \mathrm{d}x}
				\end{align*}
				%\vspace{-1.00cm}
				
			\end{addmargin}
		\end{mdframed}
	\end{mycapequ}
	
	\subsection*{Inhomogene L�sung: Variante der Konstanten}
	Eine \textit{inhomogene DGL 1. Ordnung} l�sst sich durch die \textit{Variante der Konstanten} l�sen.
	Dazu benutzt man die \textit{allgemeine L�sung} der \textit{homogenen DGL} und ersetzt die Integrationskonstante $C$ durch eine (noch unbekannte) Funktion $C\left(x\right)$ und berechnet mit dem L�sungsansatz $y_{h} \left( x \right) $ die Ableitung $y_{h}\mkern3mu' \left( x \right) $ unter Verwendung der \textit{Produkt- und Kettenregel}.
	\begin{mycapequ}[!ht]
		\vspace{-0.75cm}
		\begin{align*}
			% 1. Zeile
			y & = C\left(x\right) \cdot \mathrm{e}^{-\int f\left(x\right)\; \mathrm{d}x} \\
			% 2. Zeile
			y\hskip2pt' & = \dbox{$C\hskip2pt'\left(x\right)$} \cdot
			                \mathrm{e}^{-\int f\left(x\right)\; \mathrm{d}x} - 
			                C\left(x\right) \cdot \dbox{$f\left(x\right) \cdot \mathrm{e}^{-\int f\left(x\right)\; \mathrm{d}x}$} \\
           % Linie
           \cline{1-2}
			% 3. Zeile
			i \left( t \right) & = C \left( t \right) \cdot \mathrm{e}^{-\frac{R}{L} \, \cdot \, t} \\
			% 4. Zeile
			i\hskip2pt'\left( t \right) & = \dbox{$C\hskip2pt'\left( t \right)$} \cdot \mathrm{e}^{-\frac{R}{L} \, \cdot \, t} + C \left( t \right) \cdot \dbox{$\left(-\frac{R}{L}\right) \cdot \mathrm{e}^{-\frac{R}{L} \, \cdot \, t}$} \\
		\end{align*}
		\vspace{-1.75cm}
	\end{mycapequ}

	\begin{mycapequ}[!ht]
		\begin{mdframed}[leftmargin=0cm,
			skipabove=1cm,    
			linecolor=black,
			backgroundcolor=gray!10,
			linewidth=0.75pt,
			innerleftmargin=0em,
			innerrightmargin=0em,
			innertopmargin=0.75em,
			innerbottommargin=0.5em,
			]
			\begin{addmargin}[1em]{2em}% 1em left, 2em right
				\textbf{Produktregel}
				\vspace{-0.90cm}
				\begin{align*}
					y\left(x\right) &= u\left(x\right) \cdot v\left(x\right) \\
					y\hskip2pt'\left(x\right) &= u\hskip2pt'\left(x\right) \cdot v\left(x\right) + u\left(x\right) \cdot v\hskip2pt'\left(x\right)
				\end{align*}
				
				%\vspace{-0.75cm}
				\textbf{Kettenregel}
				\vspace{-0.88cm}
				\begin{align*}
					y\left(x\right) & = f\left(x\right) \\
					y\hskip2pt'\left(x\right) & = f\hskip2pt'\left(x\right) = F\hskip2pt'\left(u\right) \cdot u\hskip2pt'\left(x\right) \phantom{aaa\;} \\
				\end{align*}
				\vspace{-1.25cm}
			\end{addmargin}
		\end{mdframed}
	\end{mycapequ}

	%\vspace{-0.50cm}
	Einsetzen der gefundenen Funktion und deren Ableitung in die DGL.
	\begin{mycapequ}[!ht]
		\vspace{-0.50cm}
		\begin{alignat*}{3}
			% 1. Zeile
			\dbox{$y\mkern3mu'$} & + f\left(x\right) \cdot \dbox{$y$} && = g\left(x\right) \\
			% 2. Zeile
			\dbox{$C\hskip2pt'\left(x\right) \cdot
			\mathrm{e}^{-\int f\left(x\right)\; \mathrm{d}x} - 
			C\left(x\right) \cdot f\left(x\right) \cdot \mathrm{e}^{-\int f\left(x\right)\; \mathrm{d}x}$}
			& + f\left(x\right) \cdot \dbox{$C\left(x\right) \cdot \mathrm{e}^{-\int f\left(x\right)\;
			\mathrm{d}x}$} && = g\left(x\right) \\
			% 3. Zeile
			\phantom{\;}C\hskip2pt'\left(x\right) \cdot
			\mathrm{e}^{-\int f\left(x\right)\; \mathrm{d}x} 
			\underbrace{- C\left(x\right) \cdot f\left(x\right) \cdot \mathrm{e}^{-\int f\left(x\right)\;
			\mathrm{d}x}  + f\left(x\right) \cdot C\left(x\right) \cdot \mathrm{e}^{-\int f\left(x\right)\;
			\mathrm{d}x}}_{= 0} \span && = g\left(x\right) \\
			% Linie
			\cline{1-4}
			% 4. Zeile
			\dbox{$i\mkern3mu'$} & + \frac{R}{L} \cdot \dbox{$i$} && = \frac{u_{0}}{L} \cdot \sin \left( \omega \cdot t \right) \\
			% 5. Zeile
			\dbox{$C\hskip2pt'\left( t \right) \cdot \mathrm{e}^{-\frac{R}{L} \, \cdot \, t} + C \left( t \right) \cdot \left(-\frac{R}{L}\right) \cdot \mathrm{e}^{-\frac{R}{L} \, \cdot \, t}$} & + \frac{R}{L} \, \cdot \dbox{$C \left( t \right) \cdot \mathrm{e}^{-\frac{R}{L} \, \cdot \, t}$} && = \frac{u_{0}}{L} \cdot \sin \left( \omega \cdot t \right) \\
			% 6. Zeile
			\phantom{aaaa\;\,}C\hskip2pt'\left( t \right) \cdot \mathrm{e}^{-\frac{R}{L} \, \cdot \, t} \underbrace{- \frac{R}{L} \cdot C \left( t \right) \cdot \mathrm{e}^{-\frac{R}{L} \, \cdot \, t} + \frac{R}{L} \, \cdot C \left( t \right) \cdot \mathrm{e}^{-\frac{R}{L} \, \cdot \, t}}_{= 0} \span && = \frac{u_{0}}{L} \cdot \sin \left( \omega \cdot t \right) \\
		\end{alignat*}
		\vspace{-2.0cm}
	\end{mycapequ}

    Durch K�rzen bekommt man wieder eine einfachere Form.
	\begin{mycapequ}[!ht]
		\vspace{-0.50cm}
		\begin{align*}
			% 1. Zeile
			C\hskip2pt'\left(x\right) \cdot
				\mathrm{e}^{-\int f\left(x\right)\; \mathrm{d}x} & = g\left(x\right) \\
			% 2. Zeile
			C\hskip2pt'\left(x\right)  & = g\left(x\right) \cdot
			\mathrm{e}\mkern3mu^{\int f\left(x\right)\; \mathrm{d}x}\\
			% 3. Zeile
			C\left(x\right)  & = \int g\left(x\right) \cdot
			\mathrm{e}\mkern3mu^{\int f\left(x\right)\; \mathrm{d}x} \; \mathrm{d}x + K\\
			% Linie
			\cline{1-2}
			% 4. Zeile
			C\hskip2pt'\left( t \right) \cdot \mathrm{e}^{-\frac{R}{L} \, \cdot \, t} & = \frac{u_{0}}{L} \cdot \sin \left( \omega \cdot t \right) \\
			% 5. Zeile
			C\hskip2pt'\left( t \right) & = \frac{u_{0}}{L} \cdot \sin \left( \omega \cdot t \right)   \cdot \mathrm{e}^{\frac{R}{L} \, \cdot \, t} \\
			% 6. Zeile
			C \left( t \right) & = \frac{u_{0}}{L} \cdot \int \sin \left( \omega \cdot t \right)   \cdot \mathrm{e}^{\frac{R}{L} \, \cdot \, t} \; \mathrm{d}t + K\\
		\end{align*}
		\vspace{-1.75cm}
	\end{mycapequ}

	\newpage

	Das Integral ausrechnen geht so.
	\begin{mycapequ}[!ht]
		\vspace{-0.50cm}
		\begin{align*}
			% 1. Zeile
			\int \sin \left( \omega \cdot t \right)   \cdot \mathrm{e}^{\frac{R}{L} \, \cdot \, t} \; \mathrm{d}t = \overbrace{\int \sin \left( \omega \cdot t \right)   \cdot \mathrm{e}\mkern3mu^{\tau \, t} \; \mathrm{d}t}^{I} = \left[...\right] \span \\
			% Box
			\fbox{$
				\begin{aligned}
				u &= \sin\left(\omega \cdot t \right) & v\hskip2pt' &= \mathrm{e}\mkern3mu^{\tau \, t}\\
				   & \downarrow \scriptstyle\frac{u}{\mathrm{d}t} &  &\downarrow \scriptstyle\int \mathrm{d}t\\
				u\hskip2pt' &= \omega \cdot \cos\left(\omega \cdot t \right) & v &= \frac{1}{\tau} \cdot \mathrm{e}\mkern3mu^{\tau \, t}\\
				\end{aligned}
			$} \span \\
			% 2. Zeile
			\left[...\right] = \underbrace{\dbox{$\sin\left(\omega \cdot t \right)$}}_{u} \cdot \underbrace{\dbox{$\frac{1}{\tau} \cdot \mathrm{e}\mkern3mu^{\tau \, t}$}}_{v} - \int \underbrace{\dbox{$\omega \cdot \cos\left(\omega \cdot t \right)$}}_{u\hskip2pt'} \cdot \underbrace{\dbox{$\frac{1}{\tau} \cdot \mathrm{e}\mkern3mu^{\tau \, t}$}}_{v}  \; \mathrm{d}t \span \\
			% 3. Zeile
			\left[...\right] = \frac{1}{\tau} \cdot \sin\left(\omega \cdot t \right) \cdot \mathrm{e}\mkern3mu^{\tau \, t} - \frac{\omega}{\tau} \int \cos\left(\omega \cdot t \right) \cdot \mathrm{e}\mkern3mu^{\tau \, t}  \; \mathrm{d}t \span\\
			% Box
			\fbox{$
				\begin{aligned}
				u &= \cos\left(\omega \cdot t \right) & v\hskip2pt' &= \mathrm{e}\mkern3mu^{\tau \, t}\\
				& \downarrow \scriptstyle\frac{u}{\mathrm{d}t} &  &\downarrow \scriptstyle\int \mathrm{d}t\\
				u\hskip2pt' &= -\omega \cdot \sin\left(\omega \cdot t \right) & v &= \frac{1}{\tau} \cdot \mathrm{e}\mkern3mu^{\tau \, t}\\
				\end{aligned}
			$} \span \\
			% 4. Zeile
			\left[...\right] = \frac{1}{\tau} \cdot \sin\left(\omega \cdot t \right) \cdot \mathrm{e}\mkern3mu^{\tau \, t} - \frac{\omega}{\tau}
			\left[
				\underbrace{\dbox{$\cos\left(\omega \cdot t \right)$}}_{u} \cdot
				\underbrace{\dbox{$\frac{1}{\tau} \cdot \mathrm{e}\mkern3mu^{\tau \, t}$}}_{v} - \int
				\underbrace{\dbox{$-\omega \cdot \sin\left(\omega \cdot t \right)$}}_{u\hskip2pt'} \cdot
				\underbrace{\dbox{$\frac{1}{\tau} \cdot \mathrm{e}\mkern3mu^{\tau \, t}$}}_{v}   \; \mathrm{d}t
			\right] \span\\
			% 5. Zeile
			\left[...\right] = \frac{1}{\tau} \cdot \sin\left(\omega \cdot t \right) \cdot \mathrm{e}\mkern3mu^{\tau \, t} - \frac{\omega}{\tau}
			\left[
				\cos\left(\omega \cdot t \right) \cdot
				\frac{1}{\tau} \cdot \mathrm{e}\mkern3mu^{\tau \, t} + \frac{\omega}{\tau} \cdot \int
				\sin\left(\omega \cdot t \right) \cdot
				\mathrm{e}\mkern3mu^{\tau \, t}  \; \mathrm{d}t
			\right] \span\\
			% 6. Zeile
			\underbrace{\int \sin \left( \omega \cdot t \right)   \cdot \mathrm{e}\mkern3mu^{\tau \, t} \; \mathrm{d}t}_{I}
			=
			\frac{1}{\tau} \cdot \sin\left(\omega \cdot t \right) \cdot \mathrm{e}\mkern3mu^{\tau \, t} - \frac{\omega}{\tau^{2}} \cdot \cos\left(\omega \cdot t \right) \cdot
			\mathrm{e}\mkern3mu^{\tau \, t} - \frac{\omega^{2}}{\tau^{2}} \cdot
			\underbrace{\int \sin \left( \omega \cdot t \right)   \cdot \mathrm{e}\mkern3mu^{\tau \, t} \; \mathrm{d}t}_{I} \span\\
			% 7. Zeile
			\underbrace{\int \sin \left( \omega \cdot t \right)   \cdot \mathrm{e}\mkern3mu^{\tau \, t} \; \mathrm{d}t}_{I} +
			\frac{\omega^{2}}{\tau^{2}} \cdot
			\underbrace{\int \sin \left( \omega \cdot t \right)   \cdot \mathrm{e}\mkern3mu^{\tau \, t} \; \mathrm{d}t}_{I}
			=
			\frac{1}{\tau} \cdot \sin\left(\omega \cdot t \right) \cdot \mathrm{e}\mkern3mu^{\tau \, t} - \frac{\omega}{\tau^{2}} \cdot \cos\left(\omega \cdot t \right) \cdot
			\mathrm{e}\mkern3mu^{\tau \, t} \span\\
			% 8. Zeile
			\left[ \int \sin \left( \omega \cdot t \right)   \cdot \mathrm{e}\mkern3mu^{\tau \, t} \; \mathrm{d}t \right] \left( 1 + \frac{\omega^{2}}{\tau^{2}} \right)
			=
			\frac{1}{\tau} \cdot \sin\left(\omega \cdot t \right) \cdot \mathrm{e}\mkern3mu^{\tau \, t} - \frac{\omega}{\tau^{2}} \cdot \cos\left(\omega \cdot t \right) \cdot
			\mathrm{e}\mkern3mu^{\tau \, t} \span\\
			% 9. Zeile
			\int \sin \left( \omega \cdot t \right)   \cdot \mathrm{e}\mkern3mu^{\tau \, t} \; \mathrm{d}t
			=
			\frac{1}{1 + \frac{\omega^{2}}{\tau^{2}}}
			\left( \frac{1}{\tau} \cdot \sin\left(\omega \cdot t \right)  - \frac{\omega}{\tau^{2}} \cdot 
			\cos\left(\omega \cdot t \right) \right) \cdot \mathrm{e}\mkern3mu^{\tau \, t} \span\\
			% 10. Zeile
			\int \sin \left( \omega \cdot t \right)   \cdot \mathrm{e}\mkern3mu^{\tau \, t} \; \mathrm{d}t
			=
			\frac{1}{\tau^{2} + \omega^{2}}
			\left( \tau \cdot \sin\left(\omega \cdot t \right) - \omega \cdot 
			\cos\left(\omega \cdot t \right) \right) \cdot \mathrm{e}\mkern3mu^{\tau \, t} \span\\
			% 11. Zeile
			\int \sin \left( \omega \cdot t \right)   \cdot \mathrm{e}\mkern3mu^{\frac{R}{L} \, t} \; \mathrm{d}t
			=
			\frac{1}{\frac{R^{2}}{L^{2}} + \omega^{2}}
			\left( \frac{R}{L} \cdot \sin\left(\omega \cdot t \right) - \omega \cdot 
			\cos\left(\omega \cdot t \right) \right) \cdot \mathrm{e}\mkern3mu^{\frac{R}{L} \, t} \span\\
			% 12. Zeile
			\int \sin \left( \omega \cdot t \right)   \cdot \mathrm{e}\mkern3mu^{\frac{R}{L} \, t} \; \mathrm{d}t
			=
			\frac{L}{R^{2} + L^{2} \cdot \omega^{2}}
			\left( R  \cdot \sin\left(\omega \cdot t \right) - L \omega \cdot 
			\cos\left(\omega \cdot t\right) \right) \cdot \mathrm{e}\mkern3mu^{\frac{R}{L} \, t} \span\\
		\end{align*}
		\vspace{-1.0cm}
	\end{mycapequ}
	
	\begin{mycapequ}[!ht]
		\begin{mdframed}[leftmargin=0cm,
			skipabove=1cm,    
			linecolor=black,
			backgroundcolor=gray!10,
			linewidth=0.75pt,
			innerleftmargin=0em,
			innerrightmargin=0em,
			innertopmargin=0.75em,
			innerbottommargin=0.5em,
			]
			\begin{addmargin}[1em]{2em}% 1em left, 2em right
				\textbf{Partielle Integration}
				\begin{align*}
					\int u\left(x\right) \cdot v\hskip2pt'\left(x\right) \; \mathrm{d}t  & = 
					u\left(x\right) \cdot v\left(x\right) - \int u\hskip2pt'\left(x\right) \cdot v\left(x\right)  \\
				\end{align*}
				\vspace{-1.25cm}
			\end{addmargin}
		\end{mdframed}
	\end{mycapequ}

	\newpage
	
	Einsetzen des gel�sten Integrals.
	\begin{mycapequ}[!ht]
		\vspace{-0.50cm}
		\begin{align*}
			% 1. Zeile
			C \left( t \right) & = \frac{u_{0}}{L} \cdot \int \sin \left( \omega \cdot t \right)   \cdot \mathrm{e}^{\frac{R}{L} \, \cdot \, t} \; \mathrm{d}t + K \\
			% 2. Zeile
			C \left( t \right) & = \frac{u_{0}}{L} \cdot \frac{L}{R^{2} + L^{2} \cdot \omega^{2}}
			\left( R  \cdot \sin\left(\omega \cdot t \right) - L \omega \cdot 
			\cos\left(\omega \cdot t \right) \right) \cdot \mathrm{e}\mkern3mu^{\frac{R}{L} \, t} + K \\
			% 2. Zeile
			C \left( t \right) & = \frac{u_{0}}{R^{2} + L^{2} \cdot \omega^{2}}
			\left( R  \cdot \sin\left(\omega \cdot t \right) - L \omega \cdot 
			\cos\left(\omega \cdot t \right) \right) \cdot \mathrm{e}\mkern3mu^{\frac{R}{L} \, t} + K \\
		\end{align*}
		\vspace{-1.75cm}
	\end{mycapequ}
	

	...die dann in den L�sungsansatz eingef�gt wird und erh�lt nun die \textit{allgemeine L�sung
	der inhomogenen DGL}.
	\begin{mycapequ}[!ht]
		\vspace{-0.50cm}
		\begin{align*}
			% 1. Zeile
			y & = \dbox{$C\left(x\right)$} \cdot \mathrm{e}^{-\int f\left(x\right)\; \mathrm{d}x} \\
			% 2. Zeile
			y & = \dbox{$\left(\int g\left(x\right) \cdot
			\mathrm{e}\mkern3mu^{\int f\left(x\right)\; \mathrm{d}x} \; \mathrm{d}x + K \right)$} \cdot \mathrm{e}^{-\int f\left(x\right)\; \mathrm{d}x} \\
			% Linie
			\cline{1-2}
			% 4. Zeile
			i \left( t \right) & = \underbrace{\dbox{$C\left(t\right)$} \cdot \mathrm{e}^{-\frac{R}{L} \, \cdot \, t}}_{i_{h}} \\
			% 5. Zeile
			i \left( t \right) & = \dbox{$\left( \frac{u_{0}}{R^{2} + L^{2} \cdot \omega^{2}}
			\left( R  \cdot \sin\left(\omega \cdot t \right) - L \omega \cdot 
			\cos\left(\omega \cdot t \right) \right) \cdot \mathrm{e}\mkern3mu^{\frac{R}{L} \, t} + K \right)$} \cdot \mathrm{e}^{-\frac{R}{L} \, \cdot \, t} \\
			% 6. Zeile
			i \left( t \right) & =
			K \cdot \mathrm{e}^{-\frac{R}{L} \, \cdot \, t} +
			\frac{R \cdot u_{0}}{R^{2} + L^{2} \cdot \omega^{2}} \cdot \sin\left(\omega \cdot t \right) -
			\frac{L \omega \cdot u_{0}}{R^{2} + L^{2} \cdot \omega^{2}} \cdot 
			\cos\left(\omega \cdot t \right) \\
		\end{align*}
		\vspace{-1.75cm}
	\end{mycapequ}

	Eine \textit{spezielle L�sung} wird berechnet mit dem Anfangswert $i\left(0\right) = 0$
	\begin{mycapequ}[!ht]
		\vspace{-0.50cm}
		\begin{align*}
		% 1. Zeile
		i \left( 0 \right) & =
		K \cdot \underbrace{\dbox{$\mathrm{e}^{-\frac{R}{L} \, \cdot \, t}$}}_{= 1} && +
		\frac{R \cdot u_{0}}{R^{2} + L^{2} \cdot \omega^{2}} \cdot  \underbrace{\dbox{$\sin\left(\omega \cdot t \right)$}}_{= 0} && -
		\frac{L \omega \cdot u_{0}}{R^{2} + L^{2} \cdot \omega^{2}} \cdot 
		 \underbrace{\dbox{$\cos\left(\omega \cdot t \right)$}}_{= 1} &= & \;\; 0\\
		 % 2. Zeile
		 i \left( 0 \right) & =
		 K && + 0 && -
		 \frac{L \omega \cdot u_{0}}{R^{2} + L^{2} \cdot \omega^{2}} &= & \;\; 0 \\
		 % 3. Zeile
		 &  &&  && \pushright{K} &= & \;\; \frac{L \omega \cdot u_{0}}{R^{2} + L^{2} \cdot \omega^{2}} \\
		 % 4. Zeile
		 i \left( t \right) & =
		 \frac{L \omega \cdot u_{0}}{R^{2} + L^{2} \cdot \omega^{2}} \cdot \mathrm{e}^{-\frac{R}{L} \, \cdot \, t} +
		 \frac{R \cdot u_{0}}{R^{2} + L^{2} \cdot \omega^{2}} \cdot \sin\left(\omega \cdot t \right) -
		 \frac{L \omega \cdot u_{0}}{R^{2} + L^{2} \cdot \omega^{2}} \cdot 
		 \cos\left(\omega \cdot t \right) \span\span\span\span\span\span \\
		 % 5. Zeile
		 i \left( t \right) & =
		 \frac{u_{0}}{R^{2} + L^{2} \cdot \omega^{2}} \cdot
		 \left(
		 	L \omega \cdot \mathrm{e}^{-\frac{R}{L} \, \cdot \, t} +
		 	R  \cdot \sin\left(\omega \cdot t \right) -
		 	L \omega \cdot \cos\left(\omega \cdot t \right)
		 \right) \span\span\span\span\span\span \\
		\end{align*}
		\vspace{-1.75cm}
	\end{mycapequ}
	

\end{document}