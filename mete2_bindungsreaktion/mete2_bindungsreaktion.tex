

%-------------------------------------------------------------------------------
% Dokumenten Klasse
\documentclass[
	final,
	a4paper,
	oneside,
	parskip=full,
	headings=standardclasses,
	headings=big,
	pointednumbers
]{scrartcl}

%-------------------------------------------------------------------------------
% Packete nutzen
\usepackage{ngerman,palatino,setspace}
\usepackage[T1]{fontenc}
\usepackage[latin9]{inputenc}
\usepackage[left=35mm,right=35mm,top=25mm,bottom=25mm]{geometry}
\usepackage{graphicx}
\usepackage{scrpage2}
\usepackage{listings}
\usepackage[usenames,dvipsnames,svgnames]{xcolor}
\usepackage[hidelinks]{hyperref}
\usepackage{amsmath}
\usepackage{caption}
\usepackage{mdframed}
\usepackage{dashbox}

\usepackage{multirow}% http://ctan.org/pkg/multirow
\usepackage{hhline}% http://ctan.org/pkg/hhline

%-------------------------------------------------------------------------------
% Hintergrundfarbe von Packet 'framed' mit Kommando 'shaded' nutzen
\colorlet{shadecolor}{gray!25}

%-------------------------------------------------------------------------------
% F�r "align" Umgebung eine Zelle nach links/rechzs verschieben
\makeatletter
\newcommand{\pushright}[1]{\ifmeasuring@#1\else\omit\hfill$\displaystyle#1$\fi\ignorespaces}
\newcommand{\pushleft}[1]{\ifmeasuring@#1\else\omit$\displaystyle#1$\hfill\fi\ignorespaces}
\makeatother

%-------------------------------------------------------------------------------
% Kopf- und Fusszeile

\pagestyle{scrheadings}
\clearscrheadfoot
\cfoot[Seite \thepage]{Seite \thepage}

%-------------------------------------------------------------------------------
%
\date{\today}

%-------------------------------------------------------------------------------
% Dokumenten Einstellungen

% Section Abst�nde
\RedeclareSectionCommand[
	beforeskip=-1\baselineskip,
	afterskip=0.001\baselineskip
]{section}
\RedeclareSectionCommand[
	beforeskip=0\baselineskip,
	afterskip=0.001\baselineskip
]{subsection}

% Formeln
\DeclareCaptionType{mycapequ}[aa][bb]
\captionsetup[mycapequ]{labelformat=empty}

\def\changemargin#1#2{\list{}{\rightmargin#2\leftmargin#1}\item[]}
\let\endchangemargin=\endlist 

%-------------------------------------------------------------------------------
% Listings
\lstset{
	language=Matlab,
	breaklines=true,
	numbers=left,
	numberstyle=\tiny,
	numbersep=5pt,
	captionpos=b,
	basicstyle=\footnotesize\ttfamily,
	stringstyle=\color{magenta},
	identifierstyle=\color{black},
	keywordstyle=\color{blue}, 
	commentstyle=\color{DarkGreen}
}

\newcommand{\mylisting}[2][]{%
	\lstinputlisting[caption={\texttt{\detokenize{#2}}},#1]{#2}%
}

%-------------------------------------------------------------------------------
% Dokument
\begin{document}
	
	\section*{L�sen der Bindungsreaktions-Differentialgleichung}
	Die DGL ist wie folgt:
	\begin{mycapequ}[!ht]
		\vspace{-0.5cm}
		\begin{alignat}{3}
			\frac{d}{dt}\Gamma\left(t\right) &= \frac{d}{dt}\Gamma_\textrm{on}\left(t\right) && - \frac{d}{dt}\Gamma_\textrm{off}\left(t\right) \\
			\frac{d}{dt}\Gamma\left(t\right) &= k_\textrm{on} \cdot c \cdot \left( \Gamma_\textrm{max} - \Gamma\left(t\right)\right) && - k_\textrm{off} \cdot \Gamma\left(t\right)
		\end{alignat}
		\vspace{-0.5cm}
	\end{mycapequ}

	\begin{mycapequ}[!ht]
		\begin{mdframed}[leftmargin=0cm,
			skipabove=1cm,    
			linecolor=black,
			backgroundcolor=gray!10,
			linewidth=0.75pt,
			innerleftmargin=0em,
			innerrightmargin=0em,
			innertopmargin=0.75em,
			innerbottommargin=0.5em,
			]
			\begin{addmargin}[1em]{2em}% 1em left, 2em right
				\textbf{Integration einer Differentialgleichung durch Substitution}
				
				\vspace{-0.50cm}
				\begin{align*}
					y\mkern3mu' & = f\left(a x + b y + c \right) \\
				\end{align*}
				\vspace{-1.00cm}
				
				Mit Hilfe einer \textit{geeignete Substitution} kann sie auf eine
				\textit{separable} Differentialgleichung 1. Ordnung zur�ckgef�hrt werden.
				
				\vspace{-0.50cm}
				\begin{align*}
					u  & = a \; x + b \; y + c \\
					u\mkern3mu' & = a + b \; y\mkern3mu' \\
					u\mkern3mu' - a & = b \; y' \\
					\frac{u\mkern3mu' - a}{b} & = y' \\
					\frac{u\mkern3mu' - a}{b} & = f \left( u \right)
				\end{align*}
				%\vspace{-1.00cm}
				
			\end{addmargin}
		\end{mdframed}
	\end{mycapequ}

	L�sen durch Substitution
	\begin{mycapequ}[!ht]
		\vspace{-0.5cm}
		\begin{alignat}{5}
		%% 1.
		\Gamma\mkern3mu' &= k_\textrm{on} \cdot c \cdot \left( \Gamma_\textrm{max} - \Gamma\right) - k_\textrm{off} \cdot \Gamma \span \span \span \span \label{eq:1}  \\
		%% 2
		u                & = c && + && \; b \; y && + a \; x\\
		%% 3
		u                & = c && + && \; b \; \Gamma && + a \; t\\
		%% 4
		u                &= k_\textrm{on} \cdot c \cdot \Gamma_\textrm{max} \; && - && \; k_\textrm{on} \cdot c \cdot \Gamma - k_\textrm{off} \cdot \Gamma \label{eq:4}\\
		%% 5
		u &= k_\textrm{on} \cdot c \cdot \Gamma_\textrm{max} \; && - && \; \left( k_\textrm{on} \cdot c \cdot + k_\textrm{off} \right) \cdot \Gamma \label{eq:5} \\
		%% 6
		u\mkern3mu' &=  &&- && \; \left( k_\textrm{on} \cdot c \cdot + k_\textrm{off} \right) \cdot \Gamma\mkern3mu' \\
		%% 7
		\frac{- u\mkern3mu'}{k_\textrm{on} \cdot c \cdot + k_\textrm{off}} &=  &&  && \; \Gamma\mkern3mu' \label{eq:7} 
		\end{alignat}
		\vspace{-0.5cm}
	\end{mycapequ}

	Gleichsetzen von \eqref{eq:7} mit \eqref{eq:1} respektive \eqref{eq:4}, Trennen der Variablen und Integrieren
	
	\begin{mycapequ}[!ht]
		\vspace{-0.5cm}
		\begin{alignat}{5}
		%% 1
		u &= \frac{- \dbox{$u\mkern3mu'$}}{k_\textrm{on} \cdot c \cdot + k_\textrm{off}} \\
		%% 2
		- \left(k_\textrm{on} \cdot c \cdot + k_\textrm{off} \right) \cdot u &= \dbox{$\frac{1}{dt} \cdot du$} \\
		%% 3
		- \left(k_\textrm{on} \cdot c \cdot + k_\textrm{off} \right) \cdot \int{1 \cdot dt} &= \int{\frac{1}{u} \cdot du} \\
		%% 4
		- \left(k_\textrm{on} \cdot c \cdot + k_\textrm{off} \right) \cdot t &= \ln \left(u\right) \\
		%% 5
		e^{- \left(k_\textrm{on} \cdot c \cdot + k_\textrm{off} \right) \cdot t} &= u
		\end{alignat}
		\vspace{-0.5cm}
	\end{mycapequ}

	\newpage
	
	R�cksubstituieren von $u$ mit \eqref{eq:5}
	
	\begin{mycapequ}[!ht]
		\vspace{-0.5cm}
		\begin{alignat}{5}
		%% 1
		e^{- \left(k_\textrm{on} \cdot c \cdot + k_\textrm{off} \right) \cdot t} &= u \\
		%% 2
		e^{- \left(k_\textrm{on} \cdot c \cdot + k_\textrm{off} \right) \cdot t} &= k_\textrm{on} \cdot c \cdot \Gamma_\textrm{max} \; && - && \; \left( k_\textrm{on} \cdot c \cdot + k_\textrm{off} \right) \cdot \Gamma \\
		%% 3
		\left( k_\textrm{on} \cdot c \cdot + k_\textrm{off} \right) \cdot \Gamma &= k_\textrm{on} \cdot c \cdot \Gamma_\textrm{max} \; && - && \; e^{- \left(k_\textrm{on} \cdot c \cdot + k_\textrm{off} \right) \cdot t} \\
		%% 4
		\Gamma \left(t\right)&= \frac{k_\textrm{on} \cdot c \cdot \Gamma_\textrm{max}}{k_\textrm{on} \cdot c + k_\textrm{off}} \; && - && \; \frac{e^{- \left(k_\textrm{on} \cdot c \cdot + k_\textrm{off} \right) \cdot t}}{k_\textrm{on} \cdot c + k_\textrm{off}} \\
		%% 5
		\Gamma \left(t\right)&= \Gamma_\textrm{max} \cdot \frac{c }{c + \frac{k_\textrm{off}}{k_\textrm{on}}} \; && - && \; \frac{e^{- \left(k_\textrm{on} \cdot c \cdot + k_\textrm{off} \right) \cdot t}}{k_\textrm{on} \cdot c \cdot + k_\textrm{off}}
		\end{alignat}
		\vspace{-0.5cm}
	\end{mycapequ}

	Laut meinem Taschenrechner (TI Voyage 200) und dem Befehl \\
	\texttt{deSolve(g'\,=\,c\,$\cdot$\,kon\,$\cdot$\,(m\,-\,g)\,-\,koff\,$\cdot$\,g and g(0)\,=\,0,\,t,\,g)} \\
	soll aber die L�sung wie folgt sein:

	\begin{mycapequ}[!ht]
		\vspace{-0.5cm}
		\begin{align*}
		%% 1
		\texttt{g} &= \frac{\texttt{kon} \cdot \texttt{c} \cdot \texttt{m}}{\texttt{kon} \cdot \texttt{c} + \texttt{koff}} - \frac{\texttt{kon} \cdot \texttt{c} \cdot \texttt{m} \cdot e^{- \left(\texttt{kon} \cdot \texttt{c} \cdot + \texttt{koff} \right) \cdot \texttt{t}}}{\texttt{kon} \cdot \texttt{c} + \texttt{koff}} \\
		%% 2
		\texttt{g} &= \frac{\texttt{kon} \cdot \texttt{c} \cdot \texttt{m}}{\texttt{kon} \cdot \texttt{c} + \texttt{koff}} \left( 1 - e^{- \left(\texttt{kon} \cdot \texttt{c} \cdot + \texttt{koff} \right) \cdot \texttt{t}}\right)
		\end{align*}
		\vspace{-0.5cm}
	\end{mycapequ}
	Hat sich eine Konstante, die beim Integrieren hinzugekommen ist, bei der Rechnung oben nicht ber�cksichtigt?

	Gleichgewicht bestimmen
	
	\begin{mycapequ}[!ht]
		\vspace{-0.5cm}
		\begin{align*}
		%% 1
		\frac{d}{dt}\Gamma\left(t\right) &= 0 \\
		%% 2
		 0 &= k_\textrm{on} \cdot c \cdot \left( \Gamma_\textrm{max} - \Gamma_{\textrm{Gleichgewicht}} \right) - k_\textrm{off} \cdot \Gamma_{\textrm{Gleichgewicht}} \\
		%% 3
		0 &= k_\textrm{on} \cdot c \cdot \Gamma_\textrm{max} - k_\textrm{on} \cdot c \cdot \Gamma_{\textrm{Gleichgewicht}} - k_\textrm{off} \cdot \Gamma_{\textrm{Gleichgewicht}} \\
		%% 4
		0 &= k_\textrm{on} \cdot c \cdot \Gamma_\textrm{max} - \left( k_\textrm{on} \cdot c \cdot + k_\textrm{off} \right) \cdot \Gamma_{\textrm{Gleichgewicht}} \\
		%% 5
		\left( k_\textrm{on} \cdot c \cdot + k_\textrm{off} \right) \cdot \Gamma_{\textrm{Gleichgewicht}} &= k_\textrm{on} \cdot c \cdot \Gamma_\textrm{max} \\
		%% 6
		\Gamma_{\textrm{Gleichgewicht}} &= \frac{k_\textrm{on} \cdot c \cdot \Gamma_\textrm{max}}{k_\textrm{on} \cdot c \cdot + k_\textrm{off}} \\
		%% 7
		\Gamma_{\textrm{Gleichgewicht}} &= \Gamma_\textrm{max} \cdot \frac{c }{c + \frac{k_\textrm{off}}{k_\textrm{on}}}
		\end{align*}
		\vspace{-0.5cm}
	\end{mycapequ}
	
	

\end{document}