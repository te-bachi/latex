

%-------------------------------------------------------------------------------
% Dokumenten Klasse
\documentclass[
	final,
	a4paper,
	oneside,
	parskip=full,
	headings=standardclasses,
	headings=big,
	pointednumbers
]{scrartcl}

%------------------------------------------------
% Packete nutzen
\usepackage[T1]{fontenc}
\usepackage[utf8]{inputenc}
\usepackage[ngerman]{babel}
\usepackage[left=20mm,right=20mm,top=10mm,bottom=25mm]{geometry}
\usepackage{amsmath}
\usepackage{amssymb}
\usepackage{mathtools}


%-------------------------------------------------
% UML
\usepackage{tikz}
\usepackage{ifthen}
\usepackage{xstring}
\usepackage{calc}
\usepackage{pgfopts}

\pgfdeclarelayer{background,foreground}
\pgfsetlayers{background,main,foreground}

\usepackage{tikz-uml}

%-------------------------------------------------
% METAPOST pictures\textbf{}
\usepackage[shellescape,latex]{gmp}

\gmpoptions{everymp={input expressg;}}
\usempxpackage[T1]{fontenc}
\usempxpackage[utf8]{inputenc}

%-------------------------------------------------
% enumerate
\usepackage{enumitem}
\setlist[enumerate]{
    wide=0pt,
    leftmargin=*,
    itemsep=-1ex,
    parsep=2ex,
    labelsep=1ex,
    label=\alph*)
}

% Einzug mit Dimension (z.B. 1cm)
\newenvironment{myindent}[1]{% 
    \parskip=6pt \parindent=0pt \raggedright 
    \def\lititem{\hangindent=#1 \hangafter=1}
}{%
    \par\ignorespaces
}


\begin{document}
    
    \section{Qt}

    \subsection{Meta-Objekt System}
    QObject ist die Basis-Klasse für alle Qt Objekte, die

    \begin{enumerate}[label=\arabic*)]
        \item{
            signals and slots mechanism
        }
        \item{
            inter-object communication
        }
        \item{
            run-time type information
        }
        \item{
            dynamic property system
        }
    \end{enumerate}

    Diese Mechanismen sind nicht direkt in C++ verfügbar oder werden
    anders gemacht wie Qt es möchte. Qt benutzt einen eigenen Precompiler,
    den Meta-Object Compiler (moc), um den spezifischen Qt-eingefärbten C++-Code
    in Standard C++-Code umzuwandeln damit der C++-Compiler ihn versteht.

    Schwere QObjects

    Dabei gibt es Einschränkungen, die in Standard C++ vorhanden sind

    \begin{enumerate}[label=\arabic*)]
        \item{
            Keine Mehrfachvererbung (Multiple inheritance)
        }
        \item{
            Keine Templates
        }
    \end{enumerate}

    \subsection{Model-View}

    Objekt Hierarchie Patient -- 1:n -- Timepoint -- 1:n -- Measurement.
    Keine QObjects. Leichtgewicht Klassen.
    

    Das Model-View Konzept von Qt 

    Das Model ist die Quelle der Information. Von ihn können verschiedene View
    abgebildet werden. So kann von einem einzigen Model zum Beispiel eine QComboBox
    und eine QListView angezeigt werden.

    Ein Model ist bei Qt wie eine 2-dimensionale Struktur mit Zeilen und Spalten.
    Ursprünglich wurde dies implementiert für eine tabellenkalkulatorische Weise (QTableView),
    wurde aber auf 1-dimensionale Views (QComboBox, QListView) adaptiert.

    Ein Problem besteht, dass diese Objekt Hierarchie über Model-Klassen in der View angezeigt werden.
    Automatisches einfügen einer Zeile im Model wenn 

    \newpage

    \begin{tikzpicture}
        \umlclass{namespace::A}{
            + n : uint
        }{
            + setA(i : int) : void
        }
    \end{tikzpicture}



    \begin{figure}
        \centering
        \begin{mpost}[name=1a]
z0 = (0,0);
z1 = (300,0);
z2 = (0,-75);

for xi = 0.025 step 0.05 until 1:
	pickup pencircle scaled 0.5pt;
	draw z0--(300*(xi,xi * (xi - 1)))--z1 withcolor (1,0.5,0.5);
	pickup pencircle scaled 2pt;
	draw 300*(xi,xi * (xi - 1)) withcolor (1,0.5,0.5);
endfor;

pickup pencircle scaled 1pt;

drawarrow (-0.03[z0,z1])--(1.03[z0,z1]);
drawarrow ((-0.1)[z2,z0])--(1.1[z2,z0]);

draw (z0 shifted (0,2))--(z0 shifted (0,-2));
draw (z1 shifted (0,2))--(z1 shifted (0,-2));

label.llft(btex $0$ etex, z0);
label.top(btex $G(x,\xi)$ etex, 1.1[z2,z0]);
label.lrt(btex $1$ etex, z1);
label.urt(btex $x$ etex, 1.03[z0,z1]);

pickup pencircle scaled 1pt;
draw z0--(300*(0.325,0.325*(0.325-1)))--z1 withcolor (1,0,0);

pickup pencircle scaled 0.2pt;
draw (300*(0.325,0))--(300*(0.325,-0.325*(1-0.325)));
label.top(btex $\xi$ etex, 300*(0.325,0));
pickup pencircle scaled 2pt;
draw 300*(0.325,0);
draw 300*(0.325,0.325*(0.325-1)) withcolor (1,0,0);
        \end{mpost}
        \caption{A class diagram}
    \end{figure}

\begin{figure}
        \centering
\begin{mpost}
z0 = origin;
drawroundedbox(0, 2.5cm, 3cm, 5mm)(
  \btex {\begin{tabular}{@{}c@{}} a b c \end{tabular}} etex
);
\end{mpost}
        \caption{A class diagram}
    \end{figure}


\end{document}