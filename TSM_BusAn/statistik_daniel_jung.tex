

%-------------------------------------------------------------------------------
% Dokumenten Klasse
\documentclass[
	final,
	a4paper,
	oneside,
	parskip=full,
	headings=standardclasses,
	headings=big,
	pointednumbers
]{scrartcl}

%-------------------------------------------------------------------------------
% Packete nutzen
\usepackage[T1]{fontenc}
\usepackage[utf8]{inputenc}
\usepackage{ngerman}
\usepackage[left=15mm,right=20mm,top=17mm,bottom=17mm,footskip=0.7cm]{geometry}
\usepackage{amsmath}
\usepackage{amssymb}
\usepackage{mathtools}
\usepackage{mathtools}
\usepackage{bm}


%-------------------------------------------------------------------------------
% xcolor
\usepackage[svgnames,table]{xcolor}

%-------------------------------------------------------------------------------
% scrlayer-scrpage
\usepackage{scrlayer-scrpage}
\pagestyle{scrheadings}
\clearpairofpagestyles

\cfoot{\pagemark}

%-------------------------------------------------------------------------------
% array
\usepackage{array}
\newcolumntype{A}[1]{>{\raggedright\let\newline\\\arraybackslash\hspace{0pt}}p{#1}}
\newcolumntype{B}[1]{>{\centering\let\newline\\\arraybackslash\hspace{0pt}}p{#1}}
\newcolumntype{C}[1]{>{\raggedleft\let\newline\\\arraybackslash\hspace{0pt}}p{#1}}

\newcolumntype{D}[1]{>{\raggedright\let\newline\\\arraybackslash\hspace{0pt}}m{#1}}
\newcolumntype{E}[1]{>{\centering\let\newline\\\arraybackslash\hspace{0pt}}m{#1}}
\newcolumntype{F}[1]{>{\raggedleft\let\newline\\\arraybackslash\hspace{0pt}}m{#1}}

\newcolumntype{G}[1]{>{\raggedright\let\newline\\\arraybackslash\hspace{0pt}}b{#1}}
\newcolumntype{H}[1]{>{\centering\let\newline\\\arraybackslash\hspace{0pt}}b{#1}}
\newcolumntype{I}[1]{>{\raggedleft\let\newline\\\arraybackslash\hspace{0pt}}b{#1}}


%-------------------------------------------------------------------------------
% tabularx
\usepackage{tabularx}
\usepackage{ltablex}
\usepackage{makecell}

%-------------------------------------------------------------------------------
% TikZ
\usepackage{tikz}
\usetikzlibrary{positioning, arrows, decorations, calc, fit, intersections}

%-------------------------------------------------------------------------------
% Listings
\usepackage{listings}
\newcommand{\listingMatlab}[2][]{
	\lstset{
		language=Matlab,
		breaklines=true,
		numbers=left,
		numberstyle=\tiny,
		numbersep=5pt,
		captionpos=b,
		basicstyle=\footnotesize\ttfamily,
		stringstyle=\color{magenta},
		identifierstyle=\color{black},
		keywordstyle=\color{blue}, 
		commentstyle=\color{DarkGreen}
	}
	\lstinputlisting[caption={\texttt{\detokenize{#2}}},#1]{#2}
}
\lstnewenvironment{algorithm}[1][] %defines the algorithm listing environment
{
    \lstset{ %this is the stype
        mathescape=true,
        frame=tB,
        numbers=left, 
        numberstyle=\tiny,
        basicstyle=\scriptsize, 
        keywordstyle=\color{black}\bfseries,
        keywords={,input, output, return, datatype, function, in, if, else, foreach, while, begin, end, } 
        numbers=left,
        xleftmargin=.04\textwidth
    }
}
{}


%-------------------------------------------------------------------------------

\RedeclareSectionCommand[
  afterindent=false,
  beforeskip=0.8\baselineskip,
  afterskip=0.4\baselineskip]{section}
\RedeclareSectionCommand[
  afterindent=false,
  beforeskip=0.8\baselineskip,
  afterskip=0.4\baselineskip]{subsection}

%-------------------------------------------------------------------------------
% 

\newcommand{\colouredcircle}{%
    \tikz{
        \useasboundingbox (-0.2em,-0.32em) rectangle(0.2em,0.32em);
        \draw[line width=0.03em] (0,0) circle(0.10em);
    }}



%-------------------------------------------------------------------------------
% enumitem
\usepackage{enumitem}
\newlist{tabenum}{enumerate}{3}
\setlist[tabenum,1]{
    leftmargin=*,
    label=\protect\colouredcircle,
    topsep=0ex,
    partopsep=0ex,
    noitemsep
}
\setlist[tabenum,2]{
    leftmargin=*,
    label=\protect\colouredcircle,
    topsep=0ex,
    partopsep=0ex,
    noitemsep
}
\setlist[tabenum,3]{
    leftmargin=*,
    label=\protect\colouredcircle,
    topsep=0ex,
    partopsep=0ex,
    noitemsep
}

\newcommand{\f}[2]{\frac{#1}{#2}}
\newcommand{\fs}[2]{{\tfrac{#1}{#2}}}

% kl = ()
\newcommand{\kl}[1]{{\left( #1 \right)}}

% kq = {}
\newcommand{\kq}[1]{{\left\{ #1 \right\}}}

% ks = []
\newcommand{\ks}[1]{{\left[ #1 \right]}}

% 
\newcommand{\abs}[1]{{\vert #1 \vert}}

\begin{document}
    \section{Stochastik Übersicht}

    \begin{tabularx}{\textwidth}{lll}
        Wahrscheinlichkeitsrechnung & Beschreibende Statistik & Beurteilende Statistik
    \end{tabularx}

    \colouredcircle

    \begin{tabular}{A{5cm}|A{5cm}|A{5cm}}
        Wahrscheinlichkeitsrechnung & Beschreibende Statistik & Beurteilende Statistik \\
        \hline
       \begin{tabenum}
            \item{Kombinatorik}
            \begin{tabenum}
                \item{Abzählverfahren}
            \end{tabenum}
            \item{Urnenmodell}
            \item{Diskrete Verteilung}
            \begin{tabenum}
                \item{Baumdiagramm}
                \item{Pfadregel}
                \item{Was ist eine Wahrscheinlichkeit}
            \end{tabenum}
            \item{Gesetz grosser Zahlen}
            \item{Normalverteilung}
            \begin{tabenum}
                \item{stetig}
            \end{tabenum}
            \item{Binomialverteilung}
            \item{Grenzwertsätze}
        \end{tabenum} &
        \begin{tabenum}
            \item{Häufigkeitsverteilung}
            \begin{tabenum}
                \item{1 oder 2 Merkmale}
            \end{tabenum}
            \item{Statistische Abhängigkeit}
            \begin{tabenum}
                \item{Korrelation}
                \begin{tabenum}
                    \item{Korrelationskoeffizient}
                \end{tabenum}
                \item{Regression}
                \begin{tabenum}
                    \item{Regressionsgerade}
                \end{tabenum}
            \end{tabenum}
            \item{Mit Daten arbeiten}
            \item{Viele Datenerhebungen}
        \end{tabenum} &
        \begin{tabenum}
            \item{Tests, Thesen}
            \begin{tabenum}
                \item{Nachweisen, Widerlegen}
            \end{tabenum}
            \item{Hypothesentest}
            \item{Alternativtest}

            \item{$\text{Chi}^2$}
        \end{tabenum}
    \end{tabular}


    \subsection{Zufallsexperiment}
    
    \begin{tabenum}
        \item{Wird unter genau festgelegten Versuchsbedingungen durchgeführt}
        \item{Hat einen zufälligen Ausgang}
        \item{Versteht man einen Vorgang, bei dem}
        \begin{tabenum}
            \item{mehrere Ergebnisse eintreten können}
            \item{bei dem ein nicht vorhersagbares, erfassbares Ergebnis eintritt}
            \item{zum Beispiel das Werfen einer Münze oder eines Spielwürfels}
        \end{tabenum}
        \item{Obwohl das Ergebnis jedes einzelnen Versuchs zufällig ist, lassen sich}
        \begin{tabenum}
            \item{Gesetzmäßigkeiten erkennen, die mathematisch erfasst werden können}
            \item{Die interessierenden Größen eines Zufallsexperiments nennt man Zufallsvariablen}
        \end{tabenum}
    \end{tabenum}

    \begin{tabenum}
        \item{Ein Laplace-Experiment ist ein Zufallsexperiment, bei dem die unterschiedlichen Elementarereignisse alle gleich wahrscheinlich sind}
    \end{tabenum}


    \begin{tabular}{cll}
        $\Omega$                    & Ergebnisraum, Ergebnismenge   & $\Omega = \kq{ \; 1, \; 2, \; 3, \; 4, \; 5, \; 6 \; }$ \\
        $\omega \in \Omega$         & Ergebnis, Elementarergebnis   & $\omega_1 = 1, \; \omega_2 = 2, \; \ldots, \; \omega_6 = 6$ \\
        $\Sigma$                    & Ereignisraum                  & $\Sigma = \kq{ \; \varnothing,
                                                                                     \; \kq{1}, \; \kq{2}, \; \kq{3},
                                                                                     \; \kq{1, \; 2}, \; \ldots, \; \kq{\; 2, \; 4, \; 6 \; }, \; \ldots,
                                                                                     \; \Omega }$ \\
        $ A,E \in \Sigma$           & Ereignis                      & $A = \kq{\; 2, \; 4, \; 6 \; }$ \\
        $ \kl{\Omega, \Sigma, P}$   & Wahrscheinlichkeitsraum       & \\
    \end{tabular}

    Besondere Ereignisse

    \begin{tabular}{lll}
        Unmögliches Ereignis        & ${E:} \text{Augenzahl grösser als 6}$ & $E = \kq{ \; }$ \\
        Elementarereignis           & $E: \text{Augenzahl grösser als 6}$ & $\omega_1 = 1, \; \omega_2 = 2, \; \ldots, \; \omega_6 = 6$ \\
        Zusammengesetztes Ereignis  & $E: \text{Augenzahl grösser als 6}$ & \\
        Sicheres Ereignis           & $E: \text{Augenzahl grösser als 6}$ & $A = \kq{\; 2, \; 4, \; 6 \; }$ \\
    \end{tabular}
    

    \subsection{Zufallsvariable, Zufallsgrösse}

    \begin{tabenum}
        \item{Größe, deren Wert vom Zufall abhängig ist}
        \item{Zuordnungsvorschrift, die jedem möglichen Ergebnis eines Zufallsexperiments eine Größe zuordnet}
        \item{Versteht man einen Vorgang, bei dem}
        \begin{tabenum}
            \item{mehrere Ergebnisse eintreten können}
            \item{bei dem ein nicht vorhersagbares, erfassbares Ergebnis eintritt}
            \item{zum Beispiel das Werfen einer Münze oder eines Spielwürfels}
        \end{tabenum}
        \item{Obwohl das Ergebnis jedes einzelnen Versuchs zufällig ist, lassen sich}
        \begin{tabenum}
            \item{Gesetzmäßigkeiten erkennen, die mathematisch erfasst werden können}
            \item{Die interessierenden Größen eines Zufallsexperiments nennt man Zufallsvariablen}
        \end{tabenum}
    \end{tabenum}

    \begin{tabular}{ll}
        \textbf{Funktion} $\bm{f}$  & \textbf{Zufallsvariable} $\bm{X}$ \\
        $f: D \rightarrow W $       & $X: \Omega \rightarrow \mathbb{R}$ \\
        $f: x \mapsto y $           & $X: \omega \mapsto x $ \\
        $y = f\kl{x} $              & $x = X\kl{\omega}$ \\
    \end{tabular}


    \begin{tabular}{cll}
        $X$             & Zufallsvariable                   & (= Funktion) \\
        $\Omega$        & Ergebnisraum                      & (= Definitionsmenge) \\
        $\mathbb{R}$    & Menge der reellen Zahlen          & (= Wertemenge) \\
        $\omega$        & Ergebnis eines Zufallsexperiments & (= Element der Definitionsmenge) \\
        $x$             & Reelle Zahl                       & (= Element der Wertemenge) \\
        $X\kl{\omega}$  & Zuordnungsvorschrift              & (= Funktionsgleichung) \\
    \end{tabular}

    \section{Statistisches Testverfahren}

    \begin{tabenum}
        \item{Mit einem Gerichtsverfahren vergleichbar}
        \item{Verfahren hat den Zweck festzustellen}
        \begin{tabenum}
            \item{Ob es ausreichende Beweise gibt, den Angeklagten zu verurteilen}
            \item{Immer von der Unschuld des Verdächtigten ausgegangen}
            \item{Solange grosse Zweifel an Belegung für Vergehen bestehen, wird Angeklagter freigesprochen}
            \item{Nur wenn Indizien für die Schuld des Angeklagten deutlich überwiegen, kommt es zur Verurteilung}
        \end{tabenum}
    \end{tabenum}

    \begin{tabular}{lll}
        Hyphothese $H_0$ & Nullhypothese       & Verdächtige ist unschuldig \\
        Hyphothese $H_1$ & Alternativhypothese & Verdächtige ist schuldig \\
        Fehler 1. Art    &                     & Verurteilung eines Unschuldigen \\
        Fehler 2. Art    &                     & Freispruch eines Schuldigen \\
    \end{tabular}
    
\end{document}