%!TEX root = ../bericht.tex

%%% PACKAGE %%%%%%%%%%%%%%%%%%%%%%%%%%%%%%%%%%%%%%%%%%%%%%%%%%%%%%%%%%%%%%%%%%%%
% , numbers=enddot
\documentclass[
	a4paper,
	12pt,
	twoside,
	headings=standardclasses,
	headings=big,
	bibliography=totoc,
	index=totoc
]{scrartcl}
\usepackage[utf8x]{inputenc}
\usepackage[ngerman]{babel}
\usepackage[T1]{fontenc}



\usepackage[inner=3cm,outer=2.5cm,top=2.7cm,bottom=3.2cm]{geometry}


%%% GRAFIK %%%%%%%%%%%%%%%%%%%%%%%%%%5%%%%%%%%%%%%%%%%%%%%%%%%%%%%%%%%%%%%%%%%%%
\usepackage{graphicx}
\usepackage{pdfpages}
\usepackage{tikz}
\usepackage{caption}

\usetikzlibrary{calc,patterns,decorations.pathmorphing,decorations.markings}

%%% EINHEITEN %%%%%%%%%%%%%%%%%%%%%%%5%%%%%%%%%%%%%%%%%%%%%%%%%%%%%%%%%%%%%%%%%%
\usepackage[
	separate-uncertainty  = true,
	output-decimal-marker ={.},
	repeatunits           = false,
	range-phrase          = {\,bis\,},
]{siunitx}

%%% GLOSSAR UND ABKÜRZUNGEN %%%%%%%%%%%%%%%%%%%%%%%%%%%%%%%%%%%%%%%%%%%%%%%%%%%%
%\usepackage[
%	nonumberlist,	% keine Seitenzahlen
%	acronym,		% ermoeglicht Abkuerzungen+Abkuerzungsverz.
%	xindy
%]{glossaries}		% ermoeglicht Glossar

%\usepackage{translator}

%\GlsSetXdyCodePage{duden-utf8}

%\makeglossaries
%\setglossarystyle{meinglossar}
%\loadglsentries{index/index.tex}

%%% SCHRIFTART %%%%%%%%%%%%%%%%%%%%%%%%%%%%%%%%%%%%%%%%%%%%%%%%%%%%%%%%%%%%%%%%%

%%% Computer Modern Bright
\usepackage{cmbright}

%%% Latin Modern
% \usepackage{lmodern}

%%% ABSATZ EINSTELLUNGEN %%%%%%%%%%%%%%%%%%%%%%%%%%%%%%%%%%%%%%%%%%%%%%%%%%%%%%%
% Einzug
\setlength{\parindent}{0mm}

% Abstand zwischen Absätzen
\setlength{\parskip}{2mm}

%%% TITEL UND AUTOR %%%%%%%%%%%%%%%%%%%%%%%%%%%%%%%%%%%%%%%%%%%%%%%%%%%%%%%%%%%%
\title{Numerische Methoden Projekt}
\author{Andreas Bachmann}
\date{25.02.2018}


%%% KOPF- UND FUSSZEILE %%%%%%%%%%%%%%%%%%%%%%%%%%%%%%%%%%%%%%%%%%%%%%%%%%%%%%%%
\usepackage{scrlayer-scrpage}
%\usepackage{titling}

\pagestyle{scrheadings}
\clearscrheadfoot

\ifoot{Andreas Bachmann}
\ofoot{\pagemark}
%\ihead{\headmark}
\ohead{Numerische Methoden Projekt}
\automark[section]{section}


%%% BILDER %%%%%%%%%%%%%%%%%%%%%%%%%%%%%%%%%%%%%%%%%%%%%%%%%%%%%%%%%%%%%%%%%%%%%
\newenvironment{Figure}
{\par\medskip\noindent\minipage{\linewidth}}
{\endminipage\par\medskip}

\renewcommand{\figurename}{Abbildung}
\captionsetup[figure]{skip=1.5mm}


%%% FORMELN %%%%%%%%%%%%%%%%%%%%%%%%%%%%%%%%%%%%%%%%%%%%%%%%%%%%%%%%%%%%%%%%%%%
\usepackage{amsmath,amssymb}

\DeclareCaptionType{mycapequ}[aa][bb]
\captionsetup[mycapequ]{labelformat=empty}

%%% KAPITEL %%%%%%%%%%%%%%%%%%%%%%%%%%%%%%%%%%%%%%%%%%%%%%%%%%%%%%%%%%%%%%%%%%%%

%%% JEDE SECTION AUF NEUER SEITE

% New
\addtokomafont{section}{\clearpage}

% Old
%\usepackage[compact]{titlesec}
%\newcommand{\sectionbreak}{\clearpage}

%%% ABSTÄNDE ZWISCHEN TITEL UND TEST

%
%\titlespacing*{\section}
%{0pt}{1.0ex plus 1ex minus .2ex}{0.0ex plus .2ex}
%\titlespacing*{\subsection}
%{0pt}{1.0ex plus 1ex minus .2ex}{0.0ex plus .2ex}
%\titlespacing*{\subsubsection}
%{0pt}{0.5ex plus 1ex minus .2ex}{0.0ex plus .2ex}


%%% BOX AROUND CHARACTER %%%%%%%%%%%%%%%%%%%%%%%%%%%%%%%%%%%%%%%%%%%%%%%%%%%%%%%

\def\dolist{\afterassignment\dodolist\let\next= }

\def\dodolist{\ifx\next\endlist \let\next\relax
	\else \=\let\next\dolist \fi
	\next}
\def\endlist{\endlist}

\def\hidehrule#1#2{\kern-#1%
	\hrule height#1 depth#2 \kern-#2 }

\def\hidevrule#1#2{\kern-#1{\dimen0=#1
		\advance\dimen0 by#2\vrule width\dimen0}\kern-#2 }

\def\makeblankbox#1#2{\hbox{\lower\dp0\vbox{\hidehrule{#1}{#2}%
	\kern-#1 % overlap the rules at the corners
	\hbox to \wd0{\hidevrule{#1}{#2}%
		\raise\ht0\vbox to #1{}% set the vrule height
		\lower\dp0\vtop to #1{}% set the vrule depth
		\hfil\hidevrule{#2}{#1}}%
	\kern-#1\hidehrule{#2}{#1}}}}

\def\maketypebox{\makeblankbox{0pt}{1pt}}

\def\makelightbox{\makeblankbox{.2pt}{.2pt}}

\def\={\expandafter\if\space\next\
	\else \setbox0=\hbox{\next}\maketypebox\fi}

\def\demobox#1{\setbox0=\hbox{\dolist#1\endlist}%
	\copy0\kern-\wd0\makelightbox}

%%

\def\boxit#1{%
	\fboxsep=0pt%
	\fboxrule=.1pt%
	\@tfor\xx:=#1\do{%
		\fbox{\phantom{\xx}}%
	} #1%
}

