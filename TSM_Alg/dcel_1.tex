

%-------------------------------------------------------------------------------
% Dokumenten Klasse
\documentclass[
	final,
	a4paper,
	oneside,
	parskip=full,
	headings=standardclasses,
	headings=big,
	pointednumbers
]{scrartcl}

%-------------------------------------------------------------------------------
% Packete nutzen
\usepackage[T1]{fontenc}
\usepackage[utf8]{inputenc}
\usepackage[left=15mm,right=20mm,top=17mm,bottom=17mm,footskip=0.7cm]{geometry}
\usepackage{amsmath}
\usepackage{amssymb}
\usepackage{mathtools}
\usepackage{mathtools}

%-------------------------------------------------------------------------------
% xcolor
\usepackage[svgnames]{xcolor}

%-------------------------------------------------------------------------------
% scrlayer-scrpage
\usepackage{scrlayer-scrpage}
\pagestyle{scrheadings}
\clearpairofpagestyles

\cfoot{\pagemark}

%-------------------------------------------------------------------------------
% tabularx
\usepackage{tabularx}
\usepackage{ltablex}
\usepackage{makecell}

%-------------------------------------------------------------------------------
% TikZ
\usepackage{tikz}
\usetikzlibrary{positioning, arrows, decorations, calc, fit, intersections}

%-------------------------------------------------------------------------------
% 
\newcommand{\f}[2]{\frac{#1}{#2}}
\newcommand{\fs}[2]{{\tfrac{#1}{#2}}}

% kl = ()
\newcommand{\kl}[1]{{\left( #1 \right)}}

% kq = {}
\newcommand{\kq}[1]{{\left\{ #1 \right\}}}

% ks = []
\newcommand{\ks}[1]{{\left[ #1 \right]}}

% 
\newcommand{\abs}[1]{{\vert #1 \vert}}

% Syntax:
% \DoublLine[half of the double line distance]{first node}{second node}{options line 1}{label line 1}{options line 2}{label line 2}
\newcommand\DoubleLine[7][4pt]{%
    \path(#2)--(#3)coordinate[at start](h1)coordinate[at end](h2);
    \draw[#4]($(h1)!#1!90:(h2)$)-- node [auto=left] {#5} ($(h2)!#1!-90:(h1)$); 
    \draw[#6]($(h1)!#1!-90:(h2)$)-- node [auto=right] {#7} ($(h2)!#1!90:(h1)$);
    }


\def\pich{6.5cm}
\def\picb{9cm}

\begin{document}

    \begin{tabularx}{\textwidth}{ll}
        % === PAGE 1 ========================================================
        % --- IMAGE 1 -------------------------------------------------------
        \noindent\parbox[c][\pich]{\picb}{
        \begin{tikzpicture}[myn/.style={circle,very thick,draw,inner sep=0.25cm,outer sep=3pt}]
        
            \node[myn] (s) at (0,2) {s};
            \node[myn] (a) at (2,4) {a};
            \node[myn] (b) at (2,0) {b};
            \node[myn] (c) at (5,4) {c};
            \node[myn] (d) at (5,0) {d};
            \node[myn] (t) at (7,2) {t};
            
            \DoubleLine{s}{a}{<-,very thick,black}{1}{->,very thick,red}{7}
            \DoubleLine{s}{b}{<-,very thick,black}{2}{->,very thick,red}{6}
            \DoubleLine{a}{b}{<-,very thick,black}{3}{->,very thick,red}{5}
            \DoubleLine{a}{c}{<-,very thick,black}{4}{->,very thick,red}{4}
            \DoubleLine{b}{d}{<-,very thick,black}{5}{->,very thick,red}{3}
            \DoubleLine{c}{t}{<-,very thick,black}{6}{->,very thick,red}{2}
            \DoubleLine{d}{t}{<-,very thick,black}{7}{->,very thick,red}{1}
        
        \end{tikzpicture}} &
        \makecell[l]{$\!\begin{aligned}[t]
            s_i     &= \text{Line segment} \\
            p_i     &= \text{Upper endpoint} \\
            q_i     &= \text{Lower endpoint} \\
            r_{ij}  &= \text{Intersection point} \\
            Q       &= \kq{ p_1, p_2, p_3, p_4, p_5, q_2, q_4, q_3, p_6, q_6, q_5, q_1 } \\
            S       &= \kq{\;}
        \end{aligned}
        $}
   \end{tabularx}
\end{document}