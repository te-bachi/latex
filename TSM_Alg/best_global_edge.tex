

%-------------------------------------------------------------------------------
% Dokumenten Klasse
\documentclass[
	final,
	a4paper,
	oneside,
	parskip=full,
	headings=standardclasses,
	headings=big,
	pointednumbers
]{scrartcl}

%-------------------------------------------------------------------------------
% Packete nutzen
\usepackage[T1]{fontenc}
\usepackage[utf8]{inputenc}
\usepackage[left=15mm,right=20mm,top=17mm,bottom=17mm,footskip=0.7cm]{geometry}
\usepackage{amsmath}
\usepackage{amssymb}
\usepackage{mathtools}
\usepackage{mathtools}

%-------------------------------------------------------------------------------
% xcolor
\usepackage[svgnames]{xcolor}

%-------------------------------------------------------------------------------
% scrlayer-scrpage
\usepackage{scrlayer-scrpage}
\pagestyle{scrheadings}
\clearpairofpagestyles

\cfoot{\pagemark}

%-------------------------------------------------------------------------------
% tabularx
\usepackage{tabularx}
\usepackage{ltablex}
\usepackage{makecell}

%-------------------------------------------------------------------------------
% TikZ
\usepackage{tikz}
\usetikzlibrary{positioning, arrows, decorations, calc, fit, intersections}

%-------------------------------------------------------------------------------
% Listings
\usepackage{listings}
\newcommand{\listingMatlab}[2][]{
	\lstset{
		language=Matlab,
		breaklines=true,
		numbers=left,
		numberstyle=\tiny,
		numbersep=5pt,
		captionpos=b,
		basicstyle=\footnotesize\ttfamily,
		stringstyle=\color{magenta},
		identifierstyle=\color{black},
		keywordstyle=\color{blue}, 
		commentstyle=\color{DarkGreen}
	}
	\lstinputlisting[caption={\texttt{\detokenize{#2}}},#1]{#2}
}
\lstnewenvironment{algorithm}[1][] %defines the algorithm listing environment
{
    \lstset{ %this is the stype
        mathescape=true,
        frame=tB,
        numbers=left, 
        numberstyle=\tiny,
        basicstyle=\scriptsize, 
        keywordstyle=\color{black}\bfseries,
        keywords={,input, output, return, datatype, function, in, if, else, foreach, while, begin, end, } 
        numbers=left,
        xleftmargin=.04\textwidth
    }
}
{}

%-------------------------------------------------------------------------------
% 
\newcommand{\f}[2]{\frac{#1}{#2}}
\newcommand{\fs}[2]{{\tfrac{#1}{#2}}}

% kl = ()
\newcommand{\kl}[1]{{\left( #1 \right)}}

% kq = {}
\newcommand{\kq}[1]{{\left\{ #1 \right\}}}

% ks = []
\newcommand{\ks}[1]{{\left[ #1 \right]}}

% 
\newcommand{\abs}[1]{{\vert #1 \vert}}

\newcommand\addvmargin[1]{
  \node[fit=(current bounding box),inner ysep=#1,inner xsep=0]{};
}

\def\tiktop{%
    \def\xlines{25}
    \def\ylines{15}
    \def\raster{3.5mm}

    \coordinate (P1)  at ( 4*\raster,13*\raster);
    \coordinate (P2)  at ( 5*\raster, 9*\raster);
    \coordinate (P3)  at ( 3*\raster, 7*\raster);
    \coordinate (P4)  at ( 2*\raster, 3*\raster);
    \coordinate (P5)  at ( 8*\raster, 1*\raster);
    \coordinate (P6)  at (10*\raster,13*\raster);
    \coordinate (P7)  at (11*\raster, 4*\raster);
    \coordinate (P8)  at (14*\raster,10*\raster);
    \coordinate (P9)  at (16*\raster, 6*\raster);
    \coordinate (P10) at (19*\raster, 5*\raster);
    \coordinate (P11) at (22*\raster, 12*\raster);
    \coordinate (P12) at (23*\raster, 7*\raster);

    % draw vertical lines
    \foreach \x in {0,...,\xlines}
    {
        \draw[lightgray] (\x * \raster, 0mm) -- (\x * \raster, \ylines * \raster);
    }

    % draw horizontal lines
    \foreach \y in {0,...,\ylines}
    {
        \draw[lightgray] (0mm, \y * \raster) -- (\xlines * \raster, \y * \raster);
    }

    % border
    \draw[black] (0mm, 0mm) -- (\xlines * \raster, 0mm);
    \draw[black] (0mm, 0mm) -- (0mm, \ylines * \raster);
    \draw[black] (\xlines * \raster, 0mm) -- (\xlines * \raster, \ylines * \raster);
    \draw[black] (0mm, \ylines * \raster) -- (\xlines * \raster, \ylines * \raster);
}


\newcommand{\dashedlines}[1]{%
    \foreach \point in {#1}
    {
        \draw[black,line width=0.2mm, densely dashed]
            let
                \p1 = (\point)
            in
                (\x1, \y1) -- (\x1, 0);
    }
}

\newcommand{\points}[1]{%
    \foreach \point/\co in {#1}
    {
        \draw[\co,fill=\co,line width=0.4mm]
            (\point) circle (0.5mm);
    }
}

\newcommand{\events}[1]{%
    \foreach \event/\name/\i in {#1}
    {
        \path
            let
                \p1 = (\event)
            in 
                node at (\x1,0) [below, fill=white, inner sep=1.25, outer sep=5] {$\scriptstyle \name_{\i} $};
    }
}

\newcommand{\status}[1]{%
    \foreach \status/\i in {#1}
    {
        \path
            let
                \p1 = (\status)
            in 
                node at (0,\y1) [left, fill=white, inner sep=1.25, outer sep=3.5] {$\scriptstyle s_{\i} $};
    }
}

\newcommand{\pointtexts}[1]{%
    \foreach \point/\name/\i/\align/\co in {#1}
    {
        \node[\align, \co, fill=white, inner sep=1.25, outer sep=2] at (\point) {$\scriptstyle \name_{\i} $};
    }
}

\newcommand{\sweepline}[1]{%
    \foreach \event in {#1}
    {
        \draw[red,line width=0.4mm]
            let
                \p1 = (\event)
            in 
                (\x1, 0) -- (\x1, \ylines * \raster);
    }
}

\newcommand{\intersection}[1]{%
    \foreach \intersect/\co/\dashline in {#1}{
        \ifnum\pdfstrcmp{\dashline}{yes}=0
        \draw[\co,line width=0.4mm, densely dashed]
                let
                    \p1 = (\intersect)
                in
                    (\x1, \y1) -- (\x1, 0);
        \fi
        \draw[\co, fill=white, line width=0.4mm]
            (\intersect) circle (0.75mm);
    }
}


\def\Lupper{\mathcal{L}_{\mathrm{upper}}}
\def\Llower{\mathcal{L}_{\mathrm{lower}}}

\def\pich{6.5cm}
\def\picb{9cm}

\begin{document}

    \begin{tabularx}{\textwidth}{ll}
        % === PAGE 1 ========================================================
        % --- IMAGE 1 -------------------------------------------------------
        \noindent\parbox[c][\pich]{\picb}{
        \begin{tikzpicture}[baseline=0]
            \tiktop

            -- start- / end-points
            \points{
                P1/black,
                P2/black,
                P3/black,
                P4/black,
                P5/black,
                P6/black,
                P7/black,
                P8/black,
                P9/black,
                P10/black,
                P11/black,
                P12/black}

            -- start- / end-point text
            \pointtexts{
                P1/p/1/above right/black,
                P2/p/2/above/black,
                P3/p/3/above/black,
                P4/p/4/above right/black,
                P5/p/5/above left/black, 
                P6/p/6/below right/black,
                P7/p/7/above left/black,
                P8/p/8/right/black,
                P9/p/9/above left/black,
                P10/p/10/above right/black,
                P11/p/11/above left/black,
                P12/p/12/above right/black}
        \end{tikzpicture}} &
        \makecell[l]{A} \\
        % --- IMAGE 2 -------------------------------------------------------
        \noindent\parbox[c][\pich]{\picb}{
        \begin{tikzpicture}[baseline=0]
            \tiktop

            -- start- / end-points
            \points{
                P1/black,
                P2/black,
                P3/black,
                P4/black,
                P5/black,
                P6/black,
                P7/black,
                P8/black,
                P9/black,
                P10/black,
                P11/black,
                P12/black}

            -- start- / end-point text
            \pointtexts{
                P1/p/1/above right/black,
                P2/q/1/above/black,
                P3/p/2/above/black,
                P4/q/2/above right/black,
                P5/p/3/above left/black, 
                P6/q/3/below right/black,
                P7/p/4/above left/black,
                P8/q/4/right/black,
                P9/p/5/above left/black,
                P10/q/5/above right/black,
                P11/p/6/above left/black,
                P12/q/6/above right/black}
        \end{tikzpicture}} &
        \setlength{\tabcolsep}{2pt}
        \makecell[l]{A}
   \end{tabularx}
\end{document}