

%-------------------------------------------------------------------------------
% Dokumenten Klasse
\documentclass[
	final,
	a4paper,
	oneside,
	parskip=full,
	headings=standardclasses,
	headings=big,
	pointednumbers
]{scrartcl}

%-------------------------------------------------------------------------------
% Packete nutzen
\usepackage[T1]{fontenc}
\usepackage[utf8]{inputenc}
\usepackage[left=15mm,right=20mm,top=17mm,bottom=17mm,footskip=0.7cm]{geometry}
\usepackage{amsmath}
\usepackage{amssymb}
\usepackage{mathtools}
\usepackage{mathtools}

%-------------------------------------------------------------------------------
% xcolor
\usepackage[svgnames,table]{xcolor}

%-------------------------------------------------------------------------------
% scrlayer-scrpage
\usepackage{scrlayer-scrpage}
\pagestyle{scrheadings}
\clearpairofpagestyles

\cfoot{\pagemark}

%-------------------------------------------------------------------------------
% tabularx
\usepackage{tabularx}
\usepackage{ltablex}
\usepackage{makecell}

%-------------------------------------------------------------------------------
% TikZ
\usepackage{tikz}
\usetikzlibrary{positioning, arrows, decorations, calc, fit, intersections}

%-------------------------------------------------------------------------------
% Listings
\usepackage{listings}
\newcommand{\listingMatlab}[2][]{
	\lstset{
		language=Matlab,
		breaklines=true,
		numbers=left,
		numberstyle=\tiny,
		numbersep=5pt,
		captionpos=b,
		basicstyle=\footnotesize\ttfamily,
		stringstyle=\color{magenta},
		identifierstyle=\color{black},
		keywordstyle=\color{blue}, 
		commentstyle=\color{DarkGreen}
	}
	\lstinputlisting[caption={\texttt{\detokenize{#2}}},#1]{#2}
}
\lstnewenvironment{algorithm}[1][] %defines the algorithm listing environment
{
    \lstset{ %this is the stype
        mathescape=true,
        frame=tB,
        numbers=left, 
        numberstyle=\tiny,
        basicstyle=\scriptsize, 
        keywordstyle=\color{black}\bfseries,
        keywords={,input, output, return, datatype, function, in, if, else, foreach, while, begin, end, } 
        numbers=left,
        xleftmargin=.04\textwidth
    }
}
{}


%-------------------------------------------------------------------------------

\RedeclareSectionCommand[
  afterindent=false,
  beforeskip=0.8\baselineskip,
  afterskip=0.4\baselineskip]{section}
\RedeclareSectionCommand[
  afterindent=false,
  beforeskip=0.8\baselineskip,
  afterskip=0.4\baselineskip]{subsection}

%-------------------------------------------------------------------------------
% 
\newcommand{\f}[2]{\frac{#1}{#2}}
\newcommand{\fs}[2]{{\tfrac{#1}{#2}}}

% kl = ()
\newcommand{\kl}[1]{{\left( #1 \right)}}

% kq = {}
\newcommand{\kq}[1]{{\left\{ #1 \right\}}}

% ks = []
\newcommand{\ks}[1]{{\left[ #1 \right]}}

% 
\newcommand{\abs}[1]{{\vert #1 \vert}}

\newcommand\addvmargin[1]{
  \node[fit=(current bounding box),inner ysep=#1,inner xsep=0]{};
}

\def\tiktop{%
    \def\xlines{22}
    \def\ylines{15}
    \def\raster{3.5mm}

    \coordinate (P1) at ( 4*\raster,13*\raster);
    \coordinate (P2) at ( 2*\raster, 8*\raster);
    \coordinate (P3) at ( 4*\raster, 5*\raster);
    \coordinate (P4) at (10*\raster, 1*\raster);
    \coordinate (P5) at (17*\raster, 3*\raster);
    \coordinate (P6) at (20*\raster, 6*\raster);
    \coordinate (P7) at (18*\raster,12*\raster);
    \coordinate (P8) at (14*\raster, 7*\raster);

    % draw vertical lines
    \foreach \x in {0,...,\xlines}
    {
        \draw[lightgray] (\x * \raster, 0mm) -- (\x * \raster, \ylines * \raster);
    }

    % draw horizontal lines
    \foreach \y in {0,...,\ylines}
    {
        \draw[lightgray] (0mm, \y * \raster) -- (\xlines * \raster, \y * \raster);
    }

    % border
    \draw[black] (0mm, 0mm) -- (\xlines * \raster, 0mm);
    \draw[black] (0mm, 0mm) -- (0mm, \ylines * \raster);
    \draw[black] (\xlines * \raster, 0mm) -- (\xlines * \raster, \ylines * \raster);
    \draw[black] (0mm, \ylines * \raster) -- (\xlines * \raster, \ylines * \raster);
}

\newcommand{\points}[1]{%
    \foreach \point/\co in {#1}
    {
        \draw[\co,fill=\co,line width=0.4mm]
            (\point) circle (0.5mm);
    }
}

\newcommand{\pointtexts}[1]{%
    \foreach \point/\name/\i/\align/\co in {#1}
    {
        \node[\align, \co, fill=white, inner sep=1.25, outer sep=2] at (\point) {$\scriptstyle \name_{\i} $};
    }
}

\newcommand{\textlines}[1]{%
    \foreach \a/\b/\c/\t/\xs/\ys in {#1}
    {
        \draw[\c, line width=0.3mm]
            let
                \p1 = (\a),
                \p2 = (\b)
            in 
                (\x1,\y1) -- node[midway,
                                  fill=white,
                                  xshift=\xs, yshift=\ys,
                                  inner sep=1.25, outer sep=2] {\scriptsize \t} ++(\x2 - \x1, \y2 - \y1) ;
        
    }
}

\newcommand{\lines}[1]{%
    \foreach \a/\b/\c in {#1}
    {
        \draw[\c, line width=0.3mm]
            (\a) -- (\b);
    }
}

\def\Lupper{\mathcal{L}_{\mathrm{upper}}}
\def\Llower{\mathcal{L}_{\mathrm{lower}}}

\def\pich{6cm}
\def\picb{8cm}

\begin{document}


    % === PAGE 1 ============================================================

    \begin{minipage}[t]{.5\linewidth}
        \section*{Nearest Neighbour}
        
        \subsection*{Calculate Path}
    \end{minipage}
    \begin{minipage}[t]{.5\linewidth}
            \begin{tabular}[t]{@{} r @{\hskip 1mm} c @{\hskip 1mm} l @{}}
                         $S$ & $=$ & $\kq{p_1}$ \\
                         $R$ & $=$ & Set of vertices not yet vistied \\
                $c\kl{S, e}$ & $=$ & length of edge going from last \\
                             &     & vertex of $S$ to $e$
            \end{tabular}
    \end{minipage}

    \begin{tabularx}{\textwidth}{ll}
        % --- IMAGE 1 -------------------------------------------------------
        \noindent\parbox[c][\pich]{\picb}{
        \begin{tikzpicture}[baseline=0]
            \tiktop

            \textlines{
                P1/P2/red/5.4/0/0,
                P1/P3/blue/8.0/0/0,
                P1/P4/blue/13.4/0/0,
                P1/P5/blue/16.4/0/0,
                P1/P6/blue/17.5/0/0,
                P1/P7/blue/14.0/0/0,
                P1/P8/blue/11.7/0/0}

            -- start- / end-points
            \points{
                P1/black,
                P2/black,
                P3/black,
                P4/black,
                P5/black,
                P6/black,
                P7/black,
                P8/black}

            -- start- / end-point text
            \pointtexts{
                P1/p/1/above/black,
                P2/p/2/below left/black,
                P3/p/3/below left/black,
                P4/p/4/below right/black,
                P5/p/5/below right/black, 
                P6/p/6/below right/black,
                P7/p/7/above right/black,
                P8/p/8/below right/black}
        \end{tikzpicture}} &
        \noindent\parbox[c][\pich]{\picb}{
        \begin{tikzpicture}[baseline=0]
            \tiktop

            \textlines{
                P1/P2/black/5.4/0/0,
                P2/P3/red/3.6/-4/0,
                P2/P4/blue/10.6/0/0,
                P2/P5/blue/15.8/0/0,
                P2/P6/blue/18.1/0/-3,
                P2/P7/blue/16.5/0/0,
                P2/P8/blue/12.0/0/2}

            -- start- / end-points
            \points{
                P1/black,
                P2/black,
                P3/black,
                P4/black,
                P5/black,
                P6/black,
                P7/black,
                P8/black}

            -- start- / end-point text
            \pointtexts{
                P1/p/1/above/black,
                P2/p/2/left/black,
                P3/p/3/below left/black,
                P4/p/4/below right/black,
                P5/p/5/below right/black, 
                P6/p/6/below right/black,
                P7/p/7/above right/black,
                P8/p/8/above right/black}
        \end{tikzpicture}} \\
        % --- IMAGE 2 -------------------------------------------------------
        \noindent\parbox[c][\pich]{\picb}{
        \begin{tikzpicture}[baseline=0]
            \tiktop

            \textlines{
                P1/P2/black/5.4/0/0,
                P2/P3/black/3.6/-4/0,
                P3/P4/red/7.2/0/0,
                P3/P5/blue/13.2/0/0,
                P3/P6/blue/16.0/0/0,
                P3/P7/blue/15.7/0/0,
                P3/P8/blue/10.2/0/0}

            -- start- / end-points
            \points{
                P1/black,
                P2/black,
                P3/black,
                P4/black,
                P5/black,
                P6/black,
                P7/black,
                P8/black}

            -- start- / end-point text
            \pointtexts{
                P1/p/1/above/black,
                P2/p/2/left/black,
                P3/p/3/below left/black,
                P4/p/4/below right/black,
                P5/p/5/right/black, 
                P6/p/6/right/black,
                P7/p/7/above right/black,
                P8/p/8/right/black}
        \end{tikzpicture}} &
        \noindent\parbox[c][\pich]{\picb}{
        \begin{tikzpicture}[baseline=0]
            \tiktop

            \textlines{
                P1/P2/black/5.4/0/0,
                P2/P3/black/3.6/-4/0,
                P3/P4/black/7.2/0/0,
                P4/P5/blue/7.3/4/-2,
                P4/P6/blue/11.2/0/0,
                P4/P7/blue/13.6/14/20,
                P4/P8/red/7.2/0/0}

            -- start- / end-points
            \points{
                P1/black,
                P2/black,
                P3/black,
                P4/black,
                P5/black,
                P6/black,
                P7/black,
                P8/black}

            -- start- / end-point text
            \pointtexts{
                P1/p/1/above/black,
                P2/p/2/left/black,
                P3/p/3/below left/black,
                P4/p/4/below right/black,
                P5/p/5/above right/black, 
                P6/p/6/above right/black,
                P7/p/7/above right/black,
                P8/p/8/above left/black}
        \end{tikzpicture}} \\
        % --- IMAGE 3 -------------------------------------------------------
        \noindent\parbox[c][\pich]{\picb}{
        \begin{tikzpicture}[baseline=0]
            \tiktop

            \textlines{
                P1/P2/black/5.4/0/0,
                P2/P3/black/3.6/-4/0,
                P3/P4/black/7.2/0/0,
                P4/P8/black/7.2/0/0,
                P8/P7/blue/6.4/0/0,
                P8/P6/blue/6.1/0/0,
                P8/P5/red/5.0/0/0}

            -- start- / end-points
            \points{
                P1/black,
                P2/black,
                P3/black,
                P4/black,
                P5/black,
                P6/black,
                P7/black,
                P8/black}

            -- start- / end-point text
            \pointtexts{
                P1/p/1/above/black,
                P2/p/2/left/black,
                P3/p/3/below left/black,
                P4/p/4/below right/black,
                P5/p/5/above right/black, 
                P6/p/6/above right/black,
                P7/p/7/above right/black,
                P8/p/8/above left/black}
        \end{tikzpicture}} &
        \noindent\parbox[c][\pich]{\picb}{
        \begin{tikzpicture}[baseline=0]
            \tiktop

            \textlines{
                P1/P2/black/5.4/0/0,
                P2/P3/black/3.6/-4/0,
                P3/P4/black/7.2/0/0,
                P4/P8/black/7.2/0/0,
                P8/P5/black/5.0/0/0,
                P5/P6/red/4.2/0/0,
                P5/P7/blue/9.1/0/0}

            -- start- / end-points
            \points{
                P1/black,
                P2/black,
                P3/black,
                P4/black,
                P5/black,
                P6/black,
                P7/black,
                P8/black}

            -- start- / end-point text
            \pointtexts{
                P1/p/1/above/black,
                P2/p/2/left/black,
                P3/p/3/below left/black,
                P4/p/4/below right/black,
                P5/p/5/below/black, 
                P6/p/6/above/black,
                P7/p/7/above/black,
                P8/p/8/above left/black}
        \end{tikzpicture}} \\
        % --- IMAGE 4 -------------------------------------------------------
        \noindent\parbox[c][\pich]{\picb}{
        \begin{tikzpicture}[baseline=0]
            \tiktop

            \textlines{
                P1/P2/black/5.4/0/0,
                P2/P3/black/3.6/-4/0,
                P3/P4/black/7.2/0/0,
                P4/P8/black/7.2/0/0,
                P8/P5/black/5.0/0/0,
                P5/P6/black/4.2/0/0,
                P6/P7/red/6.3/0/0}

            -- start- / end-points
            \points{
                P1/black,
                P2/black,
                P3/black,
                P4/black,
                P5/black,
                P6/black,
                P7/black,
                P8/black}

            -- start- / end-point text
            \pointtexts{
                P1/p/1/above/black,
                P2/p/2/left/black,
                P3/p/3/below left/black,
                P4/p/4/below right/black,
                P5/p/5/below/black, 
                P6/p/6/right/black,
                P7/p/7/above/black,
                P8/p/8/above left/black}
        \end{tikzpicture}} &
        \noindent\parbox[c][\pich]{\picb}{
        \begin{tikzpicture}[baseline=0]
            \tiktop

            \textlines{
                P1/P2/black/5.4/0/0,
                P2/P3/black/3.6/-4/0,
                P3/P4/black/7.2/0/0,
                P4/P8/black/7.2/0/0,
                P8/P5/black/5.0/0/0,
                P5/P6/black/4.2/0/0,
                P6/P7/black/6.3/0/0,
                P7/P1/black/14.0/0/0}

            -- start- / end-points
            \points{
                P1/black,
                P2/black,
                P3/black,
                P4/black,
                P5/black,
                P6/black,
                P7/black,
                P8/black}

            -- start- / end-point text
            \pointtexts{
                P1/p/1/above/black,
                P2/p/2/left/black,
                P3/p/3/below left/black,
                P4/p/4/below right/black,
                P5/p/5/below/black, 
                P6/p/6/right/black,
                P7/p/7/above/black,
                P8/p/8/above left/black}
        \end{tikzpicture}}
    \end{tabularx}
\end{document}